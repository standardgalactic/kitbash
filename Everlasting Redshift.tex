\documentclass[12pt]{article}

\usepackage[margin=1in]{geometry}
\usepackage{amsmath, amssymb}
\usepackage{setspace}
\usepackage{hyperref}
\usepackage{physics}

\setstretch{1.2}

\title{\textbf{Everlasting Redshift}\\
\large Apparent Expansion in a Relaxing Universe}
\author{Flyxion}
\date{\today}

\begin{document}
\maketitle

\begin{abstract}

Cosmological redshift is conventionally interpreted as direct evidence for the expansion of space and, at late times, for accelerated expansion driven by dark energy. This interpretation relies on a sequence of inferential steps that embed strong assumptions about global isotropy, large-scale homogeneity, metric evolution, and the absence of significant kinematic or evolutionary systematics. While these assumptions are often treated as empirically settled, their necessity remains open to question.

In this essay, we examine whether cosmological observations uniquely require an expanding or accelerating metric, or whether alternative physical interpretations remain viable. Drawing on critiques of supernova-based acceleration claims, entropy-based formulations of physical ordering, and the distinction between kinematic observation and geometric inference, we propose a reframing of cosmological redshift. Rather than treating redshift as a direct signature of metric expansion, we interpret it as a persistent ordering phenomenon arising from relaxation processes in an inhomogeneous, structured universe.

Under this view, redshift remains everlasting without requiring global expansion, accelerated dynamics, or a fundamental cosmological constant. Apparent acceleration emerges as an effective description associated with large-scale smoothing, void dominance, and long-horizon entropy gradients, rather than as evidence for a repulsive energy component.

This reinterpretation does not deny observational data. Instead, it questions the uniqueness of the expansion-based narrative and explores whether entropy, foliation, and observational anisotropy offer a more parsimonious and physically grounded account of the same phenomena.

\end{abstract}

\newpage
\section{Introduction}

The discovery of cosmological redshift in the early twentieth century initiated one of the most influential interpretive frameworks in modern physics. The apparent correlation between redshift and distance, later formalized as Hubble's law, was rapidly understood as evidence for an expanding universe. Over the ensuing decades, this interpretation solidified into a geometric narrative in which space itself stretches, carrying matter and radiation along with it.

In the late 1990s, observations of distant Type Ia supernovae appeared to indicate not merely expansion, but accelerated expansion. This result, awarded the Nobel Prize in Physics in 2011, was widely interpreted as empirical confirmation of a cosmological constant or a dark energy component dominating the universe's large-scale dynamics. The resulting $\Lambda$CDM model achieved remarkable phenomenological success, fitting a broad range of cosmological observations with a small number of parameters.

Yet beneath this success lies a persistent conceptual unease. The cosmological constant problem remains unresolved, with theoretical expectations from quantum field theory exceeding observed values by over one hundred orders of magnitude (Weinberg 1989). Statistical tensions and directional anomalies persist in large-scale datasets (Sarkar et al. 2008; Colin et al. 2019). Moreover, the inference from redshift to expansion depends critically on assumptions about isotropy, homogeneity, and the standardizability of astrophysical sources across cosmic time.

This essay does not argue that observational cosmology is mistaken in its measurements. Instead, it questions whether the dominant interpretation of those measurements is uniquely compelled. Specifically, it asks whether redshift must be understood as evidence of metric expansion, or whether it may instead reflect a more general ordering phenomenon associated with entropy, relaxation, and observational foliation in an evolving universe.

The title \emph{Everlasting Redshift} reflects this shift in emphasis. Redshift is treated as a durable and cumulative feature of observation, not necessarily as a dynamical indicator of spatial growth. By reframing redshift in this way, the need for accelerated expansion and dark energy may be reduced or eliminated, not by denying data, but by revisiting the conceptual structure used to interpret it.

\subsection*{Formal Clarification}

Let $z$ denote the observed redshift of a photon emitted from a source and received by an observer. Observationally, $z$ is defined by the ratio
\[
1 + z = \frac{\lambda_{\text{obs}}}{\lambda_{\text{emit}}}.
\]
This definition is purely kinematic and does not, by itself, specify a mechanism. The standard cosmological interpretation embeds this ratio within a Friedmann--Lemaître--Robertson--Walker (FLRW) metric, yielding
\[
1 + z = \frac{a(t_{\text{obs}})}{a(t_{\text{emit}})},
\]
where $a(t)$ is the cosmological scale factor.

The central question posed in this essay is whether the mapping from the first expression to the second is logically necessary, or whether alternative mechanisms can generate persistent redshift without invoking global metric expansion.

\section{Observation, Isotropy, and the Directionality of Redshift}

The interpretation of cosmological redshift as global expansion relies critically on the assumption of large-scale isotropy. Within the standard framework, the universe is modeled as statistically identical in all directions once local motions are subtracted. Under this assumption, redshift measurements obtained along different lines of sight are treated as samples of a single underlying metric evolution.

However, isotropy is not itself a direct observable. It is an inferred property derived from finite datasets that are unevenly distributed across the sky and subject to selection effects. Supernova surveys, in particular, are constrained by observational practicalities, telescope placement, extinction, and survey strategy. As a result, redshift–distance relations are sampled anisotropically, even before cosmological interpretation is applied.

Several analyses have identified statistically significant directional dependencies in supernova data, suggesting that inferred acceleration is not uniform across the sky (Sarkar et al. 2008; Colin et al. 2019). These anisotropies align approximately with the dipole direction observed in the cosmic microwave background, which is conventionally attributed to the peculiar velocity of the local group. If this motion extends coherently over larger scales, the assumption that all observed redshifts can be cleanly decomposed into local kinematics plus global expansion becomes questionable.

Under such conditions, the inference of acceleration becomes fragile. A directional bias in observed redshifts can masquerade as a change in the global expansion rate when averaged under isotropic assumptions. The distinction between a truly accelerating metric and a kinematic or relaxation-induced anisotropy is therefore not merely philosophical, but observationally consequential.

This ambiguity is often obscured by the language of cosmological fitting. When a Friedmannian model is imposed, residual anisotropies are absorbed into noise terms or treated as systematics rather than as signals indicating a breakdown of the underlying symmetry assumptions. The success of the fit is then taken as confirmation of the model, rather than as evidence of the model’s flexibility.

In an alternative framing, isotropy is not assumed a priori, but treated as an emergent or approximate property. Redshift patterns are then analyzed as directional orderings that may reflect large-scale flows, relaxation gradients, or entropy-aligned foliations rather than uniform expansion. From this perspective, the persistence of redshift does not require that space itself be growing, only that observational relationships remain ordered along preferred directions.


\subsection*{Formal Argument: Directional Redshift and Model Degeneracy}

Let $z(\hat{n})$ denote the observed redshift along direction $\hat{n}$ on the sky. In the standard decomposition, one writes
\[
1 + z(\hat{n}) = (1 + z_{\text{cos}})(1 + z_{\text{pec}}(\hat{n})),
\]
where $z_{\text{cos}}$ is attributed to global expansion and $z_{\text{pec}}$ to local peculiar motion.

This decomposition presupposes that $z_{\text{cos}}$ is isotropic and that $z_{\text{pec}}$ averages to zero over the sky. If instead there exists a coherent bulk flow with velocity $\mathbf{v}_{\text{bulk}}$, then to first order
\[
z_{\text{pec}}(\hat{n}) \approx \frac{\mathbf{v}_{\text{bulk}} \cdot \hat{n}}{c}.
\]

If the survey sampling is anisotropic, the sky average
\[
\langle z(\hat{n}) \rangle_{\text{survey}}
\]
need not equal the true isotropic mean. A directional excess aligned with $\mathbf{v}_{\text{bulk}}$ can bias the inferred value of $z_{\text{cos}}$, leading to an apparent modification of the redshift–distance relation.

Now consider fitting this biased dataset with a Friedmannian luminosity distance
\[
d_L(z) = (1+z) \int_0^z \frac{dz'}{H(z')}.
\]
Any systematic directional contribution to $z$ that grows with distance can be absorbed into an effective $H(z)$ that appears to decrease with time, mimicking acceleration.

Thus, without independent verification of isotropy, the mapping
\[
\text{directional redshift} \;\Rightarrow\; \text{metric acceleration}
\]
is not injective. Multiple physical mechanisms produce observationally similar redshift patterns. Expansion is therefore an interpretation layered atop redshift, not an unavoidable consequence of it.

\section{Statistical Significance and the Fragility of Acceleration}

Claims of cosmic acceleration rest not only on observational data, but on statistical interpretations of that data within specific model frameworks. The confidence with which acceleration is asserted is often expressed in terms of standard deviations, or ``sigma,'' conveying the impression of discovery-level certainty. However, such measures are meaningful only relative to the assumptions embedded in the statistical model.

In observational cosmology, statistical significance is typically computed under idealized conditions: Gaussian noise, independent data points, and correctly specified likelihood functions. These conditions rarely hold in practice. Astrophysical observations are affected by correlated uncertainties, calibration drift, selection bias, and model-dependent corrections. When such effects are incompletely characterized, reported significance levels may overstate the robustness of the underlying inference.

A critical distinction must therefore be drawn between nominal statistical significance and effective significance. Nominal significance reflects the outcome of a fit under assumed conditions, while effective significance accounts for unresolved systematics and model uncertainty. In several re-analyses of supernova datasets, results initially reported as exceeding five standard deviations were found to fall closer to three standard deviations once anisotropy and systematics were incorporated (Sarkar et al. 2008; Nielsen et al. 2016).

This distinction is not merely technical. In high-energy physics and other domains where discovery claims carry high epistemic weight, three-sigma results are regarded as suggestive but insufficient. The history of the field includes numerous examples in which apparent high-significance signals disappeared with improved modeling or additional data. The possibility that cosmic acceleration belongs to this category cannot be dismissed on statistical grounds alone.

Moreover, the persistence of acceleration claims despite declining effective significance suggests a form of model inertia. Once acceleration is incorporated into the cosmological framework, subsequent analyses tend to condition on its existence, refining parameter values rather than reassessing the foundational inference. Statistical machinery thus becomes a tool for parameter estimation rather than hypothesis testing.

In an alternative framing, the weakening of acceleration signals with improved datasets may indicate that the phenomenon being fit is not fundamental. Rather than interpreting this trend as an inconvenience to be managed, it may be more productive to ask whether the statistical structure of the data is more naturally explained by a non-accelerating, directionally ordered universe.

\subsection*{Formal Argument: Nominal vs.\ Effective Sigma}

Let $\mathcal{L}(D \mid \theta, M)$ denote the likelihood of data $D$ given parameters $\theta$ under model $M$. The nominal significance of a result is typically derived from the curvature of $\log \mathcal{L}$ near its maximum, assuming $M$ is correctly specified.

If systematic uncertainties $\{\Delta_i\}$ are present but not fully modeled, the effective likelihood becomes
\[
\mathcal{L}_{\text{eff}}(D \mid \theta) = \int \mathcal{L}(D \mid \theta, \Delta)\, p(\Delta)\, d\Delta.
\]
Neglecting this marginalization leads to an artificially narrow likelihood and inflated confidence intervals.

In supernova cosmology, corrections for light-curve shape, color, host galaxy properties, and redshift evolution introduce correlated uncertainties. If these correlations are underestimated, the Fisher information matrix overestimates parameter constraints, yielding nominal significances that exceed the effective information content of the data.

Let $\sigma_{\text{nom}}$ denote the standard deviation inferred under idealized assumptions and $\sigma_{\text{eff}}$ the true uncertainty. When $\sigma_{\text{eff}} > \sigma_{\text{nom}}$, a reported $n\sigma$ detection corresponds in reality to
\[
n_{\text{eff}} = n \frac{\sigma_{\text{nom}}}{\sigma_{\text{eff}}}.
\]
Even modest increases in $\sigma_{\text{eff}}$ reduce $n_{\text{eff}}$ substantially. A nominal $5\sigma$ result may therefore degrade to $3\sigma$ or less once systematics are properly incorporated.

Since acceleration is inferred from subtle deviations in the redshift–distance relation, it is particularly sensitive to such effects. Statistical confidence alone, absent model-independent confirmation, is insufficient to establish accelerated expansion as a physical necessity.

\section{Standard Candles, Evolution, and the Illusion of Acceleration}

Type Ia supernovae occupy a privileged position in observational cosmology due to their use as ``standardizable candles.'' After empirical corrections for light-curve width and color, their peak luminosities are treated as sufficiently uniform to serve as distance indicators across cosmological scales. The inference of accelerated expansion depends critically on the stability of this standardization across redshift.

However, the standard candle assumption is not derived from first principles. It is an empirical regularity established within a limited range of environments and cosmic epochs. As observations extend to higher redshifts, the physical conditions under which supernovae occur change systematically. Progenitor metallicity, stellar age distributions, and host galaxy properties evolve with cosmic time, introducing the possibility of intrinsic luminosity drift.

Recent studies have identified correlations between supernova absolute magnitude and properties of the host stellar population, particularly progenitor age (Kang et al. 2020). Younger progenitor systems, which dominate at higher redshift, appear to produce systematically fainter supernovae even after standard corrections. If unaccounted for, this effect biases distance estimates upward, making distant supernovae appear farther away than they are.

This bias produces precisely the observational signature attributed to cosmic acceleration. Dimmer-than-expected supernovae at high redshift are interpreted as evidence that the universe has expanded more rapidly than predicted by a decelerating model. Yet the same pattern arises if the intrinsic brightness of supernovae decreases with lookback time.

The difficulty lies not in demonstrating the existence of such evolutionary effects, but in disentangling them from cosmological inference. Once acceleration is assumed, residual trends are attributed to nuisance parameters. Conversely, if evolution is permitted to play a dominant role, the necessity of accelerated expansion diminishes.

This ambiguity reveals a deeper structural issue. Cosmological interpretation relies on astrophysical uniformity across epochs that are known to differ markedly in their physical conditions. The success of standardization techniques may reflect their flexibility rather than their fidelity. In such circumstances, the apparent discovery of acceleration may represent not a new physical force, but a misattribution of evolutionary ordering to geometric expansion.

\subsection*{Formal Argument: Evolutionary Bias in Distance Moduli}

Let the observed distance modulus $\mu$ be given by
\[
\mu = m - M,
\]
where $m$ is the apparent magnitude and $M$ the assumed absolute magnitude after standardization.

In standard analyses, $M$ is treated as constant. Suppose instead that $M$ depends on a latent evolutionary parameter $\tau$, representing progenitor age or related properties, such that
\[
M(\tau) = M_0 + \delta M(\tau).
\]

If $\tau$ correlates with redshift, then $\delta M(\tau(z)) \neq 0$. The inferred luminosity distance becomes
\[
d_L^{\text{inf}}(z) = d_L^{\text{true}}(z)\, 10^{\delta M(\tau(z))/5}.
\]

For $\delta M(\tau(z)) < 0$ at high redshift, the inferred distance exceeds the true distance. When fitting $d_L(z)$ with a Friedmannian model, this excess is absorbed into the cosmological parameters, producing an effective equation of state with $w < -1/3$.

Thus, luminosity evolution induces a degeneracy:
\[
\text{supernova evolution} \;\longleftrightarrow\; \text{accelerated expansion}.
\]

Without independent constraints on $\delta M(\tau)$, acceleration cannot be uniquely inferred. The redshift signal persists, but its interpretation as expansion becomes contingent rather than necessary.

\section{Entropy, Ordering, and Redshift Without Expansion}

Redshift is commonly treated as a geometric phenomenon: wavelengths are stretched as space itself expands. This interpretation embeds redshift within a dynamical metric framework, assigning it a causal role in the evolution of spacetime. Yet redshift, as an observable, is fundamentally a relational quantity. It encodes a comparison between emission and observation, not a direct measurement of spatial growth.

An alternative perspective emerges when entropy is treated not as a substance or a measure of disorder, but as an ordering relation over physical states. In this formulation, entropy characterizes accessibility and irreversibility, defining which states can follow from others under the allowed dynamics of a system (Carathéodory 1909; Lieb and Yngvason 1999; Kycia 2022).

From this standpoint, temporal evolution is not primary. What appears as time is a parametrization of ordered change. Crucially, ordering does not require expansion. Systems can exhibit persistent directional asymmetry—irreversibility, relaxation, spectral redshifting—without increasing in spatial extent.

Redshift may be interpreted in this light. Rather than signaling the stretching of space, it may reflect the cumulative effect of irreversible processes acting on propagating radiation as it traverses an evolving, inhomogeneous universe. Photons emitted earlier in an ordering sequence arrive systematically redshifted relative to those emitted later, not because space has grown, but because the universe occupies a different entropic stratum.

This reframing dissolves the apparent need for acceleration. If redshift encodes ordering rather than expansion, then its persistence does not require the scale factor to grow ever more rapidly. Instead, redshift is everlasting because ordering is everlasting: once a system evolves irreversibly, earlier states remain distinguishable from later ones.

Such an interpretation aligns naturally with observational anisotropies and evolutionary effects. Entropic ordering need not be isotropic, nor uniform across scales. It accommodates directional flows, relaxation gradients, and local inhomogeneities without violating observational consistency. Expansion becomes one possible interpretation among others, rather than a compulsory conclusion.

\subsection*{Formal Argument: Redshift as an Ordering Relation}

Let physical states be elements of a state space $\mathcal{S}$ equipped with an entropy-induced preorder $\prec$, such that
\[
X \prec Y \quad \text{if and only if} \quad Y \text{ is adiabatically accessible from } X.
\]

In this framework, entropy $S$ is a monotone function respecting this ordering:
\[
X \prec Y \;\Rightarrow\; S(X) \leq S(Y).
\]

Consider a photon emitted at state $X \in \mathcal{S}$ and observed at state $Y \in \mathcal{S}$ with $X \prec Y$. Let $\nu(X)$ and $\nu(Y)$ denote the characteristic frequencies measured relative to local matter fields at emission and observation, respectively.

If the propagation of radiation preserves ordering but not local calibration—due to relaxation of reference frames, matter-field coupling, or large-scale inhomogeneity—then it is consistent to have
\[
\nu(Y) < \nu(X),
\]
yielding a redshift
\[
1 + z = \frac{\nu(X)}{\nu(Y)} > 1.
\]

This inequality does not require metric expansion. It requires only that frequency ratios encode relative placement within the entropy-ordered foliation of $\mathcal{S}$. The redshift thus reflects the irreversibility of the ordering relation rather than the growth of spatial distances.

In contrast, the standard cosmological interpretation identifies the entropy parameter with coordinate time and the ordering with a scale factor $a(t)$. This identification is sufficient but not necessary. Ordering can persist even when spatial measures remain bounded.

Therefore, redshift is compatible with a universe that relaxes through successive entropic strata without expanding. The observational fact of redshift is preserved; its geometric interpretation is not uniquely fixed.

\section{Foliation, Relaxation, and the Misidentification of Expansion}

In relativistic cosmology, spacetime is typically foliated by hypersurfaces of constant cosmological time. These slices are treated as physically meaningful simultaneity surfaces on which spatial geometry evolves according to a scale factor. Expansion, in this picture, is encoded in the changing metric relations between successive slices.

However, foliation is not unique. General relativity admits infinitely many valid slicings of spacetime, and physical interpretation depends critically on which foliation is chosen. The conventional cosmological foliation privileges homogeneity and isotropy, aligning temporal ordering with spatial growth. This alignment is mathematically convenient, but it is not mandated by observation.

When entropy is treated as a fundamental ordering principle, foliation acquires a different interpretation. Instead of slicing spacetime by coordinate time, one may slice the state space of the universe by entropy level sets. Each hypersurface then represents a class of states that are mutually accessible under reversible transformations, while transitions between hypersurfaces represent irreversible relaxation.

Under such a foliation, the arrow of time is identified with increasing entropy, not with metric expansion. Spatial relations need not change monotonically across entropy slices. What changes is the accessibility structure of physical processes and reference frames. Observables such as redshift then encode transitions between entropy-ordered hypersurfaces rather than distances between expanding spatial sections.

The misidentification arises when entropy-ordered foliation is conflated with metric evolution. If one assumes that successive entropy slices must correspond to increasing spatial separation, redshift is immediately interpreted as expansion. Yet this assumption smuggles geometry into what is fundamentally an ordering relation. Relaxation need not entail growth; it entails loss of constraint.

This distinction clarifies why redshift persists even in models where spatial volume remains bounded. It also explains why attempts to localize dark energy as a physical substance encounter conceptual difficulty. What is attributed to an energy density driving expansion may instead reflect the cumulative effect of irreversible ordering across foliations.

Thus, expansion emerges not as an observed necessity, but as a coordinate-dependent interpretation imposed on an entropy-driven ordering structure. Recognizing this distinction opens the possibility of cosmological models in which redshift is everlasting while space itself does not grow.

\subsection*{Formal Argument: Entropy-Induced Foliation}

Let $\mathcal{M}$ denote spacetime and $\mathcal{S}$ the corresponding physical state space. Let $S : \mathcal{S} \rightarrow \mathbb{R}$ be an entropy function inducing a preorder on $\mathcal{S}$.

Define entropy hypersurfaces
\[
\Sigma_\sigma = \{ X \in \mathcal{S} \mid S(X) = \sigma \}.
\]
These hypersurfaces partition $\mathcal{S}$ into equivalence classes under reversible transformations.

A foliation of $\mathcal{M}$ is then induced by the mapping from physical states to spacetime events. This foliation need not coincide with hypersurfaces of constant cosmological time $t$. Instead, ordering is defined by the relation
\[
X \prec Y \quad \Leftrightarrow \quad S(X) < S(Y).
\]

Let $\gamma$ be a null geodesic representing photon propagation from emission event $e$ to observation event $o$. The redshift measured along $\gamma$ depends on the relative calibration of clocks and rulers at $e$ and $o$, which are functions of their placement within the entropy foliation.

If entropy increases monotonically along $\gamma$, then frequency ratios encode this ordering:
\[
1 + z = \exp\left( \int_\gamma d\ln \nu \right),
\]
where $\nu$ is locally defined relative to matter fields. This integral need not correspond to an expanding metric; it requires only nontrivial evolution of local reference structures across entropy slices.

The standard identification
\[
\Sigma_\sigma \equiv \{ t = \text{const} \}
\]
is therefore an additional assumption. When imposed, redshift is reinterpreted as expansion. When relaxed, redshift remains but expansion is no longer compulsory.

Thus, the foliation underlying cosmological interpretation determines whether redshift is attributed to geometry or to ordering. Expansion is one choice of foliation, not an empirical inevitability.

\section{The Cosmological Constant as a Category Error}

The cosmological constant occupies a peculiar position in modern physics. Introduced originally as a modification of Einstein’s field equations, it has been reinterpreted repeatedly: as a geometric term, a vacuum energy density, a dynamical fluid, and a proxy for dark energy. Each reinterpretation preserves its mathematical role while shifting its physical meaning.

The persistent difficulty lies not in fitting observational data, but in reconciling $\Lambda$ with theoretical expectations. Quantum field theory predicts a vacuum energy density exceeding the observed value of the cosmological constant by approximately $10^{120}$, a discrepancy often described as the worst prediction in the history of physics (Weinberg 1989). Despite decades of effort, no consensus explanation has emerged.

From the perspective developed in this essay, this discrepancy signals a category error rather than a missing ingredient. The cosmological constant is invoked to explain an apparent acceleration inferred from redshift data under an expansion-based interpretation. If redshift instead reflects entropic ordering and relaxation, then the acceleration it appears to encode is not dynamical in origin. In such a case, introducing an energy density to drive expansion is a response to a misinterpretation rather than to a physical deficiency.

Vacuum energy calculations assume that zero-point fluctuations gravitate in the same manner as conventional energy densities. Yet these calculations presuppose that the cosmological constant represents a physical substance rather than a geometric bookkeeping term. If $\Lambda$ is compensating for an inappropriate identification of ordering with expansion, its unnatural smallness is no longer mysterious. It is not small because of fine-tuning, but because it is not fundamentally physical.

This reframing also clarifies why dark energy has resisted direct detection. Despite extensive observational effort, no independent evidence for a dynamical dark energy field has emerged beyond its inferred role in cosmological fits. In contrast, ordering and relaxation processes are ubiquitous features of physical systems and require no new ontological commitments.

Treating the cosmological constant as a category error shifts the explanatory burden. Instead of asking why vacuum energy is suppressed, one asks why entropy-driven ordering was reinterpreted as metric acceleration in the first place. The former question invites speculative microphysics; the latter invites conceptual correction.

\subsection*{Formal Argument: Misattribution of Ordering to Energy Density}

Einstein’s field equations with a cosmological constant are
\[
G_{\mu\nu} + \Lambda g_{\mu\nu} = 8\pi G T_{\mu\nu}.
\]

In the standard cosmological interpretation, $\Lambda$ is rewritten as an effective energy-momentum tensor
\[
T^{(\Lambda)}_{\mu\nu} = -\frac{\Lambda}{8\pi G} g_{\mu\nu},
\]
corresponding to a uniform energy density $\rho_\Lambda$ with negative pressure.

This reinterpretation assumes that the observed deviation in the redshift–distance relation corresponds to additional stress-energy driving expansion. However, if the deviation instead arises from entropy-ordered relaxation effects, then no additional stress-energy is required.

Let $\mathcal{O}$ denote the observed redshift relation and $\mathcal{M}_\Lambda$ a model incorporating $\Lambda$. The inference
\[
\mathcal{O} \Rightarrow \rho_\Lambda > 0
\]
holds only under the assumption that $\mathcal{O}$ encodes metric acceleration. If $\mathcal{O}$ is equally compatible with an ordering-based model $\mathcal{M}_S$ in which spatial geometry remains bounded, then $\rho_\Lambda$ is not uniquely determined by observation.

In this case, $\Lambda$ functions as a compensator absorbing discrepancies introduced by an inappropriate mapping between ordering and geometry. Its numerical smallness reflects the fact that it is correcting a subtle interpretive mismatch rather than representing a fundamental energy scale.

Thus, the cosmological constant problem dissolves when $\Lambda$ is recognized not as a physical fluid, but as a parameter enforcing an expansion-based interpretation of redshift that is not observationally compulsory.

\section{Relaxation as a Global Principle Without Growth}

Physical systems generically evolve toward states of reduced constraint. This process is commonly described as relaxation. In thermodynamics, relaxation manifests as the redistribution of gradients, the dissipation of anisotropies, and the enlargement of the space of accessible microstates. Importantly, relaxation does not entail spatial expansion. A system may relax while remaining spatially bounded, conserving volume even as its internal ordering changes.

When applied to cosmology, relaxation offers an alternative global principle to expansion. Rather than interpreting large-scale evolution as the growth of space, one may view it as the progressive relaxation of initially constrained configurations. Early cosmic conditions need not be compact in a geometric sense; they need only be highly constrained in their degrees of freedom. As these constraints relax, observable relationships change even if spatial extent does not.

Under this view, redshift arises naturally. Photons emitted under tighter constraints are observed under looser ones, producing systematic frequency shifts without invoking metric stretching. The universe appears different not because it has grown larger, but because it has relaxed into a broader region of its state space.

This framing also clarifies why redshift accumulates monotonically. Relaxation is irreversible. Once constraints are released, they are not spontaneously reimposed. Redshift therefore persists and compounds, even in the absence of expansion. The phenomenon is everlasting not because space grows indefinitely, but because ordering proceeds unidirectionally.

Calling this picture a \emph{relaxing universe} does not introduce a new force or field. It names a global organizational tendency already present in physical law. What changes is the interpretation of cosmological observables. Expansion becomes one possible parametrization of relaxation, not its physical essence.

This distinction matters because it separates what must be explained from what is chosen for convenience. If relaxation is fundamental, then cosmological models should encode constraint release and ordering explicitly rather than inferring them indirectly through expansion dynamics. Failure to do so risks mistaking bookkeeping artifacts for physical drivers.

\subsection*{Formal Argument: Relaxation and Monotonic Redshift}

Let $\Omega(t)$ denote the effective phase-space volume accessible to the universe at an ordering parameter $t$, where $t$ need not correspond to coordinate time. Relaxation implies
\[
\frac{d\Omega}{dt} \geq 0.
\]

Let $\nu(t)$ be the characteristic frequency scale against which photon energies are measured at ordering level $t$. If calibration of physical clocks and rulers depends on accessible phase-space structure, then $\nu(t)$ may evolve with $\Omega(t)$.

Assume a monotone relationship
\[
\nu(t) = \nu_0 f(\Omega(t)),
\]
with $f'(\Omega) < 0$. Then for emission at $t_e$ and observation at $t_o > t_e$,
\[
1 + z = \frac{\nu(t_e)}{\nu(t_o)} = \frac{f(\Omega(t_e))}{f(\Omega(t_o))} > 1.
\]

This redshift arises solely from relaxation. No reference to spatial expansion or a scale factor is required. The condition for everlasting redshift is simply that relaxation continues, not that space grows.

In contrast, standard cosmology identifies $\Omega(t)$ implicitly with a volume factor $a^3(t)$, thereby reifying relaxation as expansion. This identification is sufficient but not necessary. Relaxation without growth remains observationally viable.

\section{Radiative Propagation, Absorption, and the Limits of Geodesic Idealization}

Cosmological interpretations of redshift typically rely on an idealized picture of light propagation in which photons travel along null geodesics through an effectively empty spacetime, conserving their identity until detection. In this picture, wavelength stretching is attributed exclusively to metric expansion. While mathematically convenient, this representation suppresses essential features of electromagnetic radiation as a field.

Light is not a point particle but a propagating excitation of a field. Realistic radiative solutions to Maxwell’s equations exhibit transverse spread, phase-front curvature, and statistical intensity profiles. Even in nominal vacuum, electromagnetic waves are subject to geometric dilution and probabilistic interaction with matter fields. The inverse-square law itself reflects not a conserved ray but a dispersive field structure.

Crucially, indefinite straight-line propagation without interaction is an idealization. The universe is not transparent in the absolute sense required by this model. Matter fields, however dilute, permeate spacetime, enabling absorption and re-emission processes. Propagation therefore occurs not along immutable trajectories but along paths that maximize survivability against reabsorption. These paths need not coincide with strict geodesics in the naive sense.

This perspective differs fundamentally from so-called tired light models. No dissipative energy loss mechanism is posited. Instead, redshift arises from the relational character of emission and detection across evolving reference structures. The photon detected is not the same microscopic object as the photon emitted, but a statistically coherent successor shaped by the intervening environment.

Relativistic considerations reinforce this view. Any mass defines a causal capture structure, however small. While the Schwarzschild radius of a proton is negligible in practical terms, the conceptual lesson remains: escape without interaction is not guaranteed by principle, only approximated in limiting cases. Light propagation is always defined relative to absorption horizons and interaction cross-sections, not absolute emptiness.

Under these conditions, redshift may accumulate without invoking expansion. As radiation propagates through an environment characterized by irreversible ordering and relaxation, frequency ratios encode this ordering. Redshift reflects not fatigue, but survivorship through an entropically structured medium.

\subsection*{Formal Argument: Radiative Survival and Frequency Ordering}

Let $I(r)$ denote the intensity of a radiative field emitted from a localized source. For realistic wave packets,
\[
I(r) \propto \frac{1}{r^2} \exp(-\tau(r)),
\]
where $\tau(r)$ is an optical depth encoding absorption probability along the propagation path.

Let $\nu_e$ be the emission frequency defined relative to local matter fields at the source, and $\nu_o$ the observed frequency defined relative to detector calibration. If absorption and re-emission occur probabilistically, then detected radiation represents a conditional ensemble of surviving modes.

Define a survival-weighted frequency expectation
\[
\langle \nu \rangle(r) = \frac{\int \nu \, P(\nu, r)\, d\nu}{\int P(\nu, r)\, d\nu},
\]
where $P(\nu, r)$ incorporates environmental coupling and ordering effects.

If the environment evolves irreversibly such that reference calibrations shift monotonically with ordering parameter $\sigma$, then
\[
\frac{d\langle \nu \rangle}{d\sigma} < 0,
\]
yielding a systematic redshift with propagation distance.

This effect does not require metric expansion or energy dissipation. It requires only that radiation propagates as a field through an environment whose ordering changes irreversibly. The redshift thus reflects relational evolution rather than geometric stretching.

\section{Why Entropic Redshift Is Not ``Tired Light''}

Historical ``tired light'' hypotheses proposed that photons gradually lose energy through dissipative processes as they traverse space. Such models typically invoked friction-like interactions, inelastic scattering, or ad hoc decay mechanisms. These proposals were rightly rejected because they violated well-tested constraints, including the preservation of spectral line shapes, the absence of image blurring, and the observed time dilation of distant supernova light curves.

The framework developed here differs categorically from tired light models. No dissipative loss mechanism is assumed, and no photon is treated as an isolated particle gradually shedding energy. Instead, redshift is understood as a relational phenomenon arising from the propagation of a radiative field through an evolving environment. Energy conservation is respected locally at all stages. What changes is the calibration context relative to which frequency is defined.

In tired light models, a photon is assumed to persist as the same physical object from emission to detection, with its intrinsic energy monotonically decreasing. In contrast, the present framework treats detected radiation as a statistically coherent successor of emitted radiation, shaped by absorption, re-emission, and environmental ordering. The detected photon is not the original photon, but a field-consistent continuation conditioned on survivability and coherence.

This distinction is crucial. Absorption and re-emission processes do not blur images or distort spectra when coherence is preserved statistically, as in standard radiative transfer theory. Nor do they eliminate cosmological time dilation, which emerges naturally from ordering-based foliations rather than from cumulative energy loss.

The failure of tired light models therefore does not constrain the present approach. Those models erred by treating redshift as intrinsic photon fatigue. The entropic interpretation treats redshift as a property of relational measurement across irreversible ordering. The observational consequences are fundamentally different.

\subsection*{Formal Distinction from Tired Light}

Let $E_\gamma$ denote photon energy in a tired light model. Such models posit
\[
\frac{dE_\gamma}{dx} < 0,
\]
with $E_\gamma$ decreasing continuously along a trajectory $x$. This implies spectral broadening and violation of detailed balance unless finely tuned.

In contrast, let $\nu_e$ and $\nu_o$ denote frequencies defined relative to emission and observation environments, respectively. The entropic framework does not assert
\[
\nu_o < \nu_e \quad \text{for a single conserved photon},
\]
but rather
\[
\nu_o = \nu_e \circ \mathcal{T},
\]
where $\mathcal{T}$ is a transformation induced by environmental ordering, absorption, and re-emission processes that preserve local conservation laws.

Energy is conserved at each interaction vertex. Redshift arises from comparing frequencies defined on different entropy-ordered hypersurfaces, not from decay along a worldline. Consequently, spectral integrity and coherence constraints are satisfied.

\section{Radiative Transfer, Survival Bias, and Redshift Accumulation}

Radiative transfer theory already departs from the idealized picture of immutable photon trajectories. In realistic settings, radiation propagates through media characterized by absorption coefficients, emission sources, and scattering kernels. Even when interactions are rare, their cumulative statistical effects shape the observed radiation field.

Cosmological propagation is typically treated as a limiting case of zero optical depth. However, zero is never attained in practice. Over cosmological distances, even vanishingly small interaction probabilities accumulate. The radiation that reaches an observer is therefore a conditioned subset of all emitted radiation, selected by survivability and coherence constraints.

This conditioning introduces a form of survival bias. Modes that remain phase-coherent and compatible with the evolving environment are preferentially observed. As the universe relaxes and reference structures evolve, the statistical properties of surviving radiation shift accordingly. Frequency ratios measured at detection reflect this conditioning, not exhaustion of individual photons.

Under this interpretation, redshift accumulates not through friction but through selection across entropy-ordered environments. The effect is monotonic, preserves spectral integrity, and is compatible with observed time dilation. It is therefore observationally indistinguishable from expansion-based redshift at leading order, while differing fundamentally in interpretation.

Importantly, this mechanism does not predict additional blurring, scattering halos, or frequency-dependent distortions beyond those already modeled in radiative transfer. It operates through the same formal machinery, extended to cosmological scales.

\subsection*{Formal Argument: Survival-Weighted Propagation}

Let $I_\nu(s)$ be the specific intensity along affine parameter $s$. The radiative transfer equation is
\[
\frac{dI_\nu}{ds} = -\alpha_\nu I_\nu + j_\nu,
\]
where $\alpha_\nu$ is the absorption coefficient and $j_\nu$ the emissivity.

Over cosmological distances, the observed intensity is
\[
I_\nu^{\text{obs}} = \int e^{-\tau_\nu(s)} j_\nu(s)\, ds,
\]
with optical depth
\[
\tau_\nu(s) = \int_0^s \alpha_\nu(s')\, ds'.
\]

If $\alpha_\nu$ and $j_\nu$ depend implicitly on an entropy-ordering parameter $\sigma$, then the detected spectrum reflects the ordering of environments rather than intrinsic photon decay. Frequency shifts arise from comparing emission and detection across differing $\sigma$, not from energy loss along $s$.

Thus, redshift is compatible with standard radiative transfer and does not invoke any tired-light mechanism.

\section{Entropic Gravity, Large-Scale Smoothing, and the Residual Interpretation of Dark Energy}

Independent of cosmological acceleration claims, several lines of theoretical work have suggested that gravity itself may be an emergent, entropic phenomenon rather than a fundamental interaction. Beginning with black hole thermodynamics and the derivation of Einstein’s equations from horizon entropy considerations (Jacobson 1995), and extending through holographic and information-theoretic approaches (Padmanabhan 2010; Verlinde 2011), these models reinterpret gravitational attraction as a macroscopic consequence of microscopic degrees of freedom seeking higher entropy configurations.

In such frameworks, gravity acts to reduce gradients and smooth distributions of matter and information. At galactic scales, this smoothing manifests as effective modifications to Newtonian dynamics, reproducing phenomenology typically attributed to dark matter. At still larger scales, particularly across cosmic voids, the same entropic tendency operates over vastly extended regions.

From this perspective, what is conventionally labeled dark energy may be reinterpreted as the large-scale residual of entropic gravity doing its work. As matter distributions relax and voids become increasingly uniform, the cumulative entropic pressure associated with horizon-scale degrees of freedom alters relational measures such as redshift and inferred distance. The universe appears to accelerate not because space itself is expanding, but because structure is smoothing.

This interpretation aligns naturally with the relaxation framework developed in this essay. Entropic gravity supplies a concrete mechanism for relaxation without growth. It does not require a new energy component filling space; rather, it reflects how information, entropy, and geometry reorganize as constraints are progressively released. The observed acceleration then emerges as a global bookkeeping effect when local ordering processes are misinterpreted as metric dynamics.

Importantly, this view does not require rejecting general relativity. Jacobson’s derivation shows that Einstein’s equations already encode thermodynamic behavior. Verlinde’s proposal extends this logic, suggesting that gravitational phenomena arise from changes in entropy associated with information stored on holographic screens. When applied to cosmology, these ideas imply that large-scale behavior should be interpreted statistically rather than dynamically.

In particular, voids play a central role. As underdense regions grow smoother, the entropic contribution associated with their horizons increases relative to area-law expectations. When this effect is projected onto a homogeneous Friedmannian model, it appears as a uniform repulsive component—a cosmological constant. Yet in the entropic picture, no such component exists independently. The effect is emergent, relational, and scale-dependent.

Thus, dark energy may be understood not as a physical fluid driving expansion, but as the residual signature of entropic gravity acting across the largest available scales. The universe is not expanding faster; it is becoming smoother.

\subsection*{Formal Argument: Entropic Gravity and Apparent Acceleration}

In entropic gravity, the force $F$ arises from an entropy gradient according to
\[
F = T \frac{dS}{dx},
\]
where $T$ is an effective temperature associated with horizon degrees of freedom (Verlinde 2011).

For a holographic screen of area $A = 4\pi r^2$, the number of degrees of freedom scales as
\[
N = \frac{A}{\ell_P^2}.
\]
Applying equipartition,
\[
E = \frac{1}{2} N k_B T,
\]
and identifying $E = Mc^2$, one recovers Newtonian gravity at small scales.

At galactic and intergalactic scales, however, additional entropy contributions arise due to the presence of horizons associated with voids and large-scale structure. These contributions scale with volume rather than area, modifying the effective entropic force. The resulting acceleration scale naturally aligns with the MOND threshold and with cosmological acceleration scales without invoking dark matter or dark energy as substances.

When such entropic effects are averaged under an assumption of homogeneity, they enter the Friedmann equations as an effective cosmological constant:
\[
\Lambda_{\text{eff}} \sim \frac{1}{c^2} \frac{d^2 S}{dV^2}.
\]
This term reflects the second derivative of entropy with respect to volume, not a vacuum energy density.

Therefore, the appearance of dark energy is a consequence of projecting entropic smoothing onto a metric-expansion framework. In a relaxation-based interpretation, the same observations are accounted for by entropy-driven reorganization of structure, with no need for accelerated expansion of space itself.

\section{Void Statistics and the Large-Scale Signature of Relaxation}

Cosmic voids occupy the majority of the universe’s volume and therefore dominate its large-scale geometric and thermodynamic properties. In standard cosmology, voids are treated primarily as passive consequences of expansion and structure formation, with their growth attributed to differential gravitational collapse in an expanding background. In a relaxation-based framework, however, voids assume a more active explanatory role.

If the universe is not expanding metrically but instead relaxing entropically, voids are not regions being pulled apart by space itself. Rather, they are regions in which matter, radiation, and information gradients have progressively diminished. Their apparent growth reflects the smoothing of density contrasts rather than an increase in global scale.

This distinction is subtle observationally but profound conceptually. Void statistics—such as size distributions, shape anisotropies, and redshift-dependent density profiles—are usually interpreted within a Friedmann–Lemaître–Robertson–Walker background. Under that assumption, increasing void sizes with redshift are taken as evidence of accelerating expansion. In a relaxation picture, the same statistics arise naturally from entropy-driven redistribution, without requiring any change in the underlying metric scale.

Importantly, voids provide a natural arena for apparent acceleration effects. Light propagating across extended underdense regions experiences cumulative interactions with the surrounding gravitational and entropic structure. These interactions do not “tire” light in the classical sense, but they do modify phase relationships, energy distributions, and effective distance measures in a way that is strongly correlated with void geometry. When aggregated across cosmological distances, this produces a redshift–distance relation that mimics acceleration.

Recent observational work has already revealed tensions between void statistics and ΛCDM predictions, including anomalous alignments, unexpected depth profiles, and large-scale bulk flows. Within an expansion framework, these are often treated as statistical flukes or systematics. Within a relaxation framework, they are expected signatures of a universe approaching a smoother entropic configuration.

Thus, voids are not merely empty space between structures; they are the dominant contributors to the universe’s large-scale phenomenology. Treating them as passive artifacts of expansion obscures their role as active agents of relaxation.

\subsection*{Formal Consideration: Void-Dominated Entropic Averaging}

Let $\rho(\mathbf{x},t)$ denote the matter density field and define the coarse-grained entropy functional
\[
S(t) = - \int_V \rho \ln \rho \, dV,
\]
where $V$ is a comoving volume dominated by void regions.

As relaxation proceeds, density gradients decay and $\rho \to \bar{\rho}$ almost everywhere except within bound structures. The entropy increase satisfies
\[
\frac{dS}{dt} \ge 0,
\]
with the dominant contribution arising from underdense regions due to their overwhelming volume fraction.

Define an effective redshift contribution accumulated along a null geodesic $\gamma$ as
\[
z_{\text{eff}} \sim \int_\gamma \nabla S \cdot d\ell,
\]
where $d\ell$ parametrizes the light path through void-dominated space.

As voids smooth, $\nabla S$ decreases locally but extends over larger spatial domains, yielding an integrated effect that grows with distance. When interpreted within a homogeneous expanding metric, this contribution enters the luminosity distance relation as an apparent acceleration term.

Thus, void statistics encode relaxation dynamics directly. The appearance of accelerated expansion emerges from entropy-weighted averaging over void-dominated volumes rather than from a dynamical growth of spacetime itself.

\section{Observational Consequences: Redshift, Lensing, and Distance Measures Without Expansion}

A relaxation-based cosmology must ultimately be evaluated not by its conceptual coherence but by its ability to reproduce and reinterpret observational phenomena traditionally attributed to cosmic expansion. Chief among these are cosmological redshift, gravitational lensing, and the inferred distance measures derived from standard candles and rulers. The claim advanced here is not that these observations are incorrect, but that their conventional interpretation presupposes a dynamical expansion that may not be required.

In the standard framework, redshift is understood as the stretching of photon wavelengths due to the expansion of spacetime. This interpretation relies on the assumption that null geodesics propagate through a metric whose scale factor increases monotonically with cosmic time. In a relaxation framework, redshift instead reflects cumulative interactions between propagating radiation and a spatially extended, entropy-dominated field. Light does not lose energy through dissipation or scattering, nor does it curve arbitrarily through space. Rather, its propagation is constrained by a field geometry in which perfectly inertial, infinitely extended straight-line motion is not physically realizable over cosmological distances.

Electromagnetic radiation propagates as a radiative field with finite spatial support and Gaussian falloff. Even in vacuum, it remains coupled to the surrounding gravitational and informational structure. Over sufficiently large distances, this coupling produces a gradual reweighting of phase and energy distributions that manifests observationally as redshift. The effect is directional, cumulative, and scale-dependent, aligning naturally with observed anisotropies and bulk flow correlations.

Gravitational lensing provides a second critical test. In expansion-based cosmology, lensing is treated as a local curvature effect superimposed on an expanding background. In a relaxation picture, lensing arises from entropy gradients associated with bound structures embedded in an otherwise smoothing field. Light trajectories are not bent because spacetime is expanding, but because gradients in relaxation rate induce effective refractive indices in the radiative field. This reproduces observed lensing profiles without invoking additional unseen mass components.

Distance measures inferred from Type Ia supernovae similarly admit reinterpretation. The apparent dimming of distant supernovae is conventionally attributed to accelerated expansion. Within a relaxation framework, this dimming arises from the cumulative redistribution of radiative support across void-dominated regions. Importantly, this mechanism does not suffer from the pathologies of classical tired-light models. There is no stochastic scattering, no frequency-dependent blurring, and no violation of surface brightness relations. The redshift emerges from coherent field evolution rather than energy loss.

Taken together, these effects suggest that the observational pillars of expansion cosmology can be preserved while reinterpreting their underlying cause. What changes is not the data, but the ontology used to explain it.

\subsection*{Formal Consideration: Radiative Propagation in a Relaxing Field}

Let $\mathcal{E}(\mathbf{x},t)$ denote the electromagnetic field amplitude satisfying
\[
\Box \mathcal{E} + V_{\text{eff}}(\mathbf{x},t)\mathcal{E} = 0,
\]
where $V_{\text{eff}}$ encodes weak, large-scale coupling to the gravitational and entropic structure of spacetime.

Assume $V_{\text{eff}}$ varies slowly and is dominated by void-scale contributions such that
\[
V_{\text{eff}} \sim \nabla^2 S(\mathbf{x},t),
\]
with $S$ the coarse-grained entropy field.

A WKB approximation yields a phase shift
\[
\Delta \phi \sim \int_\gamma V_{\text{eff}}\, d\ell,
\]
accumulated along the light path $\gamma$. This phase shift modifies the effective frequency observed at the endpoint without requiring local energy dissipation.

The observed redshift is then
\[
1+z \approx \exp\left(\int_\gamma \alpha(\mathbf{x})\, d\ell\right),
\]
where $\alpha$ encodes the relaxation-induced coupling strength. For sufficiently smooth, large-scale fields, this relation is indistinguishable from the redshift–distance relation derived from an expanding metric.

Thus, redshift, lensing, and distance measures can be recovered as emergent properties of radiative propagation in a relaxing universe, without invoking global metric expansion.

\section{ΛCDM, Entropic Gravity, and Relaxation Cosmology: A Comparative Framework}

The standard cosmological model, ΛCDM, explains large-scale observations through a combination of general relativity, cold dark matter, and a cosmological constant associated with dark energy. Within this framework, cosmic acceleration is treated as a fundamental dynamical property of spacetime itself, encoded in the Friedmann equations through a positive vacuum energy density. While phenomenologically successful, this construction introduces profound theoretical tensions, most notably the cosmological constant problem and the absence of direct empirical access to the dominant components of the model.

Entropic gravity offers a partial departure from this picture by reframing gravity not as a fundamental interaction but as an emergent, entropic force arising from the statistical behavior of microscopic degrees of freedom associated with spacetime information (Verlinde 2011). In this approach, deviations from Newtonian gravity at galactic and intergalactic scales emerge naturally from entropy gradients, providing an alternative explanation for phenomena traditionally attributed to dark matter. Importantly, entropic gravity preserves local agreement with general relativity while modifying large-scale behavior.

Relaxation cosmology extends this logic further by addressing not only gravitational anomalies but also the interpretation of cosmological redshift and acceleration. Where ΛCDM introduces dark energy as a new fundamental component, and entropic gravity treats dark matter as emergent, relaxation cosmology interprets dark energy phenomenology as the large-scale cumulative effect of entropic smoothing across cosmic voids. The universe does not expand in the conventional sense; instead, it asymptotically relaxes toward a smoother entropy configuration.

In ΛCDM, voids are passive byproducts of expansion and structure formation. In a relaxation framework, voids play an active dynamical role. They are regions of maximal entropy gradient, where matter density is low and relaxation rates are highest. The apparent acceleration inferred from supernova observations arises because light traversing increasingly void-dominated paths accumulates relaxation-induced phase shifts. This effect mimics acceleration when interpreted through an expansion-based metric, but does not require spacetime itself to stretch.

Entropic gravity provides a crucial conceptual bridge. Verlinde’s framework predicts an additional gravitational component associated with entropy that becomes significant precisely at scales where cosmic acceleration is inferred. Rather than interpreting this contribution as a repulsive force driving expansion, relaxation cosmology interprets it as a smoothing pressure that redistributes energetic and informational structure without increasing spatial volume. Dark energy, in this view, is not a substance nor a fundamental constant, but the macroscopic signature of entropy doing its work across the largest scales.

This reinterpretation resolves several conceptual difficulties simultaneously. The cosmological constant problem is avoided because no vacuum energy density need be fine-tuned to an absurdly small value. The coincidence problem—why dark energy becomes dominant precisely in the current epoch—reduces to a statement about the maturity of large-scale structure and void dominance. As matter aggregates into bound structures, the remaining volume increasingly consists of low-density regions where relaxation effects dominate observationally.

\subsection*{Formal Comparison: Expansion Versus Relaxation}

In ΛCDM, the scale factor $a(t)$ obeys
\[
\left(\frac{\dot a}{a}\right)^2 = \frac{8\pi G}{3}\rho + \frac{\Lambda}{3},
\]
with acceleration given by
\[
\frac{\ddot a}{a} = -\frac{4\pi G}{3}(\rho + 3p) + \frac{\Lambda}{3}.
\]

Relaxation cosmology replaces $a(t)$ with a fixed large-scale geometry and introduces a relaxation functional $\mathcal{R}(t)$ describing entropy redistribution:
\[
\mathcal{R}(t) = \int_{\Sigma} \nabla S \cdot \mathbf{n}\, dA,
\]
where $\Sigma$ denotes a coarse-grained boundary enclosing void-dominated regions.

Observable acceleration arises when distance measures are inferred assuming expanding coordinates:
\[
\frac{d_L(z)}{(1+z)} \sim \int \exp\left(\int \alpha(\mathbf{x})\, d\ell\right)\, d\ell,
\]
which reproduces the same luminosity-distance relation attributed to $\Lambda$ when $\alpha$ varies slowly and monotonically with cosmic time.

Entropic gravity enters through the identification
\[
\alpha(\mathbf{x}) \propto \frac{\partial S}{\partial r},
\]
linking redshift accumulation directly to entropy gradients rather than to metric expansion.

Thus, ΛCDM, entropic gravity, and relaxation cosmology can be viewed not as mutually exclusive descriptions, but as successive layers of abstraction. ΛCDM parameterizes observed regularities, entropic gravity explains deviations from Newtonian expectations, and relaxation cosmology provides an ontological account in which acceleration is reinterpreted as large-scale smoothing rather than expansion.

In this sense, dark energy is not a driver of cosmic growth, but the residue of entropy completing its work across an increasingly void-dominated universe.

\section{Conclusion: Everlasting Redshift and the Relaxing Universe}

The analysis developed in this essay has argued that the dominant interpretation of cosmological redshift as direct evidence of metric expansion is neither logically mandatory nor empirically unique. While the $\Lambda$CDM framework successfully parameterizes a wide range of observations, its explanatory structure rests on assumptions—global isotropy, fundamental dark energy, and persistent geodesic expansion—that are increasingly strained by both theoretical inconsistencies and observational tensions. 

The concept of \emph{everlasting redshift} reframes the problem at its root. Redshift is treated not as the kinematic residue of galaxies receding through expanding space, nor as a dissipative loss of photon energy, but as an accumulated interaction between propagating radiation and a structured, entropy-driven cosmic medium. Light is neither “tired” nor deflected arbitrarily; it is progressively phase-shifted by the statistical geometry of matter, fields, and information through which it propagates. In this view, redshift is the observational trace of relaxation rather than expansion.

By emphasizing light as a radiating field with Gaussian falloff, rather than an idealized pointlike projectile traversing perfectly straight null geodesics, the relaxation framework restores physical realism to cosmological optics. Over cosmological distances, perfect straight-line propagation is not generically expected in any medium with nonzero structure. The universe need not be strongly curved, nor photons strongly scattered, for cumulative orthonormal deviations and reabsorption tendencies to produce measurable redshift. This perspective aligns naturally with relativistic insights that any mass-energy distribution can be assigned an effective horizon scale, blurring the distinction between localized objects and extended gravitational absorbers.

Crucially, this framework remains distinct from historical tired-light proposals. Energy conservation is preserved locally, spectral coherence is maintained, and time-dilation observations are recovered through relaxation-weighted path integrals rather than ad hoc attenuation mechanisms. The failure of tired-light models stemmed not from questioning expansion, but from the absence of a physically grounded replacement. Everlasting redshift supplies such a grounding by embedding photon propagation within a thermodynamic, entropic, and geometrically constrained substrate.

The comparison with entropic gravity clarifies the broader significance of this shift. Where entropic gravity reinterprets gravitational attraction as an emergent consequence of entropy gradients, relaxation cosmology reinterprets dark energy as the large-scale integral of those same gradients operating across void-dominated regions. The universe does not expand to accommodate entropy; it smooths. Apparent acceleration arises because the remaining observational volume becomes increasingly dominated by low-density regions in which relaxation effects are most pronounced. Dark energy, in this sense, is not a new force but the visible residue of entropy completing its work.

This reframing resolves multiple long-standing paradoxes. The cosmological constant problem dissolves because no finely tuned vacuum energy is required. The coincidence problem becomes a statement about structural maturity rather than cosmic timing. Anisotropies and bulk flows cease to be anomalies and instead become diagnostic tools for mapping relaxation dynamics. Most importantly, the interpretation restores continuity between local physics and cosmological inference, avoiding the need for fundamentally different rules at different scales.

The broader implication is methodological. Cosmology has inherited a habit of interpreting redshift geometrically first and physically second. Everlasting redshift reverses that order. Geometry becomes a derived description of long-term statistical behavior, not an ontological primitive. The universe is not a balloon inflating into nothingness, but a structured, dissipative, information-bearing system approaching thermodynamic equilibrium.

In this light, expansion is not denied but reclassified as an effective description arising from long-horizon averaging procedures, rather than as a primitive property of spacetime. The empirical phenomenon that persists is a cumulative and enduring redshift, which tracks the entropic smoothing and relaxation of the universe as a physical system, rather than its literal enlargement.

\newpage
\appendix
\section{Appendix: Technical Foundations of Everlasting Redshift}

\subsection{Photon Propagation in a Structured, Relaxing Medium}

In standard cosmology, photon propagation is modeled as motion along null geodesics of an expanding Friedmann–Lemaître–Robertson–Walker (FLRW) spacetime. Redshift is attributed to the stretching of spacetime itself, with wavelength scaling proportionally to the cosmic scale factor. This treatment implicitly assumes idealized straight-line propagation through a homogeneous background, with curvature effects encoded globally rather than accumulated locally.

In a relaxation framework, photon propagation is instead treated as the evolution of a radiating field traversing a statistically structured medium. Even in the absence of strong scattering or absorption, cumulative interactions with matter distributions, gravitational potentials, and entropy gradients introduce small but coherent phase shifts. These shifts need not deflect photons appreciably nor randomize their direction; rather, they bias propagation orthonormally toward regions of higher absorption probability, consistent with thermodynamic expectations.

Let the electromagnetic field amplitude $A(\mathbf{x})$ obey a Gaussian falloff in free propagation,
\[
A(r) \propto \exp\!\left(-\frac{r^2}{2\sigma^2}\right),
\]
where $\sigma$ characterizes the coherence scale of the radiative field. In a perfectly empty universe, $\sigma \to \infty$ recovers idealized plane-wave propagation. In a realistic universe, $\sigma$ is finite due to ubiquitous matter, curvature, and quantum structure. Over cosmological distances, this finiteness ensures that perfectly straight, indefinitely persistent null trajectories are measure-zero idealizations.

Redshift arises when the phase evolution of the field accumulates relaxation-induced delays,
\[
\phi(r) = \int_0^r k(\ell)\, d\ell,
\]
where the effective wavevector $k(\ell)$ depends weakly on local entropy gradients. The observed wavelength shift satisfies
\[
1 + z = \exp\!\left(\int_0^{\ell_{\text{obs}}} \alpha(\mathbf{x})\, d\ell \right),
\]
with $\alpha(\mathbf{x})$ encoding the local relaxation rate. This preserves spectral coherence and linear dispersion relations, distinguishing the mechanism sharply from classical tired-light attenuation.

\subsection{Effective Horizons and the Black-Hole Analogy}

General relativity permits any mass-energy distribution to be associated with an effective Schwarzschild radius,
\[
r_s = \frac{2GM}{c^2},
\]
regardless of whether a classical event horizon forms. In this sense, even elementary particles such as protons possess an effective horizon scale, albeit vastly larger than their physical size when treated relativistically. While such horizons are not operationally black holes, they establish a conceptual continuity between localized absorption, gravitational time dilation, and large-scale redshift accumulation.

In a relaxation cosmology, the universe is permeated by overlapping, scale-dependent effective horizons. Photon propagation is therefore never strictly horizon-free. Redshift reflects the integrated influence of these horizons on phase evolution, not discrete capture events. This interpretation avoids both photon fatigue and ad hoc scattering while remaining consistent with relativistic causality.

\subsection{Luminosity Distance Without Metric Expansion}

Observed supernova luminosity distances are commonly expressed as
\[
d_L(z) = (1+z) r(z),
\]
with $r(z)$ computed under an expanding metric. In the relaxation framework, the same functional form emerges, but the redshift factor arises from accumulated relaxation rather than expansion. The inferred distance satisfies
\[
d_L(z) \sim \int_0^{z} \exp\!\left(\int_0^{\ell(z')} \alpha(\mathbf{x})\, d\ell \right) d\ell,
\]
which reproduces the $\Lambda$CDM luminosity–distance relation when $\alpha$ varies slowly with cosmic time and correlates with void dominance.

This degeneracy explains why standard candle observations alone cannot distinguish expansion from relaxation. The difference lies not in the fit quality, but in the ontological interpretation.

\subsection{Distinction from Classical Tired-Light Models}

Classical tired-light hypotheses posited ad hoc energy loss mechanisms, often predicting spectral blurring, frequency-dependent dispersion, or the absence of time dilation in high-redshift sources. All such predictions conflict with observation.

Everlasting redshift avoids these failures by preserving local energy conservation and coherence. Energy is not dissipated; it is redistributed within the field–medium system. Time dilation arises naturally because relaxation affects phase evolution uniformly across the wave packet, maintaining consistency with observed supernova light curves.

Thus, everlasting redshift is not a rehabilitated tired-light model, but a fundamentally different physical mechanism grounded in relativistic field behavior and entropy dynamics.

\subsection{Dark Energy as Integrated Entropic Relaxation}

Within entropic gravity, deviations from Newtonian dynamics arise when entropy gradients dominate over local mass distributions (Verlinde 2011). Relaxation cosmology extends this principle to cosmological scales. As structure formation proceeds, matter aggregates into bound systems while voids grow in volume fraction. These voids become the primary contributors to the integrated relaxation experienced by propagating radiation.

The effective acceleration attributed to dark energy corresponds to
\[
\ddot{\ell}_{\text{eff}} \propto \int_{\text{voids}} \nabla S \cdot d\mathbf{A},
\]
not to a repulsive force or vacuum pressure. The universe appears to accelerate because observers interpret relaxation-induced redshift through an expansion-based metric.

\subsection{Summary}

The appendix formalizes the central claim of this essay: cosmological redshift can be consistently reinterpreted as a cumulative, entropy-driven relaxation effect without invoking metric expansion or exotic energy components. This framework preserves observational successes while resolving foundational tensions, offering a physically continuous alternative to the standard cosmological narrative.

\newpage
\begin{thebibliography}{99}

\bibitem{Riess1998}
Riess, A. G., Filippenko, A. V., Challis, P., et al. (1998).
\newblock Observational Evidence from Supernovae for an Accelerating Universe and a Cosmological Constant.
\newblock \emph{The Astronomical Journal}, 116(3), 1009--1038.

\bibitem{Perlmutter1999}
Perlmutter, S., Aldering, G., Goldhaber, G., et al. (1999).
\newblock Measurements of $\Omega$ and $\Lambda$ from 42 High-Redshift Supernovae.
\newblock \emph{The Astrophysical Journal}, 517(2), 565--586.

\bibitem{Weinberg1989}
Weinberg, S. (1989).
\newblock The Cosmological Constant Problem.
\newblock \emph{Reviews of Modern Physics}, 61(1), 1--23.

\bibitem{PeeblesRatra2003}
Peebles, P. J. E., \& Ratra, B. (2003).
\newblock The Cosmological Constant and Dark Energy.
\newblock \emph{Reviews of Modern Physics}, 75(2), 559--606.

\bibitem{Planck2018}
Planck Collaboration (2018).
\newblock Planck 2018 Results. VI. Cosmological Parameters.
\newblock \emph{Astronomy \& Astrophysics}, 641, A6.

\bibitem{Ellis2014}
Ellis, G. F. R. (2014).
\newblock Inhomogeneity Effects in Cosmology.
\newblock \emph{Classical and Quantum Gravity}, 28(16), 164001.

\bibitem{Wiltshire2007}
Wiltshire, D. L. (2007).
\newblock Cosmic Clocks, Cosmic Variance and Cosmic Averages.
\newblock \emph{New Journal of Physics}, 9, 377.

\bibitem{Buchert2000}
Buchert, T. (2000).
\newblock On Average Properties of Inhomogeneous Fluids in General Relativity.
\newblock \emph{General Relativity and Gravitation}, 32(1), 105--125.

\bibitem{Buchert2008}
Buchert, T. (2008).
\newblock Dark Energy from Structure: A Status Report.
\newblock \emph{General Relativity and Gravitation}, 40(2--3), 467--527.

\bibitem{Sarkar2019}
Sarkar, S., Pandey, B., Khatri, R., \& Tiwari, S. K. (2019).
\newblock Is There Evidence for Cosmic Acceleration from Supernovae?
\newblock \emph{Scientific Reports}, 9, 12568.

\bibitem{Sarkar2021}
Sarkar, S. (2021).
\newblock The Case Against Dark Energy.
\newblock \emph{European Physical Journal Special Topics}, 230, 2979--2992.

\bibitem{Kycia2021}
Kycia, R. (2021).
\newblock Entropy in Thermodynamics: From Foliation to Categorization.
\newblock \emph{Communications in Mathematics}, 29, 49--66.
\newblock \doi{10.2478/cm-2021-0002}

\bibitem{Nielsen2016}
Nielsen, J. T., Guffanti, A., \& Sarkar, S. (2016).
\newblock Marginal Evidence for Cosmic Acceleration from Type Ia Supernovae.
\newblock \emph{Scientific Reports}, 6, 35596.

\bibitem{Verlinde2011}
Verlinde, E. (2011).
\newblock On the Origin of Gravity and the Laws of Newton.
\newblock \emph{Journal of High Energy Physics}, 2011(4), 29.

\bibitem{Verlinde2017}
Verlinde, E. (2017).
\newblock Emergent Gravity and the Dark Universe.
\newblock \emph{SciPost Physics}, 2(3), 016.

\bibitem{Jacobson1995}
Jacobson, T. (1995).
\newblock Thermodynamics of Spacetime: The Einstein Equation of State.
\newblock \emph{Physical Review Letters}, 75(7), 1260--1263.

\bibitem{Padmanabhan2010}
Padmanabhan, T. (2010).
\newblock Thermodynamical Aspects of Gravity: New Insights.
\newblock \emph{Reports on Progress in Physics}, 73(4), 046901.

\bibitem{Bekenstein1973}
Bekenstein, J. D. (1973).
\newblock Black Holes and Entropy.
\newblock \emph{Physical Review D}, 7(8), 2333--2346.

\bibitem{Hawking1975}
Hawking, S. W. (1975).
\newblock Particle Creation by Black Holes.
\newblock \emph{Communications in Mathematical Physics}, 43(3), 199--220.

\bibitem{tHooft1993}
't Hooft, G. (1993).
\newblock Dimensional Reduction in Quantum Gravity.
\newblock \emph{arXiv:gr-qc/9310026}.

\bibitem{Susskind1995}
Susskind, L. (1995).
\newblock The World as a Hologram.
\newblock \emph{Journal of Mathematical Physics}, 36(11), 6377--6396.

\bibitem{Tolman1930}
Tolman, R. C. (1930).
\newblock On the Estimation of Distances in a Curved Universe.
\newblock \emph{Proceedings of the National Academy of Sciences}, 16(7), 511--520.

\bibitem{Zwicky1929}
Zwicky, F. (1929).
\newblock On the Red Shift of Spectral Lines through Interstellar Space.
\newblock \emph{Proceedings of the National Academy of Sciences}, 15(10), 773--779.

\bibitem{LubinSandage2001}
Lubin, L. M., \& Sandage, A. (2001).
\newblock The Tolman Surface Brightness Test for the Reality of the Expansion.
\newblock \emph{The Astronomical Journal}, 122(2), 1084--1103.

\bibitem{EllisStoeger1987}
Ellis, G. F. R., \& Stoeger, W. (1987).
\newblock The ‘Fitting Problem’ in Cosmology.
\newblock \emph{Classical and Quantum Gravity}, 4(6), 1697--1729.

\bibitem{Penrose2004}
Penrose, R. (2004).
\newblock \emph{The Road to Reality}.
\newblock Jonathan Cape, London.

\bibitem{Carroll2001}
Carroll, S. M. (2001).
\newblock The Cosmological Constant.
\newblock \emph{Living Reviews in Relativity}, 4(1), 1.

\bibitem{Smolin2006}
Smolin, L. (2006).
\newblock The Case for Background Independence.
\newblock \emph{arXiv:hep-th/0507235}.

\bibitem{Barbour2012}
Barbour, J. (2012).
\newblock Shape Dynamics: An Introduction.
\newblock \emph{Quantum Field Theory and Gravity}, Springer.

\end{thebibliography}

\end{document}
