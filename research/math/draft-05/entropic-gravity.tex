\documentclass[11pt]{article}
\usepackage[utf8]{inputenc}
\usepackage[T1]{fontenc}
\usepackage[margin=1in]{geometry}
\usepackage{amsmath,amssymb,amsthm,physics,bm,hyperref,microtype}
\hypersetup{colorlinks=true,linkcolor=blue,citecolor=blue,urlcolor=blue}
\usepackage{enumitem}
\usepackage{booktabs}
\usepackage{natbib}
\usepackage{noto}
\theoremstyle{plain}
\newtheorem{theorem}{Theorem}
\newtheorem{lemma}{Lemma}
\theoremstyle{definition}
\newtheorem{definition}{Definition}
\newtheorem{assumption}{Assumption}
\newtheorem{proposition}{Proposition}
\newtheorem{corollary}{Corollary}
\title{The Fall of Space: Entropic Relaxation and Structure Without Expansion in a Scalar-Vector Plenum}
\author{Flyxion}
\date{\today}
\begin{document}
\maketitle
\begin{abstract}
The Relativistic Scalar-Vector Plenum (RSVP) model proposes a cosmological framework where redshift, cosmic structure, and gravitational effects emerge from interactions of a scalar density field $\Phi$, a vector flow field $\bm{v}$, and an entropy field $S$, without requiring metric expansion. This approach revisits historical debates on static versus expanding universes, from Einstein's static model to the Big Bang paradigm, highlighting why a return to a static frame with dynamic plenum reorganization addresses persistent anomalies in the standard $\Lambda$CDM model. Gravitational collapse (lamphron process) releases binding energy that enhances a vacuum-capacity field $\Phi$ (lamphrodyne process), generating outward pressure mimicking inflation and dark energy. Lamphron refers to the inward collapse releasing energy, while lamphrodyne denotes the outward expansion of vacuum capacity. Implemented on a 3D lattice, RSVP reproduces the cosmic web, explains redshift as an entropic gradient ($z \propto \Delta S$), and resolves $\Lambda$CDM anomalies like the Hubble tension and CMB cold spot. We present field equations, a minimal simulation algorithm, and testable predictions for void lensing, high-$z$ BAO, and CMB anisotropies, incorporating Cartan torsion for plenomic vorticity. These predictions can be tested with missions such as Euclid for void lensing, Square Kilometre Array (SKA) for large-scale structure, and James Webb Space Telescope (JWST) for high-redshift observations. The Trajectory-Aware Recursive Tiling with Annotated Noise (TARTAN) framework enhances simulations, enabling unistochastic quantum-like behavior to emerge from recursive field dynamics, aligning with Barandes's reformulation of quantum theory. Falsifiability criteria and comparisons with $\Lambda$CDM strengthen the model's credibility.
\end{abstract}
\section{Introduction}
The $\Lambda$CDM model, the standard cosmological framework, posits an expanding universe driven by dark energy and cold dark matter. Its successes include precise predictions of cosmic microwave background (CMB) anisotropies \citep{Planck2018}, big bang nucleosynthesis (BBN) abundances, and large-scale structure formation. However, persistent anomalies challenge its completeness: the Hubble tension, a 5--10\% discrepancy between local measurements ($H_0 \approx 73$ km/s/Mpc \citep{Riess2022}) and CMB-inferred values ($H_0 \approx 67$ km/s/Mpc, 5$\sigma$ significance); CMB irregularities, including hemispherical asymmetry, the cold spot's 3.7-degree scale, and unexpected integrated Sachs-Wolfe (ISW) effects; and the missing satellites problem, where observed dwarf galaxies are fewer than predicted.
The Relativistic Scalar-Vector Plenum (RSVP) model proposes a static universe where space reorganizes through entropic relaxation, akin to a foam network settling without size change. Redshift emerges from entropy gradients ($z \propto \Delta S$), structure forms via scalar-vector coupling, and CMB uniformity arises from plenum thermalization, eliminating the need for inflation or dark matter. Unlike Einstein's static model, abandoned due to instability and redshift evidence, RSVP is a non-metric, thermodynamic framework inspired by Jacobson's thermodynamic gravity \citep{Jacobson1995}, Verlinde's emergent gravity \citep{Verlinde2011}, and Padmanabhan's entropic cosmology \citep{Padmanabhan2015}. It aligns with modern nonequilibrium thermodynamics and non-Riemannian geometry \citep{Shao2023}, extending these by modeling gravity as an entropic process in a dynamic plenum.
Recent entropic gravity theories, such as Vopson's mass-energy-information equivalence \citep{Vopson2022}, provide a foundation for RSVP's approach. Vopson derives Newtonian gravity from information entropy minimization on a Planck-scale lattice, a concept RSVP extends through recursive causality and dynamic field interactions. Similarly, Barandes's unistochastic quantum theory \citep{Barandes2024} reformulates quantum mechanics using directed conditional probabilities, which RSVP's Trajectory-Aware Recursive Tiling with Annotated Noise (TARTAN) framework can generate as emergent phenomena from field dynamics.
Table \ref{tab:comparison} compares $\Lambda$CDM and RSVP predictions, emphasizing RSVP's parameter economy and unique signatures.
\begin{table}[ht]
\centering
\caption{Comparison of $\Lambda$CDM and RSVP Predictions}
\label{tab:comparison}
\begin{tabular}{lcc}
\toprule
Phenomenon & $\Lambda$CDM & RSVP \\
\midrule
Redshift & Metric expansion (Doppler-like) & Entropic gradient ($z \propto \Delta S$) \\
Structure Formation & Gravitational instability + dark matter & $\Phi$-$\bm{v}$-$S$ coupling + lamphron condensation \\
CMB Uniformity & Inflationary stretching & Plenum thermalization via entropic relaxation \\
BAO & Acoustic oscillations in expanding fluid & Entropy-driven oscillations in static plenum \\
Hubble Tension & Possible systematics or new physics & Anisotropic entropy gradients along lines of sight \\
\bottomrule
\end{tabular}
\end{table}
The historical context of cosmology reveals a progression from static models to expanding ones, driven by Hubble's law and CMB discovery. However, anomalies suggest revisiting static paradigms with thermodynamic twists. RSVP integrates insights from Bianconi's entropy-derived gravity \citep{Bianconi2025}, Carroll and Remmen's distinction between holographic and thermodynamic gravity \citep{CarrollRemmen2016}, and Darmos's space particle dualism \citep{Darmos2021}, where gravity arises from baryon counts rather than mass-energy.
\subsection{Contributions}
\begin{enumerate}
    \item A field-theoretic model with $\Phi$-$\bm{v}$-$S$ coupling, replacing metric expansion.
    \item Lattice simulations demonstrating cosmic web emergence and entropic redshift.
    \item Predictions for void lensing, BAO deviations, and CMB anomalies.
    \item A minimal simulation algorithm for TARTAN-style tessellations.
\end{enumerate}
To address criticisms of entropic gravity (e.g., Visser's multi-temperature baths \citep{Visser2011}, Feynman's dissipation concerns \citep{Feynman1964}), RSVP's recursive causality ensures consistent entropy production without unphysical assumptions. By incorporating directed flows and feedback loops, the model avoids issues related to equilibrium assumptions and maintains thermodynamic consistency across scales.
\section{Field Definitions and Dynamics}
The RSVP plenum is defined by three fields, each with a clear physical interpretation:
\begin{itemize}
    \item \textbf{Scalar field} $\Phi: \mathbb{R}^{1,3} \to \mathbb{R}$, representing vacuum capacity or plenum density, analogous to tension in a stretched membrane that stores and releases energy during collapse and expansion processes.
    \item \textbf{Vector field} $\bm{v}: \mathbb{R}^{1,3} \to T\mathbb{R}^3$, encoding negentropic flow (``falling space''), similar to a reversed heat flow where active pumping drives motion from low to high entropy regions.
    \item \textbf{Entropy field} $S: \mathbb{R}^{1,3} \to \mathbb{R}$, driving redshift and constraint relaxation, acting as a gradient-driven clock that quantifies the ``age'' or relaxation state of local plenum regions.
\end{itemize}
The Lagrangian density motivates the dynamics:
\begin{equation}
\mathcal{L} = \frac{1}{2} \partial_\mu \Phi \partial^\mu \Phi - U(\Phi) + \frac{\rho_m}{2} |\bm{v}|^2 - \rho_m \varphi + \lambda \Phi \sigma_g(\rho_m) - \Gamma \dot{\Phi}^2,
\label{eq:L}
\end{equation}
where $\frac{1}{2} \partial_\mu \Phi \partial^\mu \Phi$ is the kinetic term for $\Phi$, representing free propagation of vacuum capacity fluctuations; $-U(\Phi)$ is the potential energy, potentially mimicking a cosmological constant at low $\Phi$; $\frac{\rho_m}{2} |\bm{v}|^2$ is the kinetic energy of matter advected by the flow $\bm{v}$; $-\rho_m \varphi$ couples matter to the Newtonian potential, with $\nabla^2 \varphi = 4\pi G \rho_m$; $\lambda \Phi \sigma_g$ transduces gravitational strain ($\sigma_g = |\nabla \bm{g}|$, $\bm{g} = -\nabla \varphi$) into $\Phi$-field energy, linking collapse to vacuum pumping; $-\Gamma \dot{\Phi}^2$ provides damping, ensuring energy dissipation.
This form links to entropic gravity (entropy gradients drive forces), fluid dynamics (plenum as a viscous medium), and non-Riemannian cosmology (torsion from $\bm{v}$ shear). The Lagrangian encapsulates the interplay between these fields, allowing emergent gravitational effects to arise from thermodynamic principles rather than geometric curvature.
\subsection{Coupling Constants}
The constants $\lambda, \alpha, \beta, \Gamma, \kappa, \eta, \zeta$ arise from dimensional analysis or fundamental scales. For instance, $\kappa$ (matter-vacuum interchange) has units of energy density inverse, potentially set by Planck-scale physics, such as $\kappa \approx \hbar c / l_P^4$, where $l_P$ is the Planck length, ensuring consistency with quantum gravity scales. Constraints come from observations: $\alpha$ (strain-$\Phi$ coupling) tuned to match void growth rates observed in galaxy surveys; $\beta$ (entropy-$\Phi$ exchange) to fit redshift-distance relations from Type Ia supernovae. To reduce free parameters, $\lambda$ and $\alpha$ can be related via thermodynamic consistency, for example, $\lambda = \alpha / T$, where $T$ is a characteristic temperature, potentially the CMB temperature. This relation derives from equating gravitational work to thermal energy transfer. Overall, these constraints ensure the model aligns with empirical data while maintaining fewer adjustable parameters than the six in $\Lambda$CDM, enhancing its parsimony and testability.
\subsection{Role of Entropy}
$S$ drives redshift via 
\begin{equation}
z \approx \exp\left[\frac{\chi}{2} \int \partial_s \ln(1 + \chi S) \, ds\right],
\label{eq:redshift}
\end{equation}
where gradients accumulate photon energy loss. Example: In a void, high $S$ gradients yield larger $z$ for distant sources, mimicking acceleration without expansion. The parameter $\chi$ represents the coupling strength between entropy and photon frequency, dimensionally $[length]^{-1}$, and is calibrated to match observed redshift curves. This mechanism interprets redshift as a cumulative effect along the photon's path, where each infinitesimal increase in entropy contributes to energy degradation, analogous to tired light but grounded in thermodynamic principles. In dense regions like galaxy clusters, lower $S$ gradients result in smaller redshifts, providing a natural explanation for local variations in expansion rates.
\section{Physical Foundation of Entropic Redshift}
To ground entropic redshift, derive it from photon geodesics in a non-Riemannian manifold with entropy-dependent connection $\Gamma^\lambda_{\mu\nu} = \tilde{\Gamma}^\lambda_{\mu\nu} + f(S) T^\lambda_{\mu\nu}$, where $T$ is torsion and $f(S)$ modulates via entropy. The function $f(S)$ could be linear, $f(S) = \xi S$, with $\xi$ a constant determined by microscopic interactions.
The frequency shift follows from the null geodesic equation $k^\mu \nabla_\mu k^\nu = 0$, yielding 
\begin{equation}
\frac{d\nu}{ds} = -\nu \partial_s S / 2
\label{eq:freq_shift}
\end{equation}
in the eikonal approximation, analogous to refractive index variation in media. This equation indicates that the photon's frequency decreases proportionally to the entropy gradient along its path, providing a geometric interpretation of energy loss.
Microphysically, this arises as a statistical-mechanical effect on the electromagnetic field phase, where photons interact with plenum fluctuations like diffusion in plasma. The plenum acts as a medium with varying ``optical depth'' due to entropy, scattering photons subtly without violating conservation laws.
Analogues include Tolman temperature gradients (redshift in thermal equilibria), gravitational time dilation (clocks in potentials), and photon scattering in inhomogeneous media. For instance, in the Tolman effect, temperature varies with gravitational potential, leading to redshift; here, entropy plays a similar role, unifying thermal and gravitational effects.
\section{Field Equations}
The action $S = \int \mathcal{L} \sqrt{-g} d^4x$, varying with respect to fields. For $\Phi$:
\begin{equation}
\square \Phi + U'(\Phi) - \lambda \sigma_g + 2\Gamma \ddot{\Phi} = 0,
\label{eq:phi-eq}
\end{equation}
leading to a diffusion-like equation with source terms from gravitational strain and damping. This equation describes how $\Phi$ evolves, balancing kinetic propagation, potential, pumping from collapse, and dissipation.
Similar variations for $\bm{v}$ and $S$ yield coupled equations. Dimensional analysis: $[\Phi] = M^{1/2} L^{-1/2} T^{-1}$, consistent across terms.
Special cases: $\bm{v} = 0$, $\Phi$ constant reduces to Newtonian gravity via Poisson equation.
The entropy balance is:
\begin{equation}
\partial_t S + \nabla \cdot \bm{J}_S = \eta \sigma_g + \zeta (\nabla \Phi)^2,
\label{eq:entropy}
\end{equation}
where $\bm{J}_S$ is the entropy flux, typically $\bm{J}_S = S \bm{v}$, ensuring advection with the flow. This equation enforces the second law locally, with positive production terms from strain and $\Phi$ gradients.
\section{Cartan Torsion}
Cartan torsion encodes plenomic vorticity:
\begin{equation}
T^j_{ik} = \Gamma^j_{ik} - \Gamma^j_{ki},
\end{equation}
geometrically representing asymmetric connections from $\bm{v}$ shear, as illustrated in conceptual diagrams of twisted geodesics in a vortical plenum. Torsion introduces skew-symmetric components to the affine connection, allowing for non-closed parallelograms in parallel transport.
Compared to torsion-free GR, torsion alters structure formation by introducing chiral effects, potentially manifesting in galaxy spin alignments or anisotropic void dynamics, testable via surveys like SDSS. For example, torsion could lead to preferred handedness in spiral galaxies, observable as statistical asymmetries in rotation directions.
\section{Energetics and Outward Falling}
For a vacuum sphere:
\begin{equation}
U_G(R) = -\frac{4\pi G}{3} \rho_\Lambda m R^2, \quad \frac{dU_G}{dR} = -\frac{8\pi G}{3} \rho_\Lambda m R < 0,
\end{equation}
showing outward favorability. Applied to a cluster ($M \sim 10^{14} M_\odot$, $R \sim 1$ Mpc), $\Delta \Phi \sim 10^{-3} \rho_{\rm crit}$ for 10\% collapse efficiency. This calculation assumes a uniform density approximation, with energy release scaling quadratically with radius.
Analogous to buoyancy: vacuum ``rises'' in gravitational fields like less dense fluid. Numerical estimates: For void collapse, $\Delta E_{\rm bind} \sim 10^{60}$ erg pumps $\Delta \Phi$ sufficient for observed acceleration mimicry, matching the energy scale of dark energy over cosmic volumes.
\section{Lattice Implementation}
The plenum is discretized on an $N \times N \times N$ cubic lattice with fields $\rho_m, \Phi, S, \bm{v}$. Steps include:
\begin{enumerate}
    \item Solve Poisson equation $\nabla^2 \varphi = 4\pi G \rho_m$.
    \item Compute strain $\sigma_g = |\nabla \bm{g}|$, $\bm{g} = -\nabla \varphi$.
    \item Update $\Phi$ with diffusion, pumping, and damping terms.
    \item Adjust $\rho_m$ to enforce budget.
    \item Include $-\nabla p_\Phi$ in momentum equation.
\end{enumerate}
Pseudocode for the lattice integrator:
\begin{verbatim}
# Initialize grid and fields: rho_m, Phi, S, v (vector field), phi (potential)
# Assume N x N x N grid, dt timestep, parameters alpha, beta, gamma, D_Phi, kappa, G, c_Phi, etc.
for each timestep:
    # 1. Solve Poisson for gravitational potential phi
    g = -gradient(phi) # gravitational acceleration
    phi = poisson_solver(4 * pi * G * rho_m) # using FFT or iterative solver
   
    # 2. Compute gravitational strain sigma_g
    sigma_g = magnitude(gradient(g))
   
    # 3. Update Phi
    dot_Phi = (alpha * sigma_g + beta * time_derivative(S) - gamma * time_derivative(Phi) + D_Phi * laplacian(Phi))
    Phi_new = Phi + dt * dot_Phi
   
    # 4. Enforce local budget on rho_m
    rho_m = rho_m - kappa * (Phi_new - Phi)
    Phi = Phi_new
   
    # 5. Update momentum equation for v, including back-reaction from p_Phi = c_Phi**2 * Phi
    grad_p_Phi = c_Phi**2 * gradient(Phi)
    # Assume Navier-Stokes like solver for v:
    # rhs = -grad_p_m - rho_m * grad_phi - grad_p_Phi
    # v = update_velocity(v, rho_m, rhs, dt)
   
    # Optional: Update S with entropy balance
    # dot_S = eta * sigma_g + zeta * magnitude(gradient(Phi))**2 - divergence(J_S)
    # S = S + dt * dot_S
   
    # Advection for rho_m, Phi, S if needed
    # rho_m = advect(rho_m, v, dt)
    # etc.
   
    # Outputs: compute void statistics, lensing, RSVP redshift integrals along rays
\end{verbatim}
Parameter sensitivity: Increasing $\beta$ steepens redshift slope; varying $\alpha$ alters web topology from filament-dominated to void-heavy. For example, high $\alpha$ promotes rapid void evacuation, leading to sharper cosmic web features, while low $\alpha$ results in smoother distributions.
Visualization: Example outputs show early uniform $\Phi$ evolving to cosmic web by $10^3$ iterations, with filaments and voids emerging from initial perturbations amplified by entropic feedback.
\section{Engagement with Observational Evidence}
\subsection{Type Ia Supernovae}
The redshift $1+z \approx \exp(\chi \int \partial_s S ds / 2)$ fits luminosity distance $d_L = (1+z) \int dz / H(z)$ adapted to plenum, matching ZTF SN Ia DR2 (2025) with $\chi \sim 10^{-3}$ Mpc$^{-1}$, comparable to Pantheon+ fits but resolving tension via $S$ anisotropy. This adaptation replaces the expanding metric with an effective scale factor derived from entropy accumulation, providing equivalent distance measures but with anisotropic variations explaining Hubble tension.
\subsection{CMB Angular Power Spectrum}
RSVP reproduces acoustic peaks via entropy-driven oscillations in the plenum, with first peak at $\ell \sim 220$ from torsion-phase correlations, without expansion, analogous to plasma waves fixed in scale. The power spectrum arises from harmonic modes in $S$ fluctuations, with multipole moments determined by the characteristic length scales of entropy diffusion.
\subsection{BAO and Galaxy Surveys}
Comoving distance scales via $S$-integrated paths match DESI BAO at $z\sim0.11$, with potential 5\% shifts in alternatives testable against $\Lambda$CDM. These shifts arise from non-uniform entropy distributions, leading to direction-dependent BAO scales observable in wide-field surveys.
\section{Observational Consequences}
The model predicts distinct observables:
\begin{enumerate}
\item Void growth with sharp edges due to $\Phi$ peaks, differing from $\Lambda$CDM lensing profiles. Sharp edges result from rapid $\Phi$ transitions at void boundaries, leading to stronger lensing signals.
\item Hubble diagram fits with effective $w \to -1$, tracking structure formation. The effective equation of state parameter $w$ emerges from averaged entropy gradients, mimicking dark energy.
\item Enhanced ISW/Rees--Sciama effects correlated with super-voids. These effects arise from time-varying potentials due to evolving $S$, amplifying temperature anisotropies.
\item Modified cluster mass functions and halo profiles from central deepening and outward repulsion. Halos exhibit steeper central densities due to lamphron processes, testable with X-ray observations.
\item Potential MOND-like behavior in low-acceleration regions from $\Phi$ gradients. In weak fields, $\Phi$ modifications alter acceleration laws, providing a relativistic extension to MOND.
\end{enumerate}
\section{TARTAN Framework}
TARTAN stands for Trajectory-Aware Recursive Tiling with Annotated Noise. It is a conceptual and computational framework for enhancing simulation, visualization, and physical interpretability within RSVP:
\begin{itemize}
    \item \textbf{Trajectory-Aware}: Tracks history and intent of field evolution for causal memory. This component maintains a buffer of past field states and trajectories, allowing the simulation to incorporate long-term dependencies and prevent unphysical oscillations.
    \item \textbf{Recursive Tiling}: Subdivides domain into nested tiles for multiscale resolution. Starting from coarse grids, it refines regions of interest recursively, enabling efficient computation of hierarchical structures like the cosmic web.
    \item \textbf{Annotated Noise}: Semantic perturbations for structured randomness. Noise terms are tagged with physical meanings (e.g., quantum fluctuations or thermal noise), ensuring they preserve conservation laws and contribute to emergent behaviors.
\end{itemize}
Applications: Scene evolution, field interaction encoding, visualization of CMB anomalies. For instance, TARTAN can simulate CMB cold spots by annotating noise with torsion-induced vorticity, producing realistic anisotropy patterns.
\section{Recursive Causality and Dynamical Extension of Vopson's Entropic Gravity in RSVP}
While Vopson's model effectively captures gravity as an emergent entropic force from a discrete, static lattice of Planck-scale bits, the RSVP theory enriches this picture by incorporating \textbf{recursive causality} as a fundamental dynamical principle. Recursive causality in RSVP refers to the continuous, self-referential feedback loop whereby local changes in informational entropy density ($\Phi$) and directed negentropic vector fluxes ($\mathbf{v}$) not only respond to existing entropy gradients but also alter those gradients in a time-dependent manner, thus dynamically shaping the evolving scalar-vector-entropy field configuration.
This recursive interplay means that the scalar and vector fields mutually influence each other through iterative cycles of entropic relaxation and negentropic structuring. Instead of a static, one-shot minimization of information entropy as in Vopson's framework, RSVP describes a non-equilibrium, temporally recursive process in which localized fluctuations and constraints propagate and co-evolve. This yields emergent gravitational phenomena that evolve and self-organize at multiple scales, reflecting the universe's observed hierarchical structure without requiring spacetime expansion.
Mathematically, the recursive causality manifests in RSVP's coupled field equations, where time derivatives of $\Phi$ and $\mathbf{v}$ depend on spatial gradients of entropy and vector fluxes, generating feedback loops that stabilize or amplify structures depending on local entropic conditions. For example:
\begin{equation}
\partial_t \Phi = \alpha \sigma_g - \beta \partial_t S + D_\Phi \nabla^2 \Phi - \gamma \partial_t \Phi,
\label{eq:phi_recursive}
\end{equation}
\begin{equation}
\partial_t \bm{v} = -\frac{1}{\rho_m} \nabla p_\Phi + \bm{g} + \nu \nabla^2 \bm{v},
\label{eq:v_recursive}
\end{equation}
where terms like $\beta \partial_t S$ introduce feedback from entropy evolution. Hence, RSVP's recursive causality provides a crucial dynamical generalization of Vopson's entropic-information gravity, transforming a static entropic force into a continually evolving, self-consistent process. This approach better captures the irreversible, time-directed nature of entropy evolution and offers a robust theoretical framework for exploring gravitational phenomena as emergent from the deep informational structure of spacetime itself.
\section{Unistochastic Quantum Theory as an Emergent Description of the Relativistic Scalar Vector Plenum}
The unistochastic reformulation of quantum theory, as proposed by Barandes, replaces the wavefunction paradigm with directed conditional probabilities between configurations, enabling a causally local hidden-variables interpretation. We hypothesize that this unistochastic structure emerges from RSVP's recursive field dynamics, mediated by TARTAN coarse-graining.
Let the RSVP state on a lattice $L$ be fields $(\Phi,\bm{v},S)$ with evolution
\begin{align}
\partial_t \Phi &= \mathcal{F}_\Phi[\Phi,\bm{v},S] + \eta_\Phi, \label{eq:phi_evol}\\
\partial_t \bm{v} &= \mathcal{F}_v[\Phi,\bm{v},S] + \eta_v, \label{eq:v_evol}\\
\partial_t S &= \mathcal{F}_S[\Phi,\bm{v},S] + \eta_S, \label{eq:s_evol}
\end{align}
with local flux form $\bm{J} \equiv \Phi \bm{v}$ and an entropy-production constraint that enforces microreversibility up to controlled dissipation. Introduce a gauge 1-form $\mathcal{A}$ canonically conjugate (e.g., via a Clebsch or symplectic potential) so that circulation
\[
\theta(\gamma) = \oint_\gamma \mathcal{A} \cdot d\bm{x}
\]
assigns a phase to any coarse path $\gamma$. Entropy $S$ acts as a Lagrange multiplier that regulates allowed phase curvature $d\mathcal{A}$ (torsion/vorticity) to keep dynamics stable.
TARTAN provides recursive tiling $\{T_i^{(\ell)}\}$ at scales $\ell = 0,1,2,\dots$, trajectory buffers storing path counts and fluxes between tiles, Gaussian aura fields that smooth fields within tiles, and annotated noise $\eta$ with semantic tags that preserve conserved quantities and encode microreversibility.
From a short time step $\Delta t$, collect inter-tile fluxes at scale $\ell$:
\[
F_{i\to j}^{(\ell)} = \int_{\partial T_i^{(\ell)} \cap \partial T_j^{(\ell)}} \bm{J} \cdot d\bm{a} \quad (\text{with path tags}).
\]
Define the coarse transfer operator (Markov kernel) on tiles:
\[
P_{ij}^{(\ell)} = \frac{F_{i\to j}^{(\ell)}}{\sum_k F_{i\to k}^{(\ell)}}.
\]
So $P^{(\ell)}$ is row-stochastic and captures how RSVP mass/intent flows between tiles in one step.
Balance $P^{(\ell)}$ to a bistochastic $B^{(\ell)}$ using Sinkhorn-Knopp scaling with thermodynamic weights derived from $S$:
\[
B^{(\ell)} = D_L P^{(\ell)} D_R, \quad D_L, D_R > 0,
\]
with $D_L, D_R$ chosen to enforce $\sum_j B_{ij}^{(\ell)} = 1 = \sum_i B_{ij}^{(\ell)}$. This step selects the maximum-entropy, flux-preserving coarse dynamics compatible with observed inflow/outflow (Jaynesian inference with $S$ as a constraint).
A bistochastic matrix $B \in \mathbb{R}^{n \times n}$ is unistochastic if there exists a unitary $U \in U(n)$ with $|U_{ij}|^2 = B_{ij}$. Equivalently, seek phases $\Theta = \{\theta_{ij}\}$ such that
\[
U_{ij} = \sqrt{B_{ij}} e^{i \theta_{ij}}, \quad \sum_k U_{ik} \overline{U}_{jk} = \delta_{ij}.
\]
The orthogonality constraints are
\[
\forall i < j: \sum_k \sqrt{B_{ik} B_{jk}} e^{i (\theta_{ik} - \theta_{jk})} = 0. \tag{★}
\]
TARTAN’s trajectory buffers give coarse phase increments from the RSVP gauge field:
\[
\theta_{ik} = \frac{1}{\hbar_{\eff}} \oint_{\gamma(i \to k)} \mathcal{A} \cdot d\bm{x},
\]
where $\gamma(i \to k)$ is a representative coarse path cluster from tile $i$ to $k$, and $\hbar_{\eff}$ is an emergent action scale set by RSVP’s entropy–flux budget. The annotated noise enforces approximate microreversibility and phase additivity across path concatenations. Under the RSVP constraints (bounded vorticity/torsion, balanced entropy production), these phases can be tuned—by adjusting the local $S$-controlled gauge—so that the interference sums in (★) cancel. Intuitively:
- Magnitudes $\sqrt{B_{ik} B_{jk}}$ come from coarse flux counts.
- Relative phases $\theta_{ik} - \theta_{jk}$ are circulations of $\mathcal{A}$ around TARTAN’s tile-edge loops.
- Entropy $S$ regulates $d\mathcal{A}$ to make the polygonal phasor sums close to zero (orthogonality).
Thus, RSVP + TARTAN produce a feasible phase field that lifts $B$ to a unitary $U$.
Define a Hilbert space $\mathcal{H}^{(\ell)} \cong \mathbb{C}^{n_\ell}$ with basis $\{|i\rangle\}$ for tiles. Let coarse amplitudes be
\[
\psi_i = \sqrt{\mu_i} e^{i \varphi_i},
\]
where $\mu_i$ is the coarse RSVP mass/intent in tile $i$ and $\varphi_i$ the tile’s gauge phase (mean circulation). One TARTAN step acts by
\[
\psi' = U \psi, \quad \Pr(j) = |\psi'_j|^2 = \sum_i |\psi_i|^2 |U_{ji}|^2 + \text{interference}.
\]
Because $|U|^2 = B$, the marginals respect the bistochastic coarse flow, while interference terms arise from RSVP phase correlations. Measurements correspond to reading tile occupancy; the Born rule appears as the modulus-square map inherited from unistochastic structure.
2×2 case. Every bistochastic matrix is unistochastic:
\[
B = \begin{pmatrix} a & 1-a \\ 1-a & a \end{pmatrix}, \quad U = \begin{pmatrix} \sqrt{a} & \sqrt{1-a} \\ \sqrt{1-a} & -\sqrt{a} \end{pmatrix} e^{i\phi}
\]
(up to phases). RSVP/TARTAN phases determine the relative sign. This corresponds to a single gauge-controlled interference degree of freedom.
3×3 case. Not every bistochastic is unistochastic. Feasibility of phases solving (★) imposes geometric “triangle-closure” constraints on the phasors $\{\sqrt{B_{ik} B_{jk}}\}$. In RSVP terms: tile-loop circulations must permit closure—i.e., net coarse vorticity (integrated $d\mathcal{A}$) across certain TARTAN loops must sit in an allowed polytope. Tuning $S$ (entropy pressure) adjusts $\mathcal{A}$ to land $B$ inside the unistochastic subset.
Define a one-step transfer operator $\mathsf{T}$ on coarse densities via RSVP flux statistics. TARTAN’s entropy balancing yields a doubly stochastic $\mathsf{B}$ on tile space. By Stinespring dilation or polar decomposition on the $\ell^2$ space of tiles, and with RSVP-induced phases, you realize
\[
\mathsf{U}: \ell^2 \to \ell^2 \quad \text{unitary with} \quad |\mathsf{U}_{ij}|^2 = \mathsf{B}_{ij}.
\]
Thus the coarse dynamics admits a unitary dilation; the physical (observable) probabilities are the mod-squares—exactly the unistochastic recipe.
1. Compute SS from your RSVP step; solve the tile Poisson equation\
   ΔS(ℓ)=−(α/D(ℓ)) ρ(ℓ)\Delta S^{(\ell)} = -(\alpha/\mathcal{D}^{(\ell)})\,\rho^{(\ell)}.
2. Set φ=−(Θ/ν) S(ℓ)\varphi=-(\Theta/\nu)\,S^{(\ell)}; advance particles/tiles with v˙=−∇φ\dot{\mathbf v}=-\nabla\varphi.
3. For light/rays: refractive index n=exp⁡(γS(ℓ))n=\exp(\gamma S^{(\ell)}); integrate ray equations.
4. Optional GR-like update: evolve g(ℓ)g^{(\ell)} by (11) between re-tilings.
5. Toggle coherence: use your unistochastic UU to reduce D(ℓ)\mathcal{D}^{(\ell)} and watch the same mass produce deeper φ\varphi.
In RSVP + TARTAN, mass is an entropy source, smoothing drives SS to solve a Poisson law, and dynamics follow g∝∇S\mathbf g\propto\nabla S. Geometry emerges as the fixed point of multi-scale smoothing. That’s “gravity as entropic smoothing,” quantitatively.
The RSVP action (A.1) generates (A.5), with entropy gradients providing the only mesoscopic driving term.
The RSVP action (A.1) generates (A.5), with entropy gradients providing the only mesoscopic driving term.
* TARTAN turns RSVP fluxes into bistochastic flows; RSVP phases supply a unistochastic unitary UU with ∣U∣2=B|U|^2=B, giving an emergent Born rule.
* The entropy field S(ℓ)S^{(\ell)} obeys a Poisson law sourced by mass, making gravity precisely entropic smoothing with g=Θν∇S(ℓ)\mathbf g=\frac{\Theta}{\nu}\nabla S^{(\ell)} and Δφ=4πGeffρ\Delta\varphi=4\pi G_\text{eff}\rho.
* In categorical/sheaf terms, coarse-graining is a functor Cℓ:Sect(F) ⁣→ ⁣Stoch\mathsf C_\ell:\mathrm{Sect}(\mathcal F)\!\to\!\mathbf{Stoch}, quantization is Qℓ:Sect(F) ⁣→ ⁣Hilb\mathsf Q_\ell:\mathrm{Sect}(\mathcal F)\!\to\!\mathbf{Hilb}, and the Born natural transformation ∣⋅∣2:Hilb ⁣⇒ ⁣Stoch|\cdot|^2:\mathbf{Hilb}\!\Rightarrow\!\mathbf{Stoch} commutes with composition and cover refinement.
\section{Discussion and Outlook}
RSVP resolves anomalies by modeling cosmology as entropic recursion in a static plenum, extending Jacobson, Verlinde, and Padmanabhan's works rather than resurrecting Einstein's static model.
Unique predictions: Torsion-induced spin alignments, entropy-gradient Hubble variance. For example, spin alignments could be detected as correlations in galaxy rotation vectors, while Hubble variance manifests as line-of-sight dependent $H_0$ measurements.
Falsifiability: Mismatches in SN Ia z>2 curves, CMB lensing without peaks, or BAO scales incompatible with $S$-oscillations would falsify. $\Lambda$CDM outperforms in precise BBN; RSVP needs tighter constraints for parity. Specifically, if JWST observations show no deviations in high-z BAO, or if void lensing profiles match $\Lambda$CDM exactly, RSVP would be ruled out.
Objections: Early nucleosynthesis fits via adjusted $\Phi$ pumping; CMB polarization unchanged at linear order. The model accommodates BBN by modulating reaction rates through entropy-dependent barriers.
Deeper unification: RSVP may emerge from quantum gravity, e.g., loop quantum cosmology with torsion. Next steps: GPU simulations ($512^3$), Boltzmann coupling for CMB, and survey tests for falsification. These simulations will allow detailed predictions for upcoming data from Euclid and SKA.
\appendix
\section{Mathematical Appendix}
\subsection{RSVP Field Theory: Action, Euler-Lagrange Equations, and Constitutive Closure}
\subsubsection{Fields and Kinematics}
Let $(\Phi, \bm{v}, S)$ be scalar density, velocity, and entropy fields on a 3-manifold $M$ with coordinates $\bm{x}$ and time $t$. Define mass flux $\bm{J} := \Phi \bm{v}$. Impose continuity via Lagrange multiplier $\lambda$:
\[
\mathcal{C}(\Phi, \bm{v}) := \partial_t \Phi + \nabla \cdot (\Phi \bm{v}) = 0.
\]
Introduce gauge 1-form $\mathcal{A}$ for phase/rotation control:
\[
\bm{v} = \nabla \chi + \alpha \nabla \beta + \bm{v}_\perp, \quad \nabla \cdot (\Phi \bm{v}_\perp) = 0.
\]
\subsubsection{Action}
The action is:
\[
\mathcal{A}[\Phi, \bm{v}, S, \lambda] = \int dt \int_M \left( \frac{\nu}{2} \Phi |\bm{v}|^2 - \frac{\kappa}{2} |\nabla \Phi|^2 - \Theta \Phi S + \lambda (\partial_t \Phi + \nabla \cdot (\Phi \bm{v})) \right) d^3x,
\]
with $\nu, \kappa, \Theta > 0$. Include dissipation for $S$ via Rayleigh functional $\mathcal{R}[S_t] := \frac{1}{2} \int_M \zeta (\partial_t S - \mathcal{D} \Delta S - \Sigma)^2 d^3x$, where $\mathcal{D} > 0$ is entropy diffusivity and $\Sigma \geq 0$ an entropy production source.
\subsubsection{Variations}
Varying w.r.t. $\lambda$ gives $\partial_t \Phi + \nabla \cdot (\Phi \bm{v}) = 0$.
Varying w.r.t. $\bm{v}$ (after integration by parts):
\[
\nu \Phi \bm{v} + \lambda \nabla \Phi + \Phi \nabla \lambda = 0 \implies \nu \Phi \frac{D \bm{v}}{Dt} = -\nabla (\Phi \partial_t \lambda) - \Phi \nabla \left( \frac{|\nabla \lambda|^2}{2\nu} \right),
\]
yielding, with $\partial_t \lambda = \Theta S - \kappa \Delta \Phi / \Phi$:
\[
\nu \Phi \frac{D \bm{v}}{Dt} = -\kappa \nabla (\Delta \Phi) + \Theta \Phi \nabla S.
\]
Varying w.r.t. $\Phi$:
\[
-\partial_t \lambda - \bm{v} \cdot \nabla \lambda - \kappa \Delta \Phi - \Theta S + \frac{\nu}{2} |\bm{v}|^2 = 0.
\]
For $S$, $\delta (\mathcal{A} - \mathcal{R}) / \delta S = 0$ gives:
\[
\partial_t S + \bm{v} \cdot \nabla S = \mathcal{D} \Delta S + \Sigma (\Phi, \nabla \bm{v}, \dots), \quad \Sigma \geq 0.
\]
Summary (RSVP PDEs):
\[
\boxed{
\begin{aligned}
&\partial_t \Phi + \nabla \cdot (\Phi \bm{v}) = 0, \\
&\nu \Phi \frac{D \bm{v}}{Dt} = -\kappa \nabla (\Delta \Phi) + \Theta \Phi \nabla S, \\
&\partial_t S + \bm{v} \cdot \nabla S = \mathcal{D} \Delta S + \Sigma (\geq 0).
\end{aligned}
}
\]
\subsection{TARTAN Coarse-Graining and the Transfer Operator}
Let $\{T_i^{(\ell)}\}_{i=1}^{n_\ell}$ be a recursive tiling (scale $\ell$). Define tile mass and flux over $\Delta t$:
\[
\mu_i = \int_{T_i} \Phi, \quad F_{i \to j} = \int_t^{t+\Delta t} \int_{\partial T_i \cap \partial T_j} (\Phi \bm{v}) \cdot d\bm{a} \, dt'.
\]
The row-stochastic kernel is:
\[
P_{ij} = \frac{F_{i \to j}}{\sum_k F_{i \to k}}, \quad \sum_j P_{ij} = 1.
\]
TARTAN augments $F_{i \to j}$ with trajectory bundles and phase tags (circulations of $\mathcal{A}$ along paths), and smooths each tile via Gaussian auras.
\subsection{From $P$ to Bistochastic $B$ and Unistochastic Unitary $U$}
Entropy-balanced bistochasticization (Sinkhorn):
\[
B = D_L P D_R, \quad D_L, D_R > 0,
\]
with $D_L, D_R$ enforcing $\sum_j B_{ij} = 1 = \sum_i B_{ij}$.
Unistochastic lift: $B \in \mathbb{R}^{n \times n}$ is unistochastic if $\exists U \in U(n)$ with $|U_{ij}|^2 = B_{ij}$. Set:
\[
U_{ij} = \sqrt{B_{ij}} e^{i \theta_{ij}}, \quad \sum_k U_{ik} \overline{U}_{jk} = \delta_{ij}.
\]
Orthogonality constraints:
\[
\forall i < j: \sum_k \sqrt{B_{ik} B_{jk}} e^{i (\theta_{ik} - \theta_{jk})} = 0. \tag{★}
\]
RSVP phases:
\[
\theta_{ik} = \frac{1}{\hbar_{\eff}} \oint_{\gamma(i \to k)} \mathcal{A} \cdot d\bm{x}.
\]
Under bounded torsion, phases satisfy (★); otherwise, minimize violation.
Define $\mathcal{H}^{(\ell)} \cong \mathbb{C}^{n_\ell}$ with basis $\{|i\rangle\}$. Coarse amplitudes $\psi_i = \sqrt{\mu_i} e^{i \varphi_i}$. Evolve $\psi' = U \psi$; $\Pr(j) = |\psi'_j|^2$ yields Born rule.
Examples: 2×2 (every $B$ unistochastic); 3×3 (triangle closure via vorticity polytope).
Transfer-operator: $\mathsf{T}$ yields $\mathsf{B}$; Stinespring dilation gives $\mathsf{U}$ with $|U_{ij}|^2 = B_{ij}$.
\subsection{Gravity as Entropic Smoothing}
Tile-entropy Poisson law (quasi-static, $\Sigma \approx \alpha \rho^{(\ell)}$):
\[
\Delta S^{(\ell)} = -\frac{\alpha}{\mathcal{D}^{(\ell)}} \rho^{(\ell)}.
\]
Potential and acceleration:
\[
\bm{g} := \frac{d \bm{v}}{dt} = \frac{\Theta}{\nu} \nabla S^{(\ell)} = -\nabla \varphi, \quad \varphi := -\frac{\Theta}{\nu} S^{(\ell)}.
\]
Laplacian:
\[
\Delta \varphi = 4\pi G_{\eff} \rho, \quad 4\pi G_{\eff} := \frac{\Theta}{\nu} \frac{\alpha}{\mathcal{D}^{(\ell)}}.
\]
Rays/redshift: $n(\bm{x}) = \exp(\gamma S^{(\ell)}(\bm{x}))$; deflection $\propto \nabla_\perp \varphi$; $\Delta \nu / \nu \approx -(\gamma / 2) \Delta S^{(\ell)} \propto - \varphi$.
Curvature: Metric $g_{ab}^{(\ell)} = \mathbb{E}[(\delta x_a)(\delta x_b)]_{\text{TARTAN paths}}$ evolves as $\partial_\tau g_{ab}^{(\ell)} = -2\lambda \Ric_{ab} + \cdots$, fixed points align geodesics with $\nabla S^{(\ell)}$.
Quantum coherence stabilizes: $\mathcal{D}^{(\ell)}_{\eff} \downarrow$ with coherence, Geff $\uparrow$ in ordered regions.
Operational meaning: Mass sources entropy; smoothing spreads $S$; motion follows $\nabla S$; geometry is Ricci-flow fixed point.
Simulator checklist: Solve $\Delta S^{(\ell)} = -(\alpha / \mathcal{D}^{(\ell)}) \rho^{(\ell)}$; set $\varphi = -(\Theta / \nu) S^{(\ell)}$; advance with $\dot{\bm{v}} = -\nabla \varphi$; rays via $n = \exp(\gamma S^{(\ell)})$; evolve $g^{(\ell)}$ by (11).
\subsection{Categorical/Sheaf Structure}
Open-set category $\Open(M)$: objects opens $U \subseteq M$, morphisms inclusions.
Stoch: finite sets with row-stochastic matrices.
Hilb: finite-dimensional Hilbert spaces with linear maps.
Presheaf $\mathcal{F}: \Open(M)^{\op} \to \Set$, $U \mapsto \{\Phi|_U, \bm{v}|_U, S|_U\}$.
Coarse-graining functor $C_\ell: \Sect(\mathcal{F}) \to \Stoch$, $(\Phi, \bm{v}, S) \mapsto P^{(\ell)} \mapsto B^{(\ell)}$.
Quantization $Q_\ell: \Sect(\mathcal{F}) \to \Hilb$, $(\Phi, \bm{v}, S) \mapsto (\mathcal{H}_\ell \simeq \mathbb{C}^{n_\ell}, U^{(\ell)})$.
Natural transformation $|\cdot|^2: \Hilb \to \Stoch$ commutes with composition and refinement.
\bibliographystyle{unsrt}
\bibliography{references}
\end{document}
