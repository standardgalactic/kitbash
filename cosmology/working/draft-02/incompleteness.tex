\documentclass{article}
\usepackage{amsmath}
\usepackage{amsfonts}
\usepackage{amssymb}
\usepackage{geometry}
\geometry{a4paper, margin=1in}
\usepackage{noto}
\usepackage{natbib}
\usepackage{enumitem}

\begin{document}

\title{Incompleteness, Null Signals, and the Cosmological Plenum}
\author{}
\date{\today}
\maketitle

\begin{abstract}
This paper explores incompleteness as a unifying principle across logic, computation, cognition, thermodynamics, and cosmology. Beginning with G\"{o}del's incompleteness theorems, it examines how formal systems are inherently limited by their inability to encompass their embedding supersets. Karl Fant's null convention logic (NCL) operationalizes this in hardware, using null states to signal incompletion, with parallels in everyday life such as learner's permits and amber traffic lights. Monica Anderson's critiques of Good Old-Fashioned Artificial Intelligence (GOFAI) highlight similar limitations in symbolic systems, advocating for sub-symbolic adaptability. Functional programming's monadic deferral of side-effects mirrors Ilya Prigogine's dissipative structures, which export entropy to maintain order. The Relativistic Scalar-Vector-Potential (RSVP) model reframes cosmic expansion as an entropic ``falling outward,'' constrained by Cosmic Microwave Background (CMB) dipole observations to disfavor multiverse scenarios with varying Friedmann-Robertson-Walker-Lema\^{i}tre (FRW) parameters. A categorical framework, the Law of Superset Entropy, unifies these domains: order requires exporting incompleteness to a superset or internalizing it via structural reorganization. The synthesis posits that the cosmos, like logic and computation, is a plenum of null signals, perpetually incomplete yet coherent within a greater whole.
\end{abstract}

\section{Introduction}
\label{sec:intro}
Kurt G\"{o}del's incompleteness theorems revealed a profound limitation: any formal system sufficiently expressive to encode arithmetic contains true propositions that cannot be proven within its axioms \citep{godel1931}. This structural feature of categoricity---defining a domain excludes its embedding metalevel---extends beyond mathematics to computation, cognition, thermodynamics, and cosmology. This paper traces the theme of incompleteness across these domains, linking G\"{o}del's logical insights, Karl Fant's null convention logic (NCL), Monica Anderson's critiques of Good Old-Fashioned Artificial Intelligence (GOFAI), functional programming's monadic structures, Ilya Prigogine's dissipative structures, and the Relativistic Scalar-Vector-Potential (RSVP) cosmological model.

The argumentative arc progresses as follows: Section \ref{sec:godel} establishes G\"{o}delian incompleteness as a foundational limit; Section \ref{sec:ncl} details NCL as its computational embodiment, with prerequisites and everyday analogues; Section \ref{sec:programming_entropy} connects functional programming and Prigogine's entropy export; Section \ref{sec:rsvp} introduces RSVP's entropic cosmology; Section \ref{sec:dipole} applies CMB dipole constraints to reject multiverse variability; Section \ref{sec:synthesis} unifies these under a Law of Superset Entropy; and two appendices provide technical details on RSVP constraints and the ``falling outward'' analogy. The synthesis argues that incompleteness is not a flaw but an architectural necessity, with null signals ensuring continuity across domains, as exemplified by the scriptural metaphor of Matthew 26:29.

\section{G\"{o}delian Incompleteness and the Limits of Categories}
\label{sec:godel}
G\"{o}del's incompleteness theorems (1931) demonstrate that any formal system capable of encoding arithmetic contains propositions that are true but unprovable within its axioms \citep{godel1931}. Through a process of self-reference, G\"{o}del constructs a statement that asserts its own unprovability, which is true if and only if it is not provable, exposing the system's limits. This phenomenon, as clarified by \citet{nagel1958}, arises from the system's ability to encode statements about itself, akin to Russell's paradox in set theory, where the set of all sets leads to contradiction if it contains itself \citep{tarski1956}.

This self-referential folding is a structural feature of categoricity: defining a category---a set, logic, or computational model---excludes the metalevel conditions that embed it. As \citet{tarski1956} notes, truth in a formal language requires a meta-language, reinforcing that systems are blind to their supersets. Extending a system with new axioms only introduces new undecidables, leading to a pluralism of consistent frameworks \citep{nagel1958}. For example, the Continuum Hypothesis is independent of Zermelo-Fraenkel set theory, illustrating that no single system exhausts mathematical truth.

The philosophical implication is that all ordered systems are provisional, locally coherent, and globally incomplete. They require mechanisms to signal or manage their boundaries, whether through unprovable propositions or external coordinators. This insight sets the stage for examining incompleteness in computational and physical systems, where analogous boundaries emerge.

\section{Null Convention Logic as Embodied Incompleteness}
\label{sec:ncl}
Karl Fant's \textit{Computer Science Reconsidered} (2005) operationalizes G\"{o}delian incompleteness in hardware through null convention logic (NCL), a paradigm for delay-insensitive circuits \citep{fant2005}. Unlike synchronous logic, which relies on an external clock to enforce timing \citep{seitz1980}, NCL introduces a null state to signal incompletion, enabling circuits to self-coordinate. This section provides a detailed introduction to NCL, including prerequisites for readers unfamiliar with computation, and connects it to G\"{o}delian limits through everyday analogues and parallels with cognitive systems.

\subsection{Prerequisites: Synchronous vs. Asynchronous Logic}
Traditional digital circuits operate synchronously, using a global clock to synchronize operations. Each gate processes inputs assuming they are valid at clock ticks, but varying propagation delays can cause glitches or hazards \citep{seitz1980}. For example, in a Boolean AND gate with inputs A and B, if A arrives before B due to delay, the gate may output an erroneous transient state. Synchronous systems externalize coordination to the clock, akin to a mathematician managing data flow \citep{fant2005}.

Asynchronous circuits, by contrast, operate without a global clock, relying on local handshaking to ensure correct operation. This eliminates timing assumptions but requires explicit mechanisms to signal data validity. NCL, building on early asynchronous designs \citep{seitz1980}, introduces a null state to achieve this, making circuits delay-insensitive and robust to propagation variations.

\subsection{NCL Fundamentals}
NCL extends Boolean logic by incorporating a null state, creating a three-value (True, False, Null) or four-value (True, False, Null, Intermediate) system \citep{fant2005}. The key principle is the completeness criterion: a gate outputs a valid data value (True or False) only when all inputs are valid data values; otherwise, it outputs Null. This ensures no premature computation, eliminating hazards.

- **Three-Value Logic**: A gate's truth table is extended to include Null. For an AND gate, if inputs are (True, True), (True, False), (False, True), or (False, False), it outputs the Boolean result; if either input is Null, it outputs Null \citep{fant2005}. This enforces input completeness.

- **Four-Value Logic**: Adding an Intermediate value ensures both data-to-null and null-to-data transitions are complete. A gate outputs Null only when all inputs are Null, and data only when all inputs are data, with Intermediate for mixed states \citep{fant2005}.

- **Two-Value Implementation**: For hardware practicality, NCL uses dual-rail encoding: two wires represent one logical signal (e.g., (1,0) for True, (0,1) for False, (0,0) for Null, (1,1) illegal). Mutually exclusive assertion groups ensure only one wire in a group carries data \citep{fant2005}.

The data resolution wavefront propagates when all inputs are valid, and a null wavefront resets the system, forming a self-synchronizing cycle. For example, in a combinational circuit, if one input remains Null, at least one output remains Null, ensuring the system waits for complete data before proceeding \citep{fant2005}.

\subsection{Null-Data Cycle and Wavefront}
The null-data cycle is central to NCL's operation. Circuits start in an all-Null state, transition to all-data when inputs are valid, and return to all-Null before the next cycle \citep{fant2005}. This cycle is analogous to a handshake: the system signals its readiness (Null) and completion (data). The wavefront of validity propagates symbolically, not temporally, eliminating the need for a clock.

In the four-value system, the Intermediate state ensures robust transitions. For instance, a gate with mixed inputs (some Null, some data) outputs Intermediate, which is ignored by downstream gates until full data or Null states are reached. This mirrors the ``shaking bag'' analogy in \citet{fant2005}, where symbols interact spontaneously, resolving only when conditions are complete, akin to biological systems self-organizing without external control \citep{schneider2005}.

\subsection{Everyday Analogues}
NCL's null state mirrors everyday control signals, illustrating its intuitive grounding:

\begin{itemize}
    \item \textit{Do Not Disturb Sign}: Signals suspension of action, akin to Null preventing premature gate output. It informs others not to interpret the current state as actionable \citep{fant2005}.
    \item \textit{Learner's Permit}: Marks an incomplete state, allowing limited driving under supervision, like Null awaiting full data validity. It signals partial readiness, deferring full authorization \citep{fant2005}.
    \item \textit{Amber Traffic Light}: Indicates a transitional state, neither go nor stop, akin to NCL's Intermediate value signaling partial readiness before a full state change \citep{fant2005}.
    \item \textit{Matthew 26:29}: Jesus' refusal to drink wine until the kingdom signifies a null state, a suspension awaiting fulfillment, with the future act signaling completion.
\end{itemize}

These analogues highlight null as a meta-signal of incompleteness, preventing misinterpretation, much like G\"{o}del's unprovable statements expose systemic limits.

\subsection{GOFAI Critiques}
Monica Anderson's critiques of GOFAI align with NCL's philosophy \citep{anderson2006}. GOFAI relies on explicit, programmer-defined rules, exporting semantic burden to human designers, rendering systems brittle in ambiguous contexts. Sub-symbolic approaches, such as neural networks, embrace partial states, learning from data without predefined ontologies \citep{clark2013, hohwy2013}. This mirrors NCL's null, which allows circuits to defer action until inputs are complete, internalizing coordination.

\subsection{Philosophical Resonance}
NCL embodies G\"{o}delian incompleteness: synchronous logic hides timing in a superset (clock), while NCL internalizes coordination via null states. Fant's critique parallels G\"{o}del: conventional computer science assumes an external ``mathematician'' (clock or engineer) to ensure correctness, but NCL embeds this awareness in the system, akin to G\"{o}del's self-referential statements \citep{fant2005, nagel1958}. This shift from external to internal coordination is a computational solution to incompleteness, setting the stage for broader applications.

\section{Functional Programming and Deferred Entropy}
\label{sec:programming_entropy}
Functional programming, particularly in pure languages like Haskell, manages incompleteness by deferring side-effects through monads, preserving referential transparency until interaction with the external world \citep{wadler1992, turner1979}. This section explores this parallel and extends it to thermodynamic systems.

\subsection{Monads in Functional Programming}
In imperative programming, side-effects (e.g., I/O, state changes) contaminate composition, requiring external coordination (e.g., runtime systems). Pure functional languages encapsulate side-effects in monads, such as the IO monad, which describes computations without executing them until runtime \citep{wadler1992}. For example, a function \(f : X \to \text{IO } Y\) delays printing until the monad is evaluated, ensuring composition remains pure.

This mirrors NCL's null-data cycle: just as NCL gates wait for complete inputs, monads defer side-effects, internalizing the coordination that synchronous systems externalize to a runtime \citep{turner1979}. The monad's bind operator (\(\gg=\)) sequences effects, analogous to NCL's wavefront propagation, ensuring deterministic behavior without external timing.

\subsection{Dissipative Structures and Entropy Export}
Ilya Prigogine's dissipative structures provide a thermodynamic analogue \citep{prigogine1984, nicolis1977}. Systems like hurricanes, organisms, or economies maintain internal order by exporting entropy to their environment. For instance, a hurricane organizes itself by dissipating heat into the atmosphere, increasing global entropy (\(\Delta S_{\text{universe}} = \Delta S_{\text{system}} + \Delta S_{\text{environment}} \geq 0\)) \citep{nicolis1977}. Similarly, biological systems sustain complexity by expelling disorder \citep{schneider2005}.

In computation, garbage collection externalizes memory cleanup to a runtime process; in economics, platforms like Uber offload labor and maintenance costs to drivers. These systems defer incompleteness, exporting entropy to a superset, much like G\"{o}delian systems rely on metalevels for truth \citep{tarski1956}.

\subsection{Law of Superset Entropy}
The parallels suggest a unifying principle, the Law of Superset Entropy: any ordered system maintains coherence by exporting incompleteness or entropy to a superset, or internalizing it through structural reorganization. In functional programming, monads internalize side-effects; in NCL, null states internalize timing; in dissipative structures, internal organization absorbs some entropy, reducing external export \citep{wadler1992, fant2005, prigogine1984]. This law frames incompleteness as a trade-off between external dependence and internal complexity.

\section{RSVP and the Cosmological Extension}
\label{sec:rsvp}
The Relativistic Scalar-Vector-Potential (RSVP) model reinterprets cosmic redshift as an entropic relaxation process, ``falling outward,'' rather than metric expansion \citep{whittle2015}. This section details RSVP's fields and dynamics, connecting them to the incompleteness theme.

\subsection{RSVP Fields and Dynamics}
RSVP posits three interacting fields:

\begin{itemize}
    \item \textit{Scalar \(\phi(x,\eta)\)}: Drives entropic smoothing, analogous to NCL's null state, deferring structural relaxation.
    \item \textit{Vector \(\mathbf{v}(x,\eta)\)}: Represents bulk flows, akin to local deviations in computational systems.
    \item \textit{Entropy \(S(x,\eta)\)}: Quantifies disorder, redistributed during structure formation.
\end{itemize}

The entropic redshift potential is:
\[
\Upsilon \equiv \delta\phi - \beta(\eta)\varphi_m, \quad \varphi_m(k,\eta) = \frac{4\pi G a^2(\eta) \bar{\rho}_m(\eta)}{k^2} \mathcal{T}_m(k,\eta) \delta_m(k,\eta).
\]
Here, \(\delta\phi\) is the scalar perturbation, \(\varphi_m\) the dimensionless matter potential, and \(\beta(\eta)\) a time-dependent coupling \citep{dodelson2003}. \(\phi\) mediates void expansion, explaining acceleration without invoking exotic vacuum energy \citep{einstein1917, desitter1917, peebles2003}.

\subsection{Falling Outward}
Mark Whittle's ``falling outward'' analogy describes vacuum energy driving expansion by reducing gravitational potential \citep{whittle2015, whittle2006}. In a matter-dominated sphere, the potential energy \(U = -\frac{G M m}{r}\) becomes less negative as radius \(r\) increases, favoring collapse. In a vacuum-dominated sphere, mass scales as \(M \propto r^3\), making \(U \propto -r^2\), so expansion is energetically favorable. RSVP reframes this: \(\phi\) drives voids to ``smooth outward,'' generating entropy to fill expanded volumes, a self-sustaining process akin to Prigogine's dissipative structures \citep{prigogine1984, whittle2015}.

\subsection{Inflationary Extension}
In the early universe, a high-density \(\phi\) triggers a rapid ``lamphron-lamphrodyne flash,'' an explosive entropic smoothing akin to inflation \citep{whittle2015}. This creates causal uniformity, seeding later structure formation, and parallels the runaway expansion of a dense vacuum \citep{desitter1917}.

\section{Dipole Constraints and the Absence of Multiverses}
\label{sec:dipole}
RSVP imposes stringent constraints on beyond-horizon inhomogeneities through CMB dipole observations (Appendix A). A significant density or entropic gradient would produce a residual dipole misaligned with local gravitational flows (e.g., Great Attractor) \citep{riess1998, perlmutter1999}. The observed dipole (\(\varepsilon_{\rm kin} \sim 10^{-3}\)) aligns with reconstructions, and the intrinsic dipole is bounded by:
\[
\Delta \Upsilon_* \equiv \|\nabla \Upsilon_*\| R_* \lesssim \text{few} \times 10^{-5}.
\]
This implies isotropy and homogeneity extend at least one horizon length beyond the observable, disfavoring bubble universes with varying FRW parameters \citep{dodelson2003}. The universe is constrained by inherent conditions, not sampling arbitrary parameters, resonating with predictive processing models in cognition that minimize environmental uncertainty \citep{clark2013, hohwy2013].

\section{Synthesis}
\label{sec:synthesis}
Incompleteness is an architectural necessity across domains. G\"{o}del's theorems reveal logical limits; NCL's null signals enable self-coordinating circuits; GOFAI critiques advocate adaptive systems; monads defer side-effects; dissipative structures export entropy; and RSVP's \(\phi\) drives cosmic relaxation. The Law of Superset Entropy unifies these: order requires exporting incompleteness to a superset or internalizing it through reorganization.

In Matthew 26:29, Jesus' refusal to drink wine until the kingdom signifies a null state, fulfilled by future action, opening a larger incompleteness. The cosmos, as a plenum of null signals, falls outward within a greater whole, never fully complete yet coherent.

\section{Appendix A: CMB Dipole Constraints in RSVP (Entropic Redshift Form)}
\subsection{A.1 Fields, Normalization, and \texorpdfstring{\(\Lambda\)}{Lambda}CDM Dictionary}
RSVP employs:
\begin{itemize}
    \item Scalar capacity \(\phi(x,\eta)\) (drives falling outward),
    \item Matter density \(\rho_m(x,\eta)\),
    \item Bulk flow \(\mathbf{u}(x,\eta)\) (peculiar velocity).
\end{itemize}
The entropic redshift potential is:
\[
\boxed{\Upsilon \equiv \delta\phi - \beta(\eta)\,\varphi_m, \quad \varphi_m(k,\eta) = \frac{4\pi G\,a^2(\eta)\,\bar{\rho}_m(\eta)}{k^2}\,\mathcal{T}_m(k,\eta)\,\delta_m(k,\eta)}
\]
Normalized such that the instantaneous Sachs--Wolfe contribution at decoupling is:
\[
\boxed{\left(\frac{\Delta T}{T}\right)_{\rm SW} = \frac{1}{3}\Upsilon_*}
\]
In the \texorpdfstring{\(\Lambda\)}{Lambda}CDM limit, \(\Upsilon \to \Phi_N\) (Newtonian potential), with \(\alpha_\phi = 1\), \(\alpha_m = \frac{4\pi G a^2 \bar{\rho}_m}{k^2} \mathcal{T}_m(k,\eta)\).

\subsection{A.2 Large-Angle Anisotropy}
For a sightline \(\hat{\mathbf{n}}\):
\[
\frac{\Delta T}{T}(\hat{\mathbf{n}}) = \underbrace{\hat{\mathbf{n}} \cdot \frac{\mathbf{u}_0}{c}}_{\text{kinematic dipole } \varepsilon_{\rm kin} \sim 10^{-3}} + \underbrace{\frac{1}{3}\Upsilon_*(\hat{\mathbf{n}})}_{\text{entropic SW}} + \underbrace{2 \int_{\eta_*}^{\eta_0} \dot{\Upsilon} \, d\eta}_{\text{entropic ISW}}
\]
The intrinsic dipole is:
\[
\left| \left( \frac{\Delta T}{T} \right)_{\ell=1}^{\rm int} \right| \equiv \varepsilon_{\rm int} \lesssim \text{few} \times 10^{-5}
\]

\subsection{A.3 Super-Horizon Gradient Bound}
A super-horizon inhomogeneity is modeled as:
\[
\Upsilon(\mathbf{x},\eta) \simeq \Upsilon_0(\eta) + \mathbf{G}(\eta) \cdot \mathbf{x}, \quad \|\mathbf{G}\| R_* \ll 1
\]
The intrinsic dipole contribution is:
\[
D_{\rm int} \approx \frac{1}{3} \|\mathbf{G}_*\| R_* + \mathcal{O}\left( \int \dot{\Upsilon} \, d\eta \right)
\]
with bounds:
\[
\boxed{\|\nabla \Upsilon_*\| R_* \lesssim 3 \varepsilon_{\rm int}} \quad \Longleftrightarrow \quad \boxed{\|\mathbf{G}_*\| \lesssim \frac{3 \varepsilon_{\rm int}}{R_*}}
\]

\subsection{A.4 Effective Potential}
The effective potential is:
\[
\boxed{\mathbf{a}_{\rm eff} = -\nabla \Phi_{\rm eff}, \quad \Phi_{\rm eff} \equiv \phi - \gamma(\eta) \varphi_m}
\]
with bounds:
\[
\boxed{|\delta\phi|_* \lesssim \varepsilon_{\rm int}, \quad |\delta\rho_m|_* \lesssim \frac{\varepsilon_{\rm int}}{\alpha_m}}
\]

\subsection{A.5 Bulk-Flow Convergence}
The RSVP bulk-flow estimator is:
\[
\mathbf{u}_0^{\rm RSVP}(R) := \arg\min_{\mathbf{u}} \sum_{i: r_i < R} w_i \left( z_i^{\rm obs} - z_i^{\rm RSVP}(\mathbf{u}) \right)^2, \quad w_i \propto \frac{1}{\sigma_{S,i}}
\]
Converging as:
\[
\angle \left( \mathbf{u}_0^{\rm RSVP}(R), \mathbf{d}_{\rm CMB} \right) \to 0, \quad \|\mathbf{u}_0^{\rm RSVP}(R)\| \to c \varepsilon_{\rm kin}
\]

\subsection{A.6 Long-Mode Consistency}
Super-horizon adiabatic \(\Upsilon\) modes induce a semantic-slicing gauge transformation, canceling the leading dipole. Residuals are tied to the quadrupole via \(\dot{\Upsilon}\) at horizon entry. The observed quadrupole (\(\sim 10^{-5}\)) tightens the bound.

\subsection{A.7 Conclusion}
The residual dipole limit:
\[
\Delta \Upsilon_* \equiv \|\nabla \Upsilon_*\| R_* \lesssim \text{few} \times 10^{-5}
\]
and bulk-flow alignment indicate homogeneity extends beyond the observable horizon, disfavoring bubble universes.

\section{Appendix B: Falling Outward in the RSVP Framework}
Consider a spherical region in the RSVP plenum with a test particle at its boundary, governed by \(\phi\), \(\mathbf{u}\), and \(S\).

\textbf{Case 1: Matter-Dominated Sphere.} For matter-rich regions, the effective energy is:
\[
E \sim -\frac{G M m}{r},
\]
driving inward collapse.

\textbf{Case 2: Entropic Vacuum-Dominated Sphere.} In void-like regions dominated by \(\phi\), the ``mass'' scales as \(M(r) \propto r^3\), yielding:
\[
E \sim -r^2
\]
This drives outward relaxation, with entropy generated by the fall itself.

\textbf{Inflationary Extension.} In the early plenum, a high-entropy \(\phi\) dominates, producing a lamphron-lamphrodyne flash:
\[
E \sim - \rho_{\phi} r^2, \quad \frac{d^2 r}{dt^2} \propto \rho_{\phi} r
\]
This rapid smoothing establishes causal uniformity, prefiguring void expansion. The CMB dipole constraint (\(\Delta \Upsilon_* \lesssim \text{few} \times 10^{-5}\)) ensures this coherence persists, ruling out arbitrary parameter variations.

\bibliographystyle{plainnat}
\bibliography{references}
\end{document}
