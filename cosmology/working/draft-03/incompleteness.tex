\documentclass{article}
\usepackage{amsmath}
\usepackage{amsfonts}
\usepackage{amssymb}
\usepackage{geometry}
\geometry{a4paper, margin=1in}
\usepackage{noto}
\usepackage{natbib}
\usepackage{enumitem}
\usepackage{sloppypar} % To help with overfull hboxes

\begin{document}

\title{Incompleteness, Null Signals, and the Cosmological Plenum}
\author{Flyxion}
\date{\today}
\maketitle

\begin{abstract}
This paper explores incompleteness as a unifying principle across logic, computation, cognition, thermodynamics, and cosmology. Beginning with G\"{o}del's incompleteness theorems, it examines how formal systems are inherently limited by their inability to encompass their embedding supersets. Karl Fant's null convention logic (NCL) operationalizes this in hardware, using null states to signal incompletion, with parallels in everyday life such as learner's permits and amber traffic lights. Monica Anderson's critiques of Good Old-Fashioned Artificial Intelligence (GOFAI) highlight similar limitations in symbolic systems, advocating for sub-symbolic adaptability. Functional programming's monadic deferral of side-effects mirrors Ilya Prigogine's dissipative structures, which export entropy to maintain order. The Relativistic Scalar-Vector-Potential (RSVP) model reframes cosmic expansion as an entropic ``falling outward,'' constrained by Cosmic Microwave Background (CMB) dipole observations to disfavor multiverse scenarios with varying Friedmann-Robertson-Walker-Lema\^{i}tre (FRW) parameters. A categorical framework, the Law of Superset Entropy, unifies these domains: order requires exporting incompleteness to a superset or internalizing it via structural reorganization. The synthesis posits that the cosmos, like logic and computation, is a plenum of null signals, perpetually incomplete yet coherent within a greater whole.
\end{abstract}

\section{Introduction}
\label{sec:intro}
Kurt G\"{o}del's incompleteness theorems revealed a profound limitation: any formal system sufficiently expressive to encode arithmetic contains true propositions that cannot be proven within its axioms \citep{godel1931}. This structural feature of categoricity---defining a domain excludes its embedding metalevel---extends beyond mathematics to computation, cognition, thermodynamics, and cosmology. This paper traces the theme of incompleteness across these domains, linking G\"{o}del's logical insights, Karl Fant's null convention logic (NCL), Monica Anderson's critiques of Good Old-Fashioned Artificial Intelligence (GOFAI), functional programming's monadic structures, Ilya Prigogine's dissipative structures, and the Relativistic Scalar-Vector-Potential (RSVP) cosmological model.

The paper is structured to build progressively from foundational logic to cosmic scales. Section \ref{sec:godel} delves into G\"{o}delian incompleteness, explaining self-reference and its implications for formal systems. Section \ref{sec:ncl} provides a comprehensive introduction to NCL, including prerequisites on synchronous and asynchronous logic, detailed mechanics, and everyday analogues to illustrate its embodiment of incompleteness. Section \ref{sec:programming_entropy} expands on functional programming's handling of side-effects and Prigogine's thermodynamic principles, with examples from computation and economics. Section \ref{sec:rsvp} details RSVP's fields, dynamics, and ``falling outward'' analogy, drawing historical and observational context. Section \ref{sec:dipole} analyzes CMB dipole constraints, arguing against multiverse variability. Section \ref{sec:synthesis} synthesizes these under the Law of Superset Entropy, with philosophical reflections. Appendices provide technical details on RSVP constraints and the ``falling outward'' mechanism.

This interdisciplinary approach reveals incompleteness not as a flaw but as an architectural necessity, with null signals and entropy export ensuring continuity across domains, as exemplified by the scriptural metaphor of Matthew 26:29.

\section{G\"{o}delian Incompleteness and the Limits of Categories}
\label{sec:godel}
G\"{o}del's incompleteness theorems (1931) are foundational to understanding systemic limits \citep{godel1931}. The first theorem states that in any consistent formal system capable of expressing basic arithmetic, there exist propositions that cannot be proved or disproved within the system. The second theorem shows that such a system cannot prove its own consistency. These results arise from self-reference: G\"{o}del constructs a sentence that asserts ``I am unprovable,'' which is true if and only if it is unprovable, creating a paradox analogous to the liar sentence.

As \citet{nagel1958} explain, G\"{o}del uses arithmetization (G\"{o}del numbering) to encode statements about the system within the system itself, turning meta-mathematical questions into arithmetic ones. This self-referential folding is akin to Russell's paradox in set theory, where the set of all sets that do not contain themselves leads to contradiction if it contains itself \citep{tarski1956}. Tarski's work on semantics further shows that truth in a formal language requires a meta-language, reinforcing that systems are blind to their supersets \citep{tarski1956}.

The implications are profound: truth outruns proof. Extending the system with new axioms resolves some undecidables but introduces others, leading to a pluralism of consistent frameworks. For instance, the Continuum Hypothesis is independent of Zermelo-Fraenkel set theory with the axiom of choice (ZFC), meaning it can be neither proved nor disproved from standard axioms \citep{godel1931}. This pluralism underscores that no single system exhausts mathematical truth; each is locally coherent but globally incomplete.

Philosophically, this suggests all ordered systems are provisional, requiring mechanisms to signal boundaries. Incompleteness is not an error but a feature of categoricity: defining a domain excludes its embedding. This baseline informs the subsequent sections, where incompleteness manifests in computational, cognitive, thermodynamic, and cosmological systems.

\section{Null Convention Logic as Embodied Incompleteness}
\label{sec:ncl}
Karl Fant's \textit{Computer Science Reconsidered} (2005) operationalizes G\"{o}delian incompleteness in hardware through null convention logic (NCL), a paradigm for delay-insensitive circuits \citep{fant2005}. Unlike synchronous logic, which relies on an external clock to enforce timing \citep{seitz1980}, NCL introduces a null state to signal incompletion, enabling circuits to self-coordinate. This section provides a detailed introduction to NCL, including prerequisites for readers unfamiliar with computation, and connects it to G\"{o}delian limits through everyday analogues and parallels with cognitive systems.

\subsection{Prerequisites: Synchronous vs. Asynchronous Logic}
To appreciate NCL, it is essential to understand traditional digital logic. Synchronous circuits operate under a global clock, where each gate processes inputs assuming they are valid at clock ticks. However, varying propagation delays can cause glitches or hazards: temporary erroneous outputs that propagate through the system \citep{seitz1980}. For example, in a Boolean AND gate with inputs A and B, if A arrives before B due to delay, the gate may output 0 transiently before correcting to 1 if both are true. Synchronous systems externalize coordination to the clock, akin to a mathematician managing data flow \citep{fant2005}.

Asynchronous circuits eliminate the clock, relying on local handshaking to ensure correct operation. This eliminates timing assumptions but requires explicit mechanisms to signal data validity. Early asynchronous designs, such as those by \citet{seitz1980}, used completion signals, but Fant's NCL generalizes this by integrating null as a fundamental value, making circuits delay-insensitive and robust to propagation variations.

\subsection{NCL Fundamentals}
NCL extends Boolean logic by incorporating a null state, creating a three-value (True, False, Null) or four-value (True, False, Null, Intermediate) system \citep{fant2005}. The key principle is the completeness criterion: a gate outputs a valid data value (True or False) only when all inputs are valid data values; otherwise, it outputs Null. This ensures no premature computation, eliminating hazards.

- **Three-Value Logic**: A gate's truth table is extended to include Null. For an AND gate, if inputs are (True, True), (True, False), (False, True), or (False, False), it outputs the Boolean result; if either input is Null, it outputs Null \citep{fant2005}. This enforces input completeness, preventing partial states from propagating.

- **Four-Value Logic**: Adding an Intermediate value ensures both data-to-null and null-to-data transitions are complete. A gate outputs Null only when all inputs are Null, and data only when all inputs are data, with Intermediate for mixed states \citep{fant2005}. This provides robust symbolic determinacy, ignoring intermediates during monitoring.

- **Two-Value Implementation**: For hardware practicality, NCL uses dual-rail encoding: two wires represent one logical signal (e.g., (1,0) for True, (0,1) for False, (0,0) for Null, (1,1) illegal). Mutually exclusive assertion groups ensure only one wire in a group carries data \citep{fant2005}. Gates become threshold-based: a threshold-N gate outputs data if at least N inputs are data.

The data resolution wavefront propagates when all inputs are valid, and a null wavefront resets the system, forming a self-synchronizing cycle. For example, in a combinational circuit, if one input remains Null, at least one output remains Null, ensuring the system waits for complete data before proceeding \citep{fant2005}. The null-data cycle ensures circuits return to all-Null before the next dataset, preventing state overlap.

In the ``shaking bag'' analogy, Fant describes symbols interacting spontaneously, resolving only when conditions are complete, akin to biological self-organization without external control \citep{fant2005, schneider2005}.

\subsection{Null-Data Cycle and Wavefront}
The null-data cycle is central to NCL's operation. Circuits start in an all-Null state, transition to all-data when inputs are valid, and return to all-Null before the next cycle \citep{fant2005}. This cycle is analogous to a handshake: the system signals its readiness (Null) and completion (data). The wavefront of validity propagates symbolically, not temporally, eliminating the need for a clock.

In the four-value system, the Intermediate state ensures robust transitions. For instance, a gate with mixed inputs (some Null, some data) outputs Intermediate, which is ignored by downstream gates until full data or Null states are reached. This mirrors the ``shaking bag'' analogy in \citet{fant2005}, where symbols interact spontaneously, resolving only when conditions are complete, akin to biological systems self-organizing without external control \citep{schneider2005}.

The feedback solution uses hysteresis: gates ``remember'' their previous state, transitioning only when inputs are fully consistent, making NCL effectively delay-insensitive \citep{fant2005}.

\subsection{Everyday Analogues}
NCL's null state mirrors everyday control signals, illustrating its intuitive grounding:

- Do Not Disturb Sign: Signals suspension of action, akin to Null preventing premature gate output. It informs others not to interpret the current state as actionable \citep{fant2005}.
- Learner's Permit: Marks an incomplete state, allowing limited driving under supervision, like Null awaiting full data validity. It signals partial readiness, deferring full authorization \citep{fant2005}.
- Amber Traffic Light: Indicates a transitional state, neither go nor stop, akin to NCL's Intermediate value signaling partial readiness before a full state change \citep{fant2005}.
- Matthew 26:29: Jesus' refusal to drink wine until the kingdom signifies a null state, a suspension awaiting fulfillment, with the future act signaling completion.

These analogues highlight null as a meta-signal of incompleteness, preventing misinterpretation, much like G\"{o}del's unprovable statements expose systemic limits.

\subsection{GOFAI Critiques}
Monica Anderson's critiques of Good Old-Fashioned Artificial Intelligence (GOFAI) align with NCL's philosophy \citep{anderson2006}. GOFAI relies on explicit, programmer-defined rules, exporting semantic burden to human designers, rendering systems brittle in ambiguous contexts. Sub-symbolic approaches, such as neural networks, embrace partial states, learning from data without predefined ontologies \citep{clark2013, hohwy2013}. This mirrors NCL's null, which allows circuits to defer action until inputs are complete, internalizing coordination.

Anderson argues that GOFAI's modular, rule-based models fail to capture the interactive, reuse-driven nature of cognition, much like synchronous logic fails in asynchronous environments \citep{anderson2006}. Predictive processing models, where brains minimize error via feedback loops, parallel NCL's self-coordination \citep{clark2013, hohwy2013}.

\subsection{Philosophical Resonance}
NCL embodies G\"{o}delian incompleteness: synchronous logic hides timing in a superset (clock), while NCL internalizes coordination via null states. Fant's critique parallels G\"{o}del: conventional computer science assumes an external ``mathematician'' (clock or engineer) to ensure correctness, but NCL embeds this awareness in the system, akin to G\"{o}del's self-referential statements \citep{fant2005, nagel1958}. This shift from external to internal coordination is a computational solution to incompleteness, setting the stage for broader applications in programming and thermodynamics.

\section{Functional Programming and Deferred Entropy}
\label{sec:programming_entropy}
Functional programming, particularly in pure languages like Haskell, manages incompleteness by deferring side-effects through monads, preserving referential transparency until interaction with the external world \citep{wadler1992, turner1979}. This section explores this parallel and extends it to thermodynamic systems, emphasizing the export of entropy as a universal strategy for maintaining order.

\subsection{Monads in Functional Programming}
In imperative programming, side-effects (e.g., I/O, state changes) contaminate composition, requiring external coordination (e.g., runtime systems). Pure functional languages encapsulate side-effects in monads, such as the IO monad, which describes computations without executing them until runtime \citep{wadler1992}. For example, a function \(f : X \to \text{IO } Y\) delays printing until the monad is evaluated, ensuring composition remains pure.

The monad's bind operator (\(\gg=\)) sequences effects, analogous to NCL's wavefront propagation, ensuring deterministic behavior without external timing \citep{turner1979}. This internalizes the coordination that synchronous systems externalize to a runtime, mirroring NCL's shift from external clocks to null states \citep{fant2005}.

Functional programming thus defers incompleteness: monads allow designers to work with pure functions while the runtime handles ``dirty'' effects, similar to G\"{o}del's meta-level extensions \citep{wadler1992}.

\subsection{Dissipative Structures and Entropy Export}
Ilya Prigogine's dissipative structures provide a thermodynamic analogue \citep{prigogine1984, nicolis1977}. Systems like hurricanes, organisms, or economies maintain internal order by exporting entropy to their environment. The second law of thermodynamics states that total entropy increases (\(\Delta S_{\text{universe}} \geq 0\)), but locally, dissipative structures reduce entropy (\(\Delta S_{\text{system}} < 0\)) by increasing it externally (\(\Delta S_{\text{environment}} > 0\)) \citep{nicolis1977}.

For instance, a hurricane organizes itself by dissipating heat into the atmosphere, increasing global entropy. Biological systems sustain complexity by expelling disorder, as detailed in \citet{schneider2005}. In computation, garbage collection externalizes memory cleanup to a runtime process; in economics, platforms like Uber offload labor and maintenance costs to drivers. These systems defer incompleteness, exporting entropy to a superset, much like G\"{o}delian systems rely on metalevels for truth \citep{tarski1956}.

Prigogine's framework generalizes incompleteness to physics: order is provisional, dependent on environmental sinks, paralleling the Law of Superset Entropy.

\subsection{Law of Superset Entropy}
The parallels suggest a unifying principle, the Law of Superset Entropy: any ordered system maintains coherence by exporting incompleteness or entropy to a superset, or internalizing it through structural reorganization. In functional programming, monads internalize side-effects; in NCL, null states internalize timing; in dissipative structures, internal organization absorbs some entropy, reducing external export \citep{wadler1992, fant2005, prigogine1984}. This law frames incompleteness as a trade-off between external dependence and internal complexity, extending G\"{o}del's logical insights to physical and computational domains.

\section{RSVP and the Cosmological Extension}
\label{sec:rsvp}
The Relativistic Scalar-Vector-Potential (RSVP) model reinterprets cosmic redshift as an entropic relaxation process, ``falling outward,'' rather than metric expansion \citep{whittle2015}. This section details RSVP's fields and dynamics, connecting them to the incompleteness theme, and incorporates the ``falling outward'' analogy with inflationary extension.

\subsection{RSVP Fields and Dynamics}
RSVP posits three interacting fields:

\begin{itemize}
    \item Scalar \(\phi(x,\eta)\): Drives entropic smoothing, analogous to NCL's null state, deferring structural relaxation by encoding spatial incompleteness.
    \item Vector \(\mathbf{v}(x,\eta)\): Represents bulk flows and local deviations, facilitating transport of matter and energy.
    \item Entropy \(S(x,\eta)\): Quantifies disorder, redistributed during structure formation, aligning with Prigogine's entropy export \citep{prigogine1984}.
\end{itemize}

The entropic redshift potential is defined as:
\[
\Upsilon \equiv \delta\phi - \beta(\eta)\varphi_m, \quad \varphi_m(k,\eta) = \frac{4\pi G a^2(\eta) \bar{\rho}_m(\eta)}{k^2} \mathcal{T}_m(k,\eta) \delta_m(k,\eta).
\]
Here, \(\delta\phi\) is the scalar perturbation, \(\varphi_m\) the dimensionless matter potential, and \(\beta(\eta)\) a time-dependent coupling \citep{dodelson2003}. \(\phi\) mediates void expansion, explaining acceleration without invoking exotic vacuum energy \citep{einstein1917, desitter1917, peebles2003}. The effective potential governing outward/inward acceleration is:
\[
\mathbf{a}_{\rm eff} = -\nabla \Phi_{\rm eff}, \quad \Phi_{\rm eff} \equiv \phi - \gamma(\eta) \varphi_m.
\]
This balances scalar capacity (outward fall) and matter density (inward pull), with \(\gamma(\eta)\) a coupling function.

\subsection{Falling Outward}
Mark Whittle's ``falling outward'' analogy describes vacuum energy driving expansion by reducing gravitational potential \citep{whittle2015, whittle2006}. In a matter-dominated sphere, the potential energy \(U = -\frac{G M m}{r}\) becomes less negative as radius \(r\) increases, favoring collapse. In a vacuum-dominated sphere, mass scales as \(M \propto r^3\), making \(U \propto -r^2\), so expansion is energetically favorable. RSVP reframes this: \(\phi\) drives voids to ``smooth outward,'' generating entropy to fill expanded volumes, a self-sustaining process akin to Prigogine's dissipative structures \citep{prigogine1984, whittle2015}.

Consider a spherical region in the RSVP plenum with a test particle at its boundary. For matter-rich regions, the effective energy is \(E \sim -\frac{G M m}{r}\), driving inward collapse. For void-like regions dominated by \(\phi\), \(M(r) \propto r^3\), yielding \(E \sim -r^2\), driving outward relaxation. The entropy generated by the fall sustains the new capacity, mirroring Whittle's self-generating expansion \citep{whittle2015}.

\subsection{Inflationary Extension}
In the early universe, a high-density \(\phi\) triggers a rapid ``lamphron-lamphrodyne flash,'' an explosive entropic smoothing akin to inflation \citep{whittle2015}. The effective energy scales steeply, leading to exponential outward relaxation:
\[
\frac{d^2 r}{dt^2} \propto \rho_\phi r.
\]
This resolves horizon and flatness issues by smoothing entropic gradients, establishing causal uniformity. Post-flash, residual potentials drive observed void expansion, linking early smoothing to late-time acceleration \citep{desitter1917}.

\subsection{Historical and Observational Context}
Einstein introduced the cosmological constant \(\Lambda\) for a static universe \citep{einstein1917}, but de Sitter showed it yields exponential expansion without matter \citep{desitter1917}. Observations by \citet{riess1998} and \citet{perlmutter1999} confirmed acceleration, interpreted as dark energy \citep{peebles2003}. RSVP aligns with these but frames acceleration as entropic, consistent with \citet{dodelson2003}'s modern cosmology.

This extension to cosmology shows incompleteness at universal scales: the plenum's \(\phi\) field encodes entropic boundaries, deferring relaxation until conditions align, much like NCL's null states.

\section{Dipole Constraints and the Absence of Multiverses}
\label{sec:dipole}
RSVP imposes stringent constraints on beyond-horizon inhomogeneities through CMB dipole observations (Appendix A). A significant density or entropic gradient would produce a residual dipole misaligned with local gravitational flows (e.g., Great Attractor) \citep{riess1998, perlmutter1999}. The observed dipole (\(\varepsilon_{\rm kin} \sim 10^{-3}\)) aligns with reconstructions, and the intrinsic dipole is bounded by:
\[
\Delta \Upsilon_* \equiv \|\nabla \Upsilon_*\| R_* \lesssim \text{few} \times 10^{-5}.
\]
This implies isotropy and homogeneity extend at least one horizon length beyond the observable, disfavoring bubble universes with varying FRW parameters \citep{dodelson2003}. The universe is constrained by inherent conditions, not sampling arbitrary parameters, resonating with predictive processing models in cognition that minimize environmental uncertainty \citep{clark2013, hohwy2013}.

The alignment test reinforces this: RSVP bulk-flow estimators converge to the CMB dipole, indicating no super-horizon tilt. Long-mode consistency further tightens bounds, tying residuals to the small quadrupole \citep{whittle2006}.

This section expands the cosmological implications, showing how RSVP's entropic framework naturally rejects multiverse variability, aligning with the essay's theme of constrained incompleteness.

\section{Synthesis}
\label{sec:synthesis}
Incompleteness is an architectural necessity across domains. G\"{o}del's theorems reveal logical limits; NCL's null signals enable self-coordinating circuits; GOFAI critiques advocate adaptive systems; monads defer side-effects; dissipative structures export entropy; and RSVP's \(\phi\) drives cosmic relaxation. The Law of Superset Entropy unifies these: order requires exporting incompleteness to a superset or internalizing it through reorganization.

In Matthew 26:29, Jesus' refusal to drink wine until the kingdom signifies a null state, fulfilled by future action, opening a larger incompleteness. The cosmos, as a plenum of null signals, falls outward within a greater whole, never fully complete yet coherent.

\section*{Appendix: CMB Dipole Constraints in RSVP (Entropic Redshift Form)}
\subsection*{A.1 Fields, potential, and normalization}
We model the plenum with scalar capacity \(\Phi\), vector flow \(\mathbf v\), and matter density \(\rho_m\).
Perturbations define an \emph{entropic redshift potential} \(\Upsilon\) that controls photon redshift analogously to the Newtonian potential \(\Phi_{\rm N}\) in \(\Lambda\)CDM.
\paragraph{Convention (normalization).}
We normalize \(\Upsilon\) so that the instantaneous Sachs–Wolfe (SW) contribution at last scattering is
\begin{equation}
\left(\frac{\Delta T}{T}\right)_{\!\rm SW} \;=\; \tfrac{1}{3}\,\Upsilon_\*.
\label{eq:SWnorm}
\end{equation}
To make contact with \(\Lambda\)CDM when desired, one can adopt the dictionary
\begin{equation}
\Upsilon \;\equiv\; \delta\Phi \;-\; \alpha_m(\eta,k)\,\delta\rho_m,
\qquad
\alpha_m(\eta,k)\;=\;\frac{4\pi G\,a^2(\eta)\,\bar\rho_m(\eta)}{k^2}\,\mathcal T_m(\eta,k),
\label{eq:dictionary}
\end{equation}
where \(\mathcal T_m\) is the RSVP transfer from \(\delta\rho_m\) into the potential sector. Setting \(\alpha_\Phi\equiv 1\) recovers \(\Upsilon\to\Phi_{\rm N}\) in the \(\Lambda\)CDM limit.
\subsection*{A.2 Large-angle anisotropy decomposition}
For line-of-sight \(\hat{\mathbf n}\),
\begin{equation}
\frac{\Delta T}{T}(\hat{\mathbf n})
\;=\;
\underbrace{\hat{\mathbf n}\!\cdot\!\frac{\mathbf v_0}{c}}_{\text{kinematic (local flow)}}\;
+\;
\underbrace{\frac{1}{3}\,\Upsilon_\*(\hat{\mathbf n})}_{\text{entropic SW at last scattering}}
\;+\;
\underbrace{2\!\int_{\eta_\*}^{\eta_0}\!\dot{\Upsilon}\,d\eta}_{\text{entropic ISW}}\!,
\label{eq:anisodecomp}
\end{equation}
where \(\mathbf v_0\) is our local bulk velocity.
\subsection*{A.3 Super-horizon gradient bound (intrinsic dipole)}
Assume a nearly uniform super-horizon gradient across our patch,
\begin{equation}
\Upsilon(\mathbf x,\eta) \;\simeq\; \Upsilon_0(\eta) + \mathbf G(\eta)\!\cdot\!\mathbf x,
\qquad \|\mathbf G\|\,R_\* \ll 1,
\end{equation}
with \(R_\*\) the comoving radius to last scattering. The intrinsic dipole amplitude then obeys
\begin{equation}
D_{\rm int}\;\equiv\;\left|\left(\frac{\Delta T}{T}\right)_{\ell=1}^{\rm int}\right|
\;=\; \frac{1}{3}\,\|\mathbf G_\*\|\,R_\* \;+\; \mathcal O\!\left(\int \dot\Upsilon\,d\eta\right).
\end{equation}
Imposing \(D_{\rm int}\lesssim \varepsilon_{\rm dip}\) with \(\varepsilon_{\rm dip}\sim10^{-5}\) yields the dimensionless gradient bound
\begin{equation}
\boxed{\;\|\nabla\Upsilon_\*\|\,R_\* \;\lesssim\; 3\,\varepsilon_{\rm dip}\;}.
\label{eq:gradbound}
\end{equation}
\subsection*{A.4 Linking to ``falling outward'' (effective potential form)}
RSVP kinematics of outward fall are governed by an effective scalar potential
\begin{equation}
\mathbf a_{\rm out} \;=\; -\,\nabla\Phi_{\rm eff},
\qquad
\Phi_{\rm eff}\;\equiv\;\Phi \;-\; \gamma(\eta,k)\,\varphi_m,
\label{eq:phieff}
\end{equation}
with \(\varphi_m\) a dimensionless matter potential and \(\gamma\) a (model-dependent) coupling. For the linear redshift imprint, we identify \(\Upsilon=\mathcal N(\eta)\,\Phi_{\rm eff}\) for some normalization \(\mathcal N\) compatible with \eqref{eq:SWnorm}. Then the last-scattering bounds can be stated as
\begin{equation}
|\delta\Phi|_\* \;\lesssim\; \frac{\varepsilon_{\rm dip}}{\alpha_\Phi},
\qquad
|\delta\rho_m|_\* \;\lesssim\; \frac{\varepsilon_{\rm dip}}{\alpha_m},
\label{eq:componentbounds}
\end{equation}
with \(\alpha_\Phi\equiv \partial\Upsilon/\partial(\delta\Phi)\) and \(\alpha_m\) as in \eqref{eq:dictionary}.
\subsection*{A.5 Vector alignment test (entropy-weighted convergence)}
Define the RSVP bulk-flow estimator inside radius \(R\) by entropy-weighted redshift fits:
\begin{equation}
\mathbf v_0^{\rm RSVP}(R)
\;:=\;
\arg\min_{\mathbf v}\;
\sum_{i:\, r_i<R}\,w_i\,\Big(z_i^{\rm obs}-z_i^{\rm RSVP}(\mathbf v)\Big)^2,
\qquad
w_i \propto \frac{1}{\sigma_{S,i}},
\end{equation}
where \(\sigma_{S,i}\) are semantic/entropy weights in RSVP. Convergence to the CMB dipole means
\begin{equation}
\angle\!\big(\mathbf v_0^{\rm RSVP}(R),\,\mathbf d_{\rm CMB}\big)\to 0,
\qquad
\big\|\mathbf v_0^{\rm RSVP}(R)\big\|\to c\,D_{\rm kin}
\quad \text{as } R\to \infty,
\end{equation}
with \(D_{\rm kin}\) the measured kinematic dipole amplitude.
\subsection*{A.6 Long-mode consistency (RSVP gauge)}
Super-horizon adiabatic \(\Upsilon\) modes correspond to an RSVP \emph{semantic-slicing} gauge redefinition. The leading dipole cancels as a gauge artifact, and any residual is tied to the quadrupole via the transfer of \(\dot\Upsilon\) at horizon entry. Since the measured quadrupole is small, this tightens \eqref{eq:gradbound}.
\subsection*{A.7 Takeaway (RSVP)}
The entropic redshift potential \(\Upsilon=\Upsilon[\Phi,\rho_m]\) encapsulates how scalar capacity (falling outward) and mass (inward pull) imprint the CMB. The residual dipole forces \(\Upsilon\) to be nearly uniform across the last-scattering sphere,
\begin{equation}
\Delta\Upsilon_\* \;\lesssim\; \text{few}\times 10^{-5},
\end{equation}
so RSVP’s falling-outward cosmology must satisfy near-homogeneity in \(\Upsilon\) at least one horizon scale beyond the visible patch.

\section*{Appendix B: Falling Outward in the RSVP Framework}
Consider a spherical region in the RSVP plenum with a test particle at its boundary, governed by \(\phi\), \(\mathbf{u}\), and \(S\).

\textbf{Case 1: Matter-Dominated Sphere.} For matter-rich regions, the effective energy is:
\[
E \sim -\frac{G M m}{r},
\]
driving inward collapse.

\textbf{Case 2: Entropic Vacuum-Dominated Sphere.} In void-like regions dominated by \(\phi\), the ``mass'' scales as \(M(r) \propto r^3\), yielding:
\[
E \sim -r^2
\]
This drives outward relaxation, with entropy generated by the fall itself.

\textbf{Inflationary Extension.} In the early plenum, a high-entropy \(\phi\) dominates, producing a lamphron-lamphrodyne flash:
\[
E \sim - \rho_{\phi} r^2, \quad \frac{d^2 r}{dt^2} \propto \rho_{\phi} r
\]
This rapid smoothing establishes causal uniformity, prefiguring void expansion. The CMB dipole constraint (\(\Delta \Upsilon_* \lesssim \text{few} \times 10^{-5}\)) ensures this coherence persists, ruling out arbitrary parameter variations.

\newpage
\bibliographystyle{plainnat}
\bibliography{references}
\end{document}
