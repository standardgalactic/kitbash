\section*{Appendix A: CMB Dipole Constraints in RSVP (Entropic Redshift Form)}

\subsection*{A.1 Fields, Normalization, and $\Lambda$CDM Dictionary}
We model the plenum with scalar capacity \(\phi(x,\eta)\), vector flow \(\mathbf{u}(x,\eta)\), and matter density \(\rho_m(x,\eta)\). The entropic redshift potential is defined as:
\[
\boxed{\Upsilon \equiv \delta\phi - \beta(\eta)\varphi_m}, \quad \varphi_m(k,\eta) = \frac{4\pi G a^2(\eta) \bar{\rho}_m(\eta)}{k^2} \mathcal{T}_m(k,\eta) \delta_m(k,\eta).
\]
We normalize \(\Upsilon\) so that the instantaneous Sachs--Wolfe (SW) contribution at last scattering is:
\[
\boxed{\left(\frac{\Delta T}{T}\right)_{\!\rm SW} = \frac{1}{3} \Upsilon_*}.
\]
To align with \(\Lambda\)CDM, we adopt:
\[
\Upsilon = \delta\phi - \alpha_m \delta\rho_m, \quad \alpha_m = \frac{4\pi G a^2(\eta) \bar{\rho}_m(\eta)}{k^2} \mathcal{T}_m(k,\eta), \quad \alpha_\phi = 1.
\]

\subsection*{A.2 Large-Angle Anisotropy Decomposition}
For line-of-sight \(\hat{\mathbf{n}}\):
\[
\frac{\Delta T}{T}(\hat{\mathbf{n}}) = \underbrace{\hat{\mathbf{n}} \cdot \frac{\mathbf{u}_0}{c}}_{\text{kinematic dipole } \varepsilon_{\rm kin} \sim 10^{-3}} + \underbrace{\frac{1}{3} \Upsilon_*(\hat{\mathbf{n}})}_{\text{entropic SW}} + \underbrace{2 \int_{\eta_*}^{\eta_0} \dot{\Upsilon} \, d\eta}_{\text{entropic ISW}}.
\]
The intrinsic dipole after kinematic subtraction is:
\[
\left| \left( \frac{\Delta T}{T} \right)_{\ell=1}^{\rm int} \right| \equiv \varepsilon_{\rm int} \lesssim \text{few} \times 10^{-5}.
\]

\subsection*{A.3 Super-Horizon Gradient Bound}
Assume a nearly uniform super-horizon gradient:
\[
\Upsilon(\mathbf{x},\eta) \simeq \Upsilon_0(\eta) + \mathbf{G}(\eta) \cdot \mathbf{x}, \quad \|\mathbf{G}\| R_* \ll 1,
\]
with \(R_*\) the comoving radius to last scattering. The intrinsic dipole amplitude is:
\[
D_{\rm int} \approx \frac{1}{3} \|\mathbf{G}_*\| R_* + \mathcal{O}\left( \int \dot{\Upsilon} \, d\eta \right).
\]
Given \(\varepsilon_{\rm int} \sim 10^{-5}\), the dimensionless gradient bound is:
\[
\boxed{\|\nabla \Upsilon_*\| R_* \lesssim 3 \varepsilon_{\rm int}} \quad \Longleftrightarrow \quad \|\mathbf{G}_*\| \lesssim \frac{3 \varepsilon_{\rm int}}{R_*}.
\]

\subsection*{A.4 Linking to ``Falling Outward'' (Effective Potential Form)}
RSVP kinematics are governed by:
\[
\boxed{\mathbf{a}_{\rm eff} = -\nabla \Phi_{\rm eff}, \quad \Phi_{\rm eff} \equiv \phi - \gamma(\eta) \varphi_m}.
\]
The redshift imprint is \(\Upsilon = \mathcal{N}(\eta) \Phi_{\rm eff}\), with \(\mathcal{N}(\eta_*)\) set by the SW normalization. The dipole bound implies:
\[
|\delta\phi|_* \lesssim \frac{\varepsilon_{\rm int}}{\alpha_\phi} = \varepsilon_{\rm int}, \quad |\delta\rho_m|_* \lesssim \frac{\varepsilon_{\rm int}}{\alpha_m}.
\]

\subsection*{A.5 Vector Alignment Test (Entropy-Weighted Convergence)}
Define the RSVP bulk-flow estimator:
\[
\mathbf{u}_0^{\rm RSVP}(R) := \arg\min_{\mathbf{u}} \sum_{i: r_i<R} w_i \left( z_i^{\rm obs} - z_i^{\rm RSVP}(\mathbf{u}) \right)^2, \quad w_i \propto \frac{1}{\sigma_{S,i}}.
\]
Convergence to the CMB dipole means:
\[
\angle\left(\mathbf{u}_0^{\rm RSVP}(R), \mathbf{d}_{\rm CMB}\right) \to 0, \quad \|\mathbf{u}_0^{\rm RSVP}(R)\| \to c \varepsilon_{\rm kin}.
\]

\subsection*{A.6 Long-Mode Consistency (RSVP Gauge)}
Super-horizon adiabatic \(\Upsilon\) modes correspond to a semantic-slicing gauge redefinition in RSVP. The leading dipole cancels, with residuals tied to the quadrupole via \(\dot{\Upsilon}\) at horizon entry. The small quadrupole (\(\sim 10^{-5}\)) tightens the A.3 bound.

\subsection*{A.7 Takeaway}
The entropic redshift potential \(\Upsilon = \Upsilon[\phi, \rho_m]\) encapsulates scalar capacity (falling outward) and mass (inward pull). The residual dipole limit enforces:
\[
\Delta \Upsilon_* \equiv \|\nabla \Upsilon_*\| R_* \lesssim \text{few} \times 10^{-5},
\]
supporting homogeneity beyond the observable, disfavoring bubble universes with varying parameters.

\section*{Appendix B: Falling Outward in the RSVP Framework}
Consider a spherical region in the RSVP plenum with a test particle at its boundary, governed by \(\phi\), \(\mathbf{u}\), and \(S\).

\paragraph{Case 1: Matter-Dominated Sphere.} The effective energy is:
\[
E \sim -\frac{G M m}{r},
\]
driving inward collapse.

\paragraph{Case 2: Entropic Vacuum-Dominated Sphere.} For void-like regions, mass scales as \(M(r) \propto r^3\), yielding:
\[
E \sim -r^2,
\]
driving outward relaxation.

\paragraph{Inflationary Extension.} In the early plenum, high-entropy \(\phi\) triggers a lamphron-lamphrodyne flash:
\[
E \sim -\rho_{\phi} r^2, \quad \frac{d^2 r}{dt^2} \propto \rho_{\phi} r,
\]
establishing causal uniformity. The CMB dipole constraint (\(\Delta \Upsilon_* \lesssim \text{few} \times 10^{-5}\)) ensures coherence, ruling out parameter variations.