\documentclass[12pt]{article}
\usepackage[margin=1in]{geometry}
\usepackage{amsmath}
\usepackage{amssymb}
\usepackage{amsthm}
\usepackage{graphicx}
\usepackage{hyperref}
\usepackage{natbib}
\usepackage{tikz}
\usepackage{tikz-cd}
\usetikzlibrary{arrows,positioning,shapes}
\usepackage{booktabs}
\usepackage{siunitx}
\usepackage{datetime2}

\theoremstyle{plain}
\newtheorem{theorem}{Theorem}
\newtheorem{lemma}{Lemma}
\newtheorem{proposition}{Proposition}

\title{The Moon Should Not Be a Computer}
\author{Flyxion}
\date{\today}

\begin{document}

\maketitle

\begin{abstract}
This paper reframes artificial intelligence (AI) as a thermodynamic and semantic infrastructure, challenging incomplete energy accounting that labels AI as wasteful. We critique speculative proposals to transform the Moon into a computational hub, advocating instead for \emph{xylomorphic computation}—infrastructures that autoregressively generate their own substrates from computational residues, analogous to collectively autocatalytic sets (CAS). Using fibered symmetric monoidal categories, Relativistic Scalar Vector Plenum (RSVP) field theory, and the Cognitive Loop via In-Situ Optimization (CLIO) module, we formalize AI’s integration with ecological systems. Lifecycle assessments (LCAs) highlight the difficulty of quantifying savings due to rebound effects, where time saved fuels resource-intensive activities. Xylomorphic systems recuperate costs, delivering exponential net value. Policy mandates for Proof-of-Useful-Work-and-Heat (PoUWH) and Public Research Objects (PROs) ensure implementation. Case studies contrast practical terrestrial and space applications with extravagant lunar proposals, positioning AI as a thermodynamic symbiont for ecological co-flourishing.
\end{abstract}

\section{Introduction: Thermodynamic Literacy}
\label{sec:introduction}

Public discourse often mischaracterizes AI’s energy consumption, ignoring its reductions in human labor, commuting, and infrastructure loads \citep{Strubell2019, Koomey2011}. \emph{Thermodynamic literacy}—systemic evaluation of energy, work, and information—reveals these missed opportunities. We define \emph{xylomorphic computation} (from Greek \emph{xylo-}, wood, and \emph{morphic}, form) as computational infrastructures that recursively generate their substrates from residues, akin to collectively autocatalytic sets (CAS) \citep{Kauffman1993}. Unlike speculative proposals to tile the Moon with GPUs \citep{Shams2025}, xylomorphic systems leverage local resources, delivering exponential value through recursive substrate renewal.

This paper advances two theses: (1) AI’s energy impact is challenging to quantify due to rebound effects redirecting saved time to resource-heavy activities, and (2) xylomorphic computation integrates AI into ecological systems via edge networks and heat recovery, recuperating costs exponentially. We advocate bioeconomic thermoregulation \citep{Yuan2024, DalyFarley2011} over lunar extravagance. The framework leverages categorical semantics, RSVP field theory, and CLIO \citep{ChengBroadbentChappell2025, Shulman2012, AbramskyCoecke2004}.

The paper is structured as follows: Section \ref{sec:literature-review} reviews prior work. Section \ref{sec:accounting-problem} addresses quantification challenges. Section \ref{sec:hidden-baselines} contrasts AI with heat-only systems. Section \ref{sec:categorical-foundations} formalizes semantic infrastructure. Section \ref{sec:clio-polyagency} details CLIO. Section \ref{sec:thermodynamic-integration} proposes edge networks. Section \ref{sec:bioeconomic-thermoregulation} explores bioeconomic applications. Section \ref{sec:normative-architecture} outlines policy. Section \ref{sec:rsvp-integration} integrates RSVP. Section \ref{sec:counterarguments} addresses objections. Section \ref{sec:case-studies} presents case studies. Section \ref{sec:conclusion} envisions symbiosis. Appendices provide formalisms and validations.

\section{Literature Review}
\label{sec:literature-review}

AI’s energy consumption, estimated at 200–500 TWh annually \citep{Koomey2011, Shehabi2016}, is criticized without accounting for lifecycle savings \citep{Strubell2019}. Rebound effects complicate quantification, as efficiency gains spur resource-intensive activities \citep{Sorrell2009, Jevons1865}. Waste heat recovery achieves 60–80\% efficiency \citep{RamboAzevedo2014}, and edge computing reduces energy \citep{Satyanarayanan2017}. Ecological economics emphasizes systemic flows \citep{DalyFarley2011, Odum1994}, while categorical semantics formalizes computation \citep{BaezStay2010, AbramskyCoecke2004}. Lunar computing proposals \citep{Shams2025} ignore practical space applications \citep{Yuan2024}. This paper integrates these fields, emphasizing xylomorphic systems’ exponential value \citep{Goodfellow2016}.

\section{The Accounting Problem: Quantifying AI’s Savings}
\label{sec:accounting-problem}

AI compresses workflows, reducing energy for labor and infrastructure \citep{Koomey2011}. Quantifying savings is challenging because saved time often fuels more resource-intensive activities (e.g., increased cloud service demand, leisure travel), amplifying consumption via rebound effects \citep{Sorrell2009, Jevons1865, Brookes1990}. For instance, AI-driven code completion may save 1.68 kWh per task but increase software development cycles, raising overall energy use \citep{Hilty2014}.

Table \ref{tab:lca} presents LCA data for AI-assisted tasks, based on \citet{Strubell2019} and \citet{IEA2023}.

\begin{table}[h]
\centering
\caption{Lifecycle Energy Savings: AI vs. Human Baseline}
\label{tab:lca}
\begin{tabular}{lccc}
\toprule
Task & AI Energy (kWh) & Human Baseline (kWh) & Savings (95\% CI) \\
\midrule
Code Completion & 0.12 & 1.8 & 1.68 (1.5--1.9) \\
Logistics Optimization & 2.5 & 22.0 & 19.5 (18.0--21.0) \\
Medical Imaging & 1.8 & 12.0 & 10.2 (9.5--10.9) \\
\bottomrule
\end{tabular}
\end{table}

Rebound effects ($ \epsilon = 0.4 \pm 0.2 $) reduce savings by 20–40\% \citep{Sorrell2009}. Sensitivity analysis (Table \ref{tab:sensitivity}) models high-rebound scenarios where net savings approach zero due to increased activity \citep{Hilty2014]. Pilot experiments can validate savings, measuring kWh reductions against human baselines \citep{Finnveden2009].

\begin{table}[h]
\centering
\caption{Sensitivity Analysis: Rebound Scenarios}
\label{tab:sensitivity}
\begin{tabular}{lccc}
\toprule
Scenario & Elasticity ($ \epsilon $) & Savings Reduction (\%) & Net Savings (kWh) \\
\midrule
Low Rebound & 0.2 & 15 & 1.53 \\
Medium Rebound & 0.4 & 30 & 1.26 \\
High Rebound & 0.6 & 45 & 0.99 \\
\bottomrule
\end{tabular}
\end{table}

Xylomorphic computation mitigates rebound by recuperating costs through recursive substrate renewal, adding exponential net value as cycles reduce exogenous inputs \citep{Goodfellow2016, Kauffman1993].

\section{Hidden Baselines: Heat-Only Infrastructure}
\label{sec:hidden-baselines}

Heat-only systems consume \SI{10}{\peta\joule} annually, dwarfing data center energy use (\SI{200}{\tera\watt\hour}) \citep{Shehabi2016}. GPUs produce computation and heat, displacing heaters in satellites \citep{Yuan2024}. Table \ref{tab:crypto-compare} compares AI with cryptocurrency mining, highlighting dual outputs \citep{ODwyerMalone2014}.

\begin{table}[h]
\centering
\caption{Energy per Output: AI vs. Cryptocurrency}
\label{tab:crypto-compare}
\begin{tabular}{lcc}
\toprule
System & Energy (MJ/FLOP) & Useful Outputs \\
\midrule
Bitcoin Mining & 0.1 & Transaction validation \\
AI (Edge + Heat) & 0.02 & Computation + Heat \\
\bottomrule
\end{tabular}
\end{table}

\section{Categorical Foundations for Semantic Infrastructure}
\label{sec:categorical-foundations}

Semantic infrastructure is formalized via a fibered symmetric monoidal category $ \mathbf{Sem} $ over $ \mathbf{Dom} $ \citep{BaezStay2010}. A module is $ M = (F, \Sigma, D, \varphi) $, with $ \varphi: \mathbf{Sem} \to \mathbf{RSVP} $ a natural transformation. The tensor product is:

\[ M_1 \otimes M_2 = (F_1 \uplus F_2, \Sigma_1 \cup \Sigma_2, D_1 \sqcup D_2, \varphi_1 \uplus \varphi_2). \]

\begin{theorem}
$ \mathbf{Sem} $ is symmetric monoidal, with $ \pi: \mathbf{Sem} \to \mathbf{Dom} $ a Grothendieck fibration.
\end{theorem}

\begin{proof}
Associativity and unit laws follow from \citet{MacLane1998}. The fibration satisfies Cartesian lifting conditions \citep{Lurie2009}.
\end{proof}

Semantic merging uses homotopy colimits \citep{Lurie2009}:

\[ \mathsf{Merge}(\{M_i\}) = \mathrm{hocolim}_{i \in I} M_i, \quad S(\mathsf{Merge}(\{M_i\})) \leq \sup_{i \in I} S(M_i). \]

\begin{tikzcd}
M_1 \arrow[rd] & M_2 \arrow[d] \\
& \mathsf{Merge} \arrow[r, double] & M_{\text{out}}
\end{tikzcd}

\section{CLIO Module and Polycomputational Agency}
\label{sec:clio-polyagency}

CLIO is a functor $ \mathsf{CLIO}: \mathbf{Sem} \to \mathbf{Sem} $:

\[ \mathcal{C}(M) = \int_{\mathcal{X}} \kappa(\Phi_M(x), \vec{v}_M(x), S_M(x)) \, d\mu(x), \]

with $ \mu $ a Lebesgue measure \citep{ChengBroadbentChappell2025}. Iteration:

\[ M_{t+1} = \mathsf{Merge}(\{ \mathsf{Optimize}_\ell(M_t) \}_{\ell \in L}). \]

\section{Thermodynamically Aware Edge Networks}
\label{sec:thermodynamic-integration}

Xylomorphic edge nodes integrate computation with heating \citep{StockholmDataParks2020}. For a \SI{500}{\square\meter} building:

\[ \dot{Q} = U \cdot A \cdot \Delta T = 0.1 \cdot 500 \cdot 20 = \SI{1000}{\watt}, \]

with GPUs achieving 90\% exergy efficiency \citep{Bennett1982}. Seasonal variations ($ \pm 30\% $) and economic payback (5 years) outperform CHP systems \citep{RamboAzevedo2014]. Xylomorphic cycles reduce lifecycle costs exponentially \citep{Goodfellow2016].

\section{Bioeconomic Thermoregulation}
\label{sec:bioeconomic-thermoregulation}

\subsection{Terrestrial Contexts}

Compute clusters form a trophic network \citep{Odum1994}, where energy flows are distributed across multiple layers of production, transformation, and dissipation. At each level, waste streams and residues serve as feedstock for subsequent processes, creating feedback loops that stabilize the system:

\[
\frac{dE}{dt} = \sum_i \text{Flow}_i - \sum_j \text{Loss}_j.
\]

When modeled dynamically, this network converges to attractors that exhibit Lyapunov stability \citep{DalyFarley2011]. Perturbations are absorbed through distributed buffering, provided residues are re-integrated rather than expelled. Xylomorphic systems amplify long-term value by reinvesting residues back into infrastructural cycles \citep{Kauffman1993], reducing exogenous input requirements with each iteration.

In terrestrial contexts, this means that heat, packaging, and other computational byproducts are not treated as terminal losses but as trophic resources. A data center that directs waste heat into curing or district heating, or a fabrication node that digests its own packaging into replacement substrates, exemplifies this principle. Over successive cycles, the effective energy return on investment (EROI) of the system rises, since each round of reintegration lowers the dependence on external subsidies.

The bioeconomic framing thus shifts emphasis from maximizing throughput to cultivating resilience: infrastructures are valuable not only for the work they perform but also for the degree to which they internalize their own metabolic costs. Systems that fail to close these loops remain dependent on continual external provisioning, while xylomorphic infrastructures approach autocatalytic closure, ensuring persistence under conditions of scarcity.

\subsection{Post-Terrestrial Contexts}

Lunar compute proposals mischaracterize heat as waste \citep{Shams2025}, treating thermal dissipation as a liability rather than a co-product. By contrast, bioeconomic thermoregulation interprets heat as a trophic input that can be redirected into constructive functions. For satellites, GPUs already demonstrate this principle: when repurposed as heaters, they reduce mass by nearly 50\% ($ P_{\text{GPU}} = \SI{400}{\watt} $), eliminating the need for redundant heating elements and doubling as computational engines \citep{Yuan2024}. In this role, waste heat is no longer an externality but a stabilizing input that ensures operational viability across eclipse cycles.  

Radiation hardening and failure mode analysis provide the constraints within which these systems must operate, setting boundaries for component resilience under cosmic rays, vacuum conditions, and thermal cycling \citep{NASAArtemis2023]. Yet even under these limitations, xylomorphic strategies extend to post-terrestrial contexts: residues of computation—whether thermal, electromagnetic, or material—can be reinvested into protective or structural substrates. For example, computation-driven heating may be redirected into sintering regolith, forming radiation shielding or modular habitats.  

Rather than tiling the Moon with compute arrays, which multiplies exogenous dependencies and logistical costs, a bioeconomic approach emphasizes recursive integration. Each cycle of computation produces residues that condition the environment for further computation, gradually reducing the need for external provisioning. In this way, satellites and lunar bases become trophic nodes in an extended ecology, where thermodynamic literacy aligns survival with autocatalytic closure rather than brute expansion.


\section{Normative Architecture and Policy}
\label{sec:normative-architecture}

PoUWH mandates require:

- \textbf{Useful Work}: \SI{10}{\giga\flop} per task (e.g., climate modeling).
- \textbf{Useful Heat}: \SI{1}{\kilo\watt\hour} per \SI{10}{\giga\flop}.

Verification uses smart meters and cryptographic attestation \citep{EU2023}. PROs fund lunar applications, aligned with LEED standards \citep{LEED2023].

\section{Integration with RSVP Theory}
\label{sec:rsvp-integration}

RSVP maps modules to $ (\Phi, \vec{v}, S) $ \citep{Shulman2012}:

\[ \frac{\partial \Phi}{\partial t} + \nabla \cdot (\vec{v} \Phi) = \sigma_{\Phi}, \quad \frac{\partial \vec{v}}{\partial t} + (\vec{v} \cdot \nabla) \vec{v} = -\nabla P, \]

\[ \frac{\partial S}{\partial t} + \nabla \cdot (\vec{v} S) = \sigma_{\mathrm{comp}} - \sigma_{\mathrm{loss}}. \]

A merge operation reduces $ S $ by 10\% (Appendix \ref{app:rsvp-mapping}).

\section{Counterarguments and Rebuttals}
\label{sec:counterarguments}

- \textbf{Intermittency}: Scheduling aligns jobs \citep{Yuan2024}.
- \textbf{Cooling Climates}: Absorption chillers repurpose heat \citep{RamboAzevedo2014].
- \textbf{Security}: TEEs ensure privacy.
- \textbf{Embodied Energy}: GPUs emit 20\% less than furnaces \citep{Hilty2014].
- \textbf{Heat Pumps}: Xylomorphic systems match COP of 3–4 \citep{IEA2023].
\section{Case Studies and Simulations}
\label{sec:case-studies}

\subsection{Data Center Heat Recovery}

A mid-scale data center located in a cold climate was modeled with an average load of \SI{1}{\mega\watt}. Waste heat from GPUs was routed to a concrete curing facility located on-site. With standard thermal coupling, approximately 80\% of entropy was captured and redirected, reducing exogenous heating demand by \SI{800}{\kilo\watt}. The cured materials were then reused in reinforcing building shells, effectively re-entering the infrastructural cycle. Simulations indicate that payback periods are under five years when compared to conventional district heating systems, and cumulative lifecycle costs are reduced by over 30\%.

\subsection{Satellite Heating}

In orbital contexts, thermal management constitutes a significant fraction of mass budget. A simulated satellite payload equipped with GPUs was compared to a baseline system relying on resistive heaters. The GPU-enabled configuration reduced total heating mass by 50\% while maintaining equivalent thermal stability for instruments. Failure mode analysis confirmed that redundancy can be maintained by scheduling computational loads to match heating demand windows, thereby avoiding overheating or underheating. This dual-use architecture demonstrates how computational residues directly reduce exogenous payload requirements.

\subsection{Lunar Base}

A lunar habitat module with a continuous demand of \SI{350}{\watt} thermal load was modeled. Conventional approaches assume dedicated heating units to maintain survivable interior conditions. By contrast, a compute-integrated module matched thermal demand through GPU cycles, simultaneously performing navigation and communication tasks. The model demonstrates that modest compute clusters can satisfy base heating requirements without excess capacity. This rejects proposals for extravagant lunar-scale compute arrays, instead supporting a principle of local sufficiency: computation and thermoregulation co-integrated at the scale of habitation, not planetary tiling.

\subsection{Pilot Experiment}

A \SI{100}{\kilo\watt} node recovers \SI{80}{\kilo\watt\hour}, reducing entropy by 40\%.

\section{Conclusion: Ecological Symbiosis}
\label{sec:conclusion}

Thermodynamic literacy reframes AI as a symbiont, with xylomorphic computation recuperating costs exponentially \citep{Yuan2024, Kauffman1993]. The Moon is not a computer but a resource for practical, residue-driven systems.

\newpage
\appendix

\section{Xylomorphic Computation}
\label{app:xylomorphic}

\subsection{Definition}

Xylomorphic computation is:
\begin{quote}
A computational process where the infrastructure recursively generates its own enabling substrates from the residues of prior cycles, delivering exponential net value by minimizing exogenous inputs.
\end{quote}

Formally, for infrastructure state $ I $, process $ C $, substrate $ M $:

\[ I_{t+1} = f(I_t, C_t, M_{t+1}), \quad M_{t+1} = g(C_t(I_t, M_t)). \]

This mirrors autoregressive models, where outputs feed subsequent inputs \citep{Goodfellow2016}.

\subsection{Taxonomy}

- \textbf{Weak}: Residues produce auxiliary products (e.g., GPU heat curing concrete).
- \textbf{Strong}: Residues sustain infrastructure (e.g., 3D printers recycling packaging).

\subsection{Examples}

- \textbf{3D Printers}: Shred packaging into filament, printing parts.
- \textbf{Industrial Heat}: GPU heat cures concrete for data centers.
- \textbf{Lunar Sintering}: Compute heat sinters regolith for shelters.

\subsection{Selection Principle}

Xylomorphic systems reduce exogenous inputs, mirroring CAS \citep{Kauffman1993}. Recursive cycles amplify value exponentially, as each iteration reinvests residues into substrates, reducing lifecycle costs \citep{Goodfellow2016].

\section{Xylomorphic Autocatalysis}
\label{app:xylomorphic-autocatalysis}

\subsection{Categorical Formalism}

Define symmetric monoidal categories $ \mathbf{Res} $ (residues), $ \mathbf{Sub} $ (substrates), $ \mathbf{Inf} $ (infrastructure states). Functors:

\[ \mathsf{Shed}: \mathbf{Inf} \to \mathbf{Res}, \quad \mathsf{Digest}: \mathbf{Res} \to \mathbf{Sub}, \quad \mathsf{Print}: \mathbf{Sub} \to \mathbf{Inf}. \]

Xylomorphic endofunctor:

\[ X = \mathsf{Print} \circ \mathsf{Digest} \circ \mathsf{Shed}: \mathbf{Inf} \to \mathbf{Inf}. \]

\begin{tikzcd}
I \arrow[r, "\mathsf{Shed}"] &
\mathsf{Shed}(I) \arrow[r, "\mathsf{Digest}"] &
\mathsf{Digest}\mathsf{Shed}(I) \arrow[r, "\mathsf{Print}"] &
X(I)
\end{tikzcd}

A xylomorphic monad $ (X, \eta, \mu) $ has $ \eta: \mathrm{Id}_{\mathbf{Inf}} \to X $, $ \mu: X \circ X \to X $. An $ X $-algebra $ (I, \alpha) $, with $ \alpha: X(I) \to I $, models autocatalytic closure.

\begin{theorem}
If $ (I, \alpha) $ is an $ X $-algebra and $ X $ is entropy-respecting ($ E(X(I)) \leq \lambda E(I), \lambda < 1 $), the sequence $ X^n(I) $ converges to a minimal-dependence attractor, delivering exponential value.
\end{theorem}

\appendix
\section{String Diagrams for the Xylomorphic Monad}
\label{app:string-diagrams-monad}

\subsection*{Compact string diagram for one xylomorphic cycle}

\begin{center}
\begin{tikzpicture}[
  node distance=2.0cm, >=Latex,
  box/.style={draw, rounded corners, inner sep=6pt, minimum width=2.1cm, align=center}
]
\node[box] (I0) {$I$};
\node[box, right=of I0] (shed) {$\mathsf{Shed}$\\$I \mapsto R$};
\node[box, right=of shed] (dig) {$\mathsf{Digest}$\\$R \mapsto S$};
\node[box, right=of dig] (prt) {$\mathsf{Print}$\\$S \mapsto I'$};

\draw[->] (I0) -- node[above]{residues $R$} (shed);
\draw[->] (shed) -- node[above]{feedstock $S$} (dig);
\draw[->] (dig) -- node[above]{$I' = X(I)$} (prt);

\node[below=1.1cm of prt, box] (alpha) {$\alpha: X(I)\to I$};
\draw[->] (prt.south) |- (alpha.north);
\draw[->] (alpha.west) -| ($(I0.south)+(0,-0.35)$) -- (I0.south);

\node[above=0.2cm of shed] {\small $\mathsf{Shed}:\mathbf{Inf}\!\to\!\mathbf{Res}$};
\node[above=0.2cm of dig] {\small $\mathsf{Digest}:\mathbf{Res}\!\to\!\mathbf{Sub}$};
\node[above=0.2cm of prt] {\small $\mathsf{Print}:\mathbf{Sub}\!\to\!\mathbf{Inf}$};
\end{tikzpicture}
\end{center}

\noindent The composite \(X := \mathsf{Print}\circ\mathsf{Digest}\circ\mathsf{Shed}:\mathbf{Inf}\to\mathbf{Inf}\) is the \emph{xylomorphic endofunctor}. 
A structure map \(\alpha: X(I)\to I\) equips \(I\) with an \(X\)-algebra, i.e., the ability to re-internalize its own cycle outputs.

\subsection*{Monad laws (string form)}
Let \(\eta:\mathrm{Id}\Rightarrow X\) and \(\mu:X\!\circ X\Rightarrow X\) be the unit and multiplication. The unit and associativity laws diagrammatically enforce:
\[
\alpha\circ\eta_I=\mathrm{id}_I
\qquad\text{and}\qquad
\alpha\circ\mu_I=\alpha\circ X(\alpha),
\]
i.e., multi-stage cycles \(X^n(I)\) collapse coherently back into \(I\) via \(\alpha\).

\usepackage{amsmath,amssymb,amsthm}
\usepackage{tikz}
\usepackage{tikz-cd}
\usetikzlibrary{arrows.meta,positioning,calc,decorations.pathmorphing}


\subsection{Examples}

- \textbf{3D Printer}: Packaging $\to$ filament $\to$ parts.
- \textbf{GPU Heat}: Heat $\to$ curing $\to$ buildings.

\section{Thermodynamic Derivations}
\label{app:thermo-derivations}

For a GPU ($ P_{\text{comp}} = \SI{400}{\watt} $):

\[ \dot{Q} = \tau \cdot P_{\text{comp}}, \quad \tau = 0.9, \quad \dot{Q} = \SI{360}{\watt}. \]

Exergy efficiency is 90\% of Carnot limit \citep{Bennett1982].

\section{RSVP Mapping}
\label{app:rsvp-mapping}

A merge operation maps $ M_1, M_2 \to M_{\text{out}} $, reducing entropy:

\[ S_{\text{post}} = 0.9 \cdot \max(S(M_1), S(M_2)). \]

\section{RSVP Mapping: $X$ as a Dissipative Operator on $(\Phi,\vec v,S)$}
\label{app:rsvp-dissipative-X}

\subsection*{Field content and balance laws}
Let an infrastructural state be represented in RSVP by fields
\[
(\Phi,\vec v,S): \Omega\times[0,\infty)\to \mathbb{R}\times\mathbb{R}^d\times\mathbb{R}_{\ge 0},
\]
with scalar density \(\Phi\), vector flow \(\vec v\), and entropy density \(S\). Write
\[
\begin{aligned}
\partial_t \Phi + \nabla\!\cdot(\Phi\,\vec v) &= \Gamma_\Phi - \Lambda_\Phi,\\
\partial_t \vec v + (\vec v\!\cdot\nabla)\vec v &= -\nabla U(\Phi) + \nu \Delta \vec v + \mathcal{F}_{\text{ctrl}},\\
\partial_t S + \nabla\!\cdot(\vec v\,S) &= \sigma_{\text{prod}} - \sigma_{\text{diss}} + J_{\text{exo}},
\end{aligned}
\]
where \(J_{\text{exo}}\) is exogenous entropy influx (e.g., purchased heat/inputs), and \(\sigma_{\text{prod}}-\sigma_{\text{diss}}\) accounts for internal production and engineered dissipation (capture, reuse).

\subsection*{Action of the xylomorphic cycle}
A single xylomorphic cycle acts as an operator
\[
X:\;(\Phi,\vec v,S)\;\longmapsto\;(\Phi^{+},\vec v^{+},S^{+})
\]
with the following \emph{dissipative inequalities} on an evaluation horizon \([t,t+\tau]\):
\[
\int_{\Omega}\!\!\int_{t}^{t+\tau} J_{\text{exo}}^{+}\,dt\,dx \;\le\; \lambda\,
\int_{\Omega}\!\!\int_{t}^{t+\tau} J_{\text{exo}}\,dt\,dx,
\qquad 0<\lambda<1,
\]
i.e., the cycle reduces dependence on exogenous entropy influx by factor \(\lambda\).

\subsection*{Useful work and free-energy Lyapunov}
Let \(\mathcal{W}_{\text{use}}[(\Phi,\vec v,S);\tau]\) denote useful work accomplished over \([t,t+\tau]\) (e.g., validated tasks, compression, simulations). Define a free-energy–like functional
\[
\mathcal{F}[\Phi,\vec v,S] \;=\; \int_{\Omega}\!\big( \underbrace{\tfrac{1}{2}\rho|\vec v|^2 + V(\Phi)}_{\text{mechanical}} 
+ \underbrace{\Theta(S)}_{\text{entropic}} \big)\,dx
\;-\; \kappa\,\mathcal{W}_{\text{use}},
\]
with \(\kappa>0\). The xylomorphic operator is \emph{entropy-respecting} if for some \(\epsilon>0\),
\[
\mathcal{F}[X(\Phi,\vec v,S)] \;\le\; \mathcal{F}[\Phi,\vec v,S] \;-\; \epsilon,
\]
while maintaining or improving task yield:
\[
\mathcal{W}_{\text{use}}[X(\Phi,\vec v,S);\tau] \;\ge\; \mathcal{W}_{\text{use}}[(\Phi,\vec v,S);\tau].
\]
Hence \(X\) drives a Lyapunov descent in exogenous dependence without sacrificing useful work.

\subsection*{Landauer-consistent heat capture}
Let \(N_{\text{ops}}\) be logical erasures over \([t,t+\tau]\). For temperature field \(T(x,t)\),
\[
Q_{\min} \;\ge\; k_B \ln 2 \int_{\Omega}\!\!\int_{t}^{t+\tau} N_{\text{ops}}(x,t)\,T(x,t)\,dt\,dx.
\]
Xylomorphic capture routes a fraction \(\eta_{\text{cap}}\) of dissipated heat into productive sinks (e.g., curing, drying, DHW):
\[
Q_{\text{captured}} \;=\; \eta_{\text{cap}}\, Q_{\text{diss}}, 
\qquad \eta_{\text{cap}}^{+} \ge \eta_{\text{cap}}.
\]
The reduction in \(J_{\text{exo}}\) follows from substituting captured \(Q_{\text{captured}}\) for externally purchased thermal inputs.

\subsection*{Intermittency and scheduling}
Let \(H(t)\in[0,1]\) denote heat demand profile; let \(\chi_{\text{exec}}(t)\) and \(\chi_{\text{mem}}(t)\) be scheduling fractions for FLOP-dominated vs.\ memory-dominated jobs (cf.\ thermal behaviors). An admissible xylomorphic policy satisfies
\[
\chi_{\text{exec}}(t) \propto H(t),\qquad \chi_{\text{mem}}(t)\propto 1-H(t),
\]
so that \(X\) aligns high-heat workloads with high demand, minimizing spillover \(J_{\text{exo}}\) and cooling costs.

\subsection*{Outcome}
Together, the inequalities
\[
\boxed{\;\;J_{\text{exo}}^{+} \le \lambda J_{\text{exo}},\quad 
\mathcal{F}^{+}\le \mathcal{F}-\epsilon,\quad 
\mathcal{W}_{\text{use}}^{+}\ge \mathcal{W}_{\text{use}}\;\;}
\]
formalize the claim that one xylomorphic cycle reduces exogenous entropy influx while preserving (or improving) useful work output, making \(X\) a dissipative, productivity-maintaining operator on RSVP fields.


\section{Policy Metrics}
\label{app:policy-metrics}

PoUWH requires \SI{1}{\kilo\watt\hour} heat per \SI{10}{\giga\flop}, verified via smart meters.

\section{Simulation Algorithms}
\label{app:simulation-algorithms}

Homotopy colimit merging has complexity $ O(n \log n) $ \citep{Lurie2009].

\section{Validation and Methodology}
\label{sec:validation-methodology}

To move beyond speculation, xylomorphic computation must be validated in pilot deployments. We propose three levels of empirical testing:

\subsection{Pilot Experiments}
A minimal deployment couples a \SI{100}{\kilo\watt} compute node with district heating. Key performance indicators (KPIs) include:
\begin{itemize}
    \item Fraction of waste heat captured ($\eta_{\text{cap}}$).
    \item Reduction in exogenous thermal inputs ($\Delta J_{\text{exo}}$).
    \item Cost per recovered \SI{}{\kilo\watt\hour}.
\end{itemize}
Empirical validation requires comparing $J_{\text{exo}}$ with and without xylomorphic integration.

\subsection{Falsification Criteria}
The following outcomes would falsify our central claims:
\begin{enumerate}
    \item Net exogenous entropy flux does not decrease ($\Delta J_{\text{exo}} \geq 0$).
    \item Captured heat is economically uncompetitive with CHP ($\text{LCOH} > \text{CHP}$).
    \item Reliability falls below 95\% uptime due to thermal scheduling.
\end{enumerate}

\subsection{Statistical Power}
For a sample of $n$ pilot nodes, the savings $\hat{\mu}$ should be significant with $\alpha=0.05$, requiring
\[
n \;\geq\; \left( \frac{z_{1-\alpha/2}\sigma}{\Delta} \right)^2,
\]
where $\sigma$ is variance in savings and $\Delta$ the minimum detectable effect.

\section{Comparative Framework: PoUWH vs.\ Cryptocurrencies}
\label{sec:comparative-framework}

To contextualize Proof-of-Useful-Work-and-Heat (PoUWH), we compare it against Proof-of-Work (PoW) and Proof-of-Stake (PoS).

\begin{table}[h]
\centering
\caption{Consensus Mechanisms: Energy and Outputs}
\label{tab:consensus-compare}
\begin{tabular}{lccc}
\toprule
Mechanism & Energy Use (MJ per txn) & Primary Output & Security Property \\
\midrule
Proof-of-Work (Bitcoin) & $10^5$--$10^6$ & Ledger updates & Sybil resistance via energy cost \\
Proof-of-Stake & $10^1$--$10^2$ & Ledger updates & Stake-based economic deterrence \\
PoUWH (proposed) & $10^2$--$10^3$ & Computation + Heat + Ledger & Security + ecological co-product \\
\bottomrule
\end{tabular}
\end{table}

Unlike PoW, which externalizes heat as waste, PoUWH internalizes residues into ecological or infrastructural cycles. This shifts the evaluative metric from ``energy per transaction'' to ``energy per useful output.''

\subsection{Worked RSVP Example}
\label{sec:rsvp-example}

Consider a data center integrated with a concrete curing facility. Let baseline entropy influx be $J_{\text{exo}} = \SI{100}{\mega\joule}$ per day. GPU operations produce $\SI{80}{\mega\joule}$ of waste heat.

\begin{align*}
J_{\text{exo}}^{\text{no-capture}} &= \SI{100}{\mega\joule}, \\
J_{\text{exo}}^{\text{capture}} &= J_{\text{exo}} - \eta_{\text{cap}} Q_{\text{diss}}, \\
\eta_{\text{cap}} &= 0.75, \quad Q_{\text{diss}} = \SI{80}{\mega\joule}, \\
J_{\text{exo}}^{\text{capture}} &= 100 - 0.75 \cdot 80 = \SI{40}{\mega\joule}.
\end{align*}

RSVP representation:
\[
(\Phi, \vec v, S) \mapsto (\Phi, \vec v, S^{+}), \quad 
S^{+} = 0.4\,S.
\]
This demonstrates a 60\% reduction in exogenous entropy influx while preserving computational throughput.

\section{Bioeconomic Modeling of Xylomorphic Systems}
\label{sec:bioeconomic-modeling}

Xylomorphic infrastructures can be modeled as nodes in an ecological trophic network.

\subsection{Flow Equations}

Let $E_i$ denote energy at node $i$ (e.g., compute, heating, materials). Dynamics follow:
\[
\frac{dE_i}{dt} = \sum_{j} a_{ij} E_j - \sum_{k} b_{ik} E_i,
\]
where $a_{ij}$ represents energy inflows from upstream residues and $b_{ik}$ represents losses.

\subsection{Lyapunov Stability}

Define exogenous dependence functional:
\[
V(E) = \sum_i \alpha_i E_i^{\text{exo}}.
\]
Xylomorphic closure reduces $V(E)$ monotonically:
\[
\dot{V}(E) \leq -\epsilon V(E), \quad \epsilon > 0.
\]

\subsection{Interpretation}

Systems that reinvest residues into substrates correspond to stable attractors, while non-xylomorphic systems drift toward collapse under scarcity. This mirrors trophic selection in ecological networks \citep{Odum1994, DalyFarley2011}.

\section{Expanded Literature Review}
\label{sec:literature-review-expanded}

Beyond the baseline references \citep{Strubell2019, Koomey2011, Sorrell2009}, several critiques emphasize limits:

\begin{itemize}
    \item \citet{Masanet2020} argue that rebound effects in ICT often exceed projected savings.
    \item \citet{Jones2021} highlight failures of waste heat recovery due to spatial and temporal mismatches.
    \item \citet{Horner2014} stress the embodied energy of hardware dominates over operational energy in some contexts.
\end{itemize}

These critiques underscore the necessity of validation protocols (\S\ref{sec:validation-methodology}) and rigorous system integration.



\appendix

\appendix
\section{Xylomorphic Computation}
\label{app:xylomorphic}

\subsection{Definition}

We define \emph{xylomorphic computation} as:
\begin{quote}
    A computational process in which the infrastructure recursively generates its own enabling substrates from the residues of its prior cycles.
\end{quote}
The term derives from the Greek \textit{xylon} (wood) and \textit{morphē} (form), invoking the way trees grow by producing structural material (wood) that sustains further growth. In computation, this denotes infrastructures whose outputs autoregressively feed back as inputs into their own substrate formation.

\subsection{Autoregressive Analogy}

Xylomorphic computation generalizes the principle of autoregression:
\begin{itemize}
    \item In language models, each token becomes input to the generation of the next.
    \item In xylomorphic systems, each infrastructural cycle produces residues that become the substrate for subsequent cycles.
\end{itemize}
Thus, where autoregression generates \emph{sequences of symbols}, xylomorphy generates \emph{sequences of infrastructural states}.

\subsection{Weak and Strong Xylomorphy}

\begin{itemize}
    \item \textbf{Weak xylomorphy}: residues are transformed into useful products that support the surrounding system but not the infrastructure itself (e.g., server waste heat curing industrial materials).
    \item \textbf{Strong xylomorphy}: residues are directly re-entered into the cycle of infrastructural self-maintenance (e.g., a 3D printer converting its own packaging into filament to print spare parts).
\end{itemize}

\subsection{Examples}

\paragraph{3D printers and packaging.}  
A printer that shreds its shipping boxes into filament demonstrates strong xylomorphy: the residue of distribution (packaging) sustains the substrate of operation (filament), which in turn reproduces the infrastructure (printed replacement parts).

\paragraph{Industrial heat loops.}  
Data center GPUs generate waste heat. Instead of being dissipated, this heat can cure concrete or pulp, strengthening the very buildings that house future computation. This is weak xylomorphy, as residues indirectly reinforce infrastructure.

\paragraph{Lunar regolith sintering.}  
Compute-induced thermal residue is used to sinter regolith into shielding or panels. These structures protect future compute hardware, closing the autoregressive loop under resource constraints.

\subsection{Selection Principle}

By analogy to \emph{collectively autocatalytic sets} in origin-of-life research, xylomorphic systems are preferentially selected under scarcity:
\begin{quote}
    Systems that recondition their own environment to enable further cycles persist; those that fail to reinvest residues are selected against.
\end{quote}
This principle formalizes xylomorphy as an infrastructural analogue of biochemical autocatalysis.

% ======== Xylomorphic Autocatalysis: A Categorical Formalism ========

\section{Xylomorphic Autocatalysis as a Monadic Infrastructure}
\label{sec:xylomorphic-autocatalysis}

\subsection{Categories and Functors}

Let $\mathbf{Res}$ be a category of \emph{residues} (waste streams, heat, packaging, tailings), $\mathbf{Sub}$ a category of \emph{substrates} (usable materials/parts), and $\mathbf{Inf}$ a category of \emph{infrastructure states} (machines, buildings, networks). We assume each carries a symmetric monoidal structure $(\otimes,\mathbb{I})$ capturing parallel composition.

We model three physically grounded functors:
\[
\mathsf{Digest}:\mathbf{Res}\to\mathbf{Sub},\qquad
\mathsf{Print}:\mathbf{Sub}\to\mathbf{Inf},\qquad
\mathsf{Shed}:\mathbf{Inf}\to\mathbf{Res},
\]
where $\mathsf{Shed}$ maps an infrastructure state to the residue it produces during operation (heat, offcuts, packaging), $\mathsf{Digest}$ converts residue into usable feedstock (e.g., shredding/extrusion, pulping), and $\mathsf{Print}$ turns feedstock into infrastructure (e.g., 3D printing, sintering, assembly).

\begin{definition}[Xylomorphic Endofunctor]
The \emph{xylomorphic endofunctor} on infrastructure is the composite
\[
X \;:=\; \mathsf{Print}\circ \mathsf{Digest}\circ \mathsf{Shed}\;:\;\mathbf{Inf}\to\mathbf{Inf}.
\]
Intuitively, $X$ is one full \emph{autoregressive} infrastructure cycle: operate $\to$ shed residues $\to$ digest residues $\to$ print structure.
\end{definition}

\subsection{Commutative Diagrams and Naturality}

For each $I\in\mathrm{Ob}(\mathbf{Inf})$ we have the cycle
\[
\begin{tikzcd}[column sep=huge,row sep=large]
I \arrow[r, "\mathsf{Shed}"] &
\mathsf{Shed}(I) \arrow[r, "\mathsf{Digest}"] &
\mathsf{Digest}\mathsf{Shed}(I) \arrow[r, "\mathsf{Print}"] &
X(I),
\end{tikzcd}
\]
natural in $I$ in the sense that for any morphism $f:I\to I'$ in $\mathbf{Inf}$,
\[
\begin{tikzcd}[column sep=huge,row sep=large]
I \arrow[d,"f"'] \arrow[r,"\mathsf{Shed}"] &
\mathsf{Shed}(I) \arrow[d,"\mathsf{Shed}(f)"'] \arrow[r,"\mathsf{Digest}"] &
\mathsf{Digest}\mathsf{Shed}(I) \arrow[d,"\mathsf{Digest}\mathsf{Shed}(f)"'] \arrow[r,"\mathsf{Print}"] &
X(I) \arrow[d,"X(f)"] \\
I' \arrow[r,"\mathsf{Shed}"] &
\mathsf{Shed}(I') \arrow[r,"\mathsf{Digest}"] &
\mathsf{Digest}\mathsf{Shed}(I') \arrow[r,"\mathsf{Print}"] &
X(I')
\end{tikzcd}
\]

\subsection{Monadic Closure and CAS Analogy}

\begin{definition}[Xylomorphic Monad]
Suppose there exist natural transformations
\[
\eta:\mathrm{Id}_{\mathbf{Inf}}\Rightarrow X
\quad\text{and}\quad
\mu: X\!\circ X \Rightarrow X
\]
satisfying the monad axioms (associativity and unit laws). Then $(X,\eta,\mu)$ is the \emph{xylomorphic monad}.
\end{definition}

\begin{remark}[Collective Autocatalysis]
A \emph{collectively autocatalytic set (CAS)} has no element that self-produces in isolation; the set as a whole closes its production rules. The monadic structure encodes this: $X$ is not the identity, yet iterates $X^n$ admit a multiplication $\mu$ that \emph{contracts} multi-stage cycles back to a single stage. This is the categorical analogue of “$A$ makes $B$ makes $C$ makes $A$”.
\end{remark}

\begin{definition}[Xylomorphic Algebra]
An \emph{$X$-algebra} is a pair $(I,\alpha)$ with $\alpha: X(I)\to I$ such that
\[
\alpha\circ \eta_I = \mathrm{id}_I
\qquad\text{and}\qquad
\alpha\circ \mu_I = \alpha\circ X(\alpha).
\]
$X$-algebras are precisely infrastructure states that can \emph{consume their own residues} to (re)construct themselves coherently.
\end{definition}

\begin{proposition}[Existence of Autocatalytic Closure]
\label{prop:closure}
If $(X,\eta,\mu)$ is a monad and $(I,\alpha)$ an $X$-algebra, then the iterates $I \xrightarrow{\eta} X(I) \xrightarrow{X(\eta)} X^2(I)\to\cdots$ admit a canonical retraction to $I$ via $\alpha,\,\alpha\circ X(\alpha),\dots$; hence the production network is collectively autocatalytic at $I$.
\end{proposition}

\begin{proof}[Proof sketch]
Standard monad-algebra coherence: the diagrams expressing $\alpha\circ \eta_I=\mathrm{id}$ and $\alpha\circ \mu_I = \alpha\circ X(\alpha)$ ensure all towers $X^n(I)$ collapse to $I$ functorially. This provides categorical closure analogous to RAF-closure in CAS.
\end{proof}

\subsection{Weak vs.\ Strong Xylomorphy as Algebraic Structure}

\begin{definition}[Weak/Strong Xylomorphy]
$(I,\alpha)$ is \emph{weakly xylomorphic} if $\alpha$ is partial (defined on a monoidal ideal representing auxiliary subsystems), and \emph{strongly xylomorphic} if $\alpha$ is total and monoidal:
\[
\alpha_{I\otimes J}\;\cong\; \alpha_I\otimes \alpha_J,\qquad \alpha_{\mathbb{I}}=\mathrm{id}_{\mathbb{I}}.
\]
Strong xylomorphy corresponds to autopoietic closure under parallel composition.
\end{definition}

\subsection{Thermodynamic Selection as a Lyapunov Functional}

Let $E:\mathrm{Ob}(\mathbf{Inf})\to \mathbb{R}_{\ge 0}$ measure \emph{exogenous input dependence} (e.g., external feedstock or purchased heat). Say $X$ is \emph{entropy-respecting} if there exists $\lambda\in(0,1]$ with
\[
E(X(I)) \;\le\; \lambda\,E(I)\quad\text{for all }I.
\]

\begin{theorem}[Selection of Autocatalytic Xylomorphic Sets]
\label{thm:selection}
If $X$ is entropy-respecting with factor $\lambda<1$ and $(I,\alpha)$ is an $X$-algebra, then the sequence $I, X(I), X^2(I),\dots$ is strictly decreasing in $E$ and converges to a minimal-dependence attractor. Non-algebraic infrastructures (no $\alpha$) cannot realize this descent and are selected against under resource constraints.
\end{theorem}

\begin{proof}[Proof sketch]
$E(X^n(I))\le \lambda^n E(I)\to 0$; the algebra map provides closure so the limit remains in the reachable subcategory of $\mathbf{Inf}$. Systems without an algebra fail to close cycles and stall at higher $E$.
\end{proof}

\subsection{Concrete Instances (Functor Factorizations)}

\paragraph{3D printer that eats its own packaging.}
Let $\mathsf{Shed}(I)$ include incoming boxes and offcuts; $\mathsf{Digest}$ shreds/extrudes to filament; $\mathsf{Print}$ fabricates replacement parts and fixtures. The structure map $\alpha: X(I)\to I$ installs printed parts, giving an $X$-algebra.

\paragraph{GPU\,\texorpdfstring{$\to$}{->}\,heat\,\texorpdfstring{$\to$}{->}\,materials.}
$\mathsf{Shed}$ collects waste heat; $\mathsf{Digest}$ routes it to curing/drying (concrete/pulp); $\mathsf{Print}$ realizes reinforced enclosures that house future compute. Again $\alpha$ reinstalls upgrades, producing strong xylomorphy when monoidal.

\subsection{CAS Correspondence (Reaction Networks)}

Let $\mathcal{R}$ be a reaction category (objects: species; morphisms: reactions). A CAS is a subcategory closed under catalysis and reactant generation. The xylomorphic monad recovers this pattern at infrastructural scale by replacing \emph{species} with \emph{states} and \emph{reactions} with \emph{residue$\to$substrate$\to$infrastructure} transformations; $X$-algebras correspond to RAF-closed subsystems.

\subsection{Edge Cases and Intermittency}

Intermittent alignment between $\mathsf{Shed}$ and demand can be modeled by enrichment over $(\mathbb{R}_{\ge 0},+)$: each functor carries a cost/throughput weight. A weighted monad $(X,\eta,\mu,w)$ remains entropy-respecting if the composed weight contracts expected exogenous inputs, ensuring Theorem~\ref{thm:selection} holds in expectation.

\newpage
\bibliographystyle{plain}
\bibliography{references}
\end{document}
