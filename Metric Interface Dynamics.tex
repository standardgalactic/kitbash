\documentclass[11pt]{article}

\usepackage{amsmath,amssymb}
\usepackage{geometry}
\geometry{margin=1in}

\begin{document}

\title{From Irreversible Constraint to Metric Interface Dynamics}
\author{Flyxion}
\date{\today}
\maketitle

\begin{abstract}
This work develops a constraint-first formulation of fundamental physics in which the primary objects are globally admissible histories rather than instantaneous states evolving in time. Within this framework, irreversibility is taken as primitive, and familiar structures such as locality, causality, and analytic time evolution are understood as emergent properties of coarse-grained interface descriptions. We show that spacetime geometry arises as the unique covariant interface capable of locally compressing admissibility information, with General Relativity appearing as the leading-order approximation. Enforcing ultraviolet admissibility while preserving locality and minimal field content uniquely forces the inclusion of curvature-squared terms, yielding Quadratic Gravity as an inevitable interface theory rather than a speculative extension. Apparent ghost pathologies are reinterpreted as reversed-causal regulator modes required by interface compression, not as physical degrees of freedom.
\end{abstract}

\section{Introduction: The Failure of State-First Foundations}

Modern theoretical physics is overwhelmingly formulated in a state-first language. Classical mechanics, quantum mechanics, and quantum field theory all begin with the specification of a state on a spatial hypersurface and prescribe rules for its evolution in time. Even path-integral formulations, which sum over histories, ultimately encode transition amplitudes between states defined at temporal boundaries. Within this paradigm, time evolution is fundamental, analytic continuation is assumed to be well-defined, and causality is encoded through microscopic prescriptions such as time ordering and the $+i\epsilon$ rule.

Despite its empirical success, this framework encounters persistent difficulties when applied to gravity at short distances. Perturbative quantization of General Relativity produces nonrenormalizable divergences. Attempts to regulate these divergences through higher-derivative terms introduce modes that appear to violate unitarity or stability. Efforts to preserve locality, microcausality, and spectral positivity simultaneously have repeatedly failed. These problems are often treated as technical obstacles requiring new degrees of freedom, additional symmetries, or radical reformulations.

This paper advances a different diagnosis. We argue that these difficulties do not arise from the specific form of the gravitational action, but from a deeper mismatch between the state-first ontology and the structural role gravity plays in physical theory. In particular, gravity is unique in that it governs the geometry of spacetime itself, which already functions as a bookkeeping structure for causal and dynamical relations. Treating geometry as a state evolving in time may therefore be conceptually inverted.

We propose an ontological reversal. Rather than taking states and their evolution as primitive, we take the fundamental object to be a space of histories, together with a set of admissibility constraints that determine which histories are physically allowed. Dynamics, causality, and locality then emerge as properties of effective interface theories that compress admissibility information into local variables suitable for macroscopic description.

Within this architecture, Quadratic Gravity will emerge as a necessary consequence of enforcing ultraviolet admissibility on a local metric interface. It will not be introduced as a conjectural quantum gravity proposal, but will appear as the inevitable form taken by geometry once the interface is required to remain well-defined at arbitrarily fine scales.

\section{Ontological Reversal: From State Evolution to Admissible History}

We begin by formalizing the ontological shift that underlies the remainder of the paper. The central move is to replace the notion of time-evolving states with that of globally admissible histories.

\subsection{Histories as Fundamental Objects}

Let $M$ denote a spacetime manifold and let $\Phi$ denote the space of field configurations defined over $M$. We make the following definition.

\paragraph{Definition 1 (History Space).}
A history $H$ is a map
\[
H : M \to \Phi,
\]
assigning to each spacetime point a field configuration. The space of all histories is denoted by
\[
\mathcal{H} = \{ H : M \to \Phi \}.
\]

This definition is intentionally broad. No equations of motion are imposed at this stage. A history is not required to satisfy locality, causality, or any dynamical law. It is simply a complete assignment of physical quantities over spacetime.

\subsection{Admissibility Constraints}

Physical law enters through admissibility rather than evolution.

\paragraph{Definition 2 (Admissibility Constraint).}
An admissibility constraint is a functional
\[
C : \mathcal{H} \to \{0,1\},
\]
where $C(H)=1$ indicates that the history $H$ is admissible and $C(H)=0$ indicates that it violates consistency conditions. The admissible subspace of histories is
\[
\mathcal{H}_{\mathrm{adm}} = \{ H \in \mathcal{H} \mid C(H)=1 \}.
\]

The functional $C$ may encode local constraints, global consistency requirements, boundary conditions, conservation laws, or other structural restrictions. Importantly, $C$ is not assumed to factorize over spacetime points, nor is it assumed to be expressible as the integral of a local density.

\subsection{Irreversibility as a Primitive Feature}

A crucial distinction between admissibility constraints and ordinary equations of motion is irreversibility.

\paragraph{Definition 3 (Irreversible Constraint).}
An admissibility constraint $C$ is said to be irreversible if there exists a history $H \in \mathcal{H}$ such that
\[
C(H)=0,
\]
and for any local modification $\delta H$ supported on a bounded region $R \subset M$,
\[
C(H + \delta H)=0.
\]
That is, once the constraint is violated, no local change confined to any finite region can restore admissibility.

This captures formally the idea that admissibility violations accumulate and cannot be undone by later adjustments. Examples include violations of global consistency, entropy-decreasing trajectories, or topological obstructions.

\subsection{Irreversibility and Time Reversal}

Irreversibility has immediate consequences for temporal symmetry.

\paragraph{Lemma 1 (Irreversibility Breaks Time-Reversal Symmetry).}
If an admissibility constraint $C$ is irreversible, then $C$ is not invariant under time reversal. That is, there exists a history $H$ such that
\[
C(H) \neq C(T[H]),
\]
where $T$ denotes the time-reversal operator acting on histories.

\paragraph{Proof.}
Suppose $C$ is irreversible and let $H$ be a history such that $C(H)=0$ due to a violation occurring at some spacetime region that is temporally localized. By irreversibility, no local modification can restore admissibility. Consider the time-reversed history $T[H]$. For $C$ to be invariant under $T$, the violation would have to be removable by modifications occurring after the reversed violation point, effectively ``repairing'' the history by local changes. This contradicts irreversibility. Therefore $C(H) \neq C(T[H])$. $\square$

This lemma makes precise why time-reversal symmetry cannot be fundamental in a constraint-first ontology. Any framework that presupposes microscopic reversibility must either forbid irreversible constraints or relegate them to external boundary conditions. In contrast, we take irreversibility as primitive and treat reversibility, when observed, as an emergent property of restricted classes of histories.

\subsection{Consequences for Dynamical Frameworks}

The shift from state evolution to admissible histories has immediate implications. There is no fundamental notion of a Hamiltonian generating time evolution. Hilbert spaces, when they appear, are tools for organizing subsets of admissible histories rather than ontological arenas. Analytic continuation in time, which relies on reversible evolution and reflection positivity, is not guaranteed to exist globally.

These consequences will become essential in later sections, where we examine the spectral and causal structure of local interface theories. At this stage, the key point is conceptual: once admissibility rather than evolution is fundamental, many assumptions built into standard quantum field theory lose their axiomatic status and must instead be derived or abandoned.

\section{Irreversibility and the Breakdown of Microscopic Analyticity}

\subsection*{Assumptions and Scope}

In this section we assume the ontology developed in Section 2. In particular, we take as given that physical law is specified by a space of admissible histories subject to irreversible constraints, and that time-reversal symmetry is not fundamental. No assumptions are made about quantization, particle content, or the form of any effective action. The goal of this section is to identify which structural features of standard quantum field theory are incompatible with irreversibility at the ontological level.

\subsection{Analyticity in State-First Theories}

In conventional relativistic quantum field theory, analyticity is not merely a technical convenience but a structural principle. The analyticity properties of correlation functions encode causality, stability, and spectral positivity. These properties are tightly interwoven and are most clearly expressed through the Källén--Lehmann spectral representation.

Let $\phi(x)$ be a scalar field operator in a Lorentz-invariant vacuum. The Wightman two-point function
\[
W(x-y) = \langle 0 | \phi(x)\phi(y) | 0 \rangle
\]
admits the representation
\[
W(x-y) = \int_0^\infty d\mu(m^2)\,\rho(m^2)\,\Delta_+(x-y;m^2),
\]
where $\Delta_+(x-y;m^2)$ is the positive-frequency Wightman function for mass $m$, and $\rho(m^2)$ is the spectral density. Lorentz invariance, positivity of the Hilbert space inner product, and completeness of asymptotic states imply that $\rho(m^2)\ge 0$ (Källén and Lehmann 1957).

The time-ordered Feynman propagator
\[
D_F(x-y) = \langle 0 | T\{\phi(x)\phi(y)\} | 0 \rangle
\]
is obtained by analytic continuation and admits the momentum-space form
\[
D_F(k) = \int_0^\infty d\mu(m^2)\,\frac{\rho(m^2)}{k^2 - m^2 + i\epsilon}.
\]
This representation encodes several assumptions simultaneously. First, all excitations propagate forward in time with positive energy. Second, the $+i\epsilon$ prescription selects a unique arrow of causality. Third, the analyticity domain of $D_F$ in complexified momentum space is such that Wick rotation to Euclidean signature is well-defined.

These properties are often treated as axiomatic, but in fact they follow from the deeper assumption that the theory is built on a reversible, state-first foundation.

\subsection{Irreversibility and Spectral Positivity}

We now examine the compatibility of these assumptions with a constraint-first ontology.

\paragraph{Theorem 1 (Incompatibility of Irreversibility with Källén--Lehmann Positivity).}
If admissibility constraints are irreversible in the sense of Definition 3, then the two-point correlation function of any local interface field cannot satisfy a Källén--Lehmann spectral representation with nonnegative spectral density.

\paragraph{Proof Sketch.}
The Källén--Lehmann representation presupposes that correlation functions are invariant under time reversal up to complex conjugation, since $\rho(m^2)$ arises from a sum over intermediate states weighted by squared matrix elements. Positivity of $\rho$ is equivalent to the statement that probabilities are preserved under time-reversed evolution.

Irreversible admissibility constraints violate this symmetry. By Lemma 1, there exist histories $H$ such that admissibility depends on temporal ordering. Any local interface field $\phi(x)$ that compresses admissibility information must therefore encode this asymmetry in its correlation functions. Consequently, the associated two-point function cannot be invariant under $t \mapsto -t$ in the manner required for reflection positivity.

Reflection positivity is a necessary condition for $\rho(m^2)\ge 0$. If it fails, then either the spectral density must take indefinite sign, or the analytic structure of the propagator must be modified so that the standard spectral representation does not apply. In either case, the Källén--Lehmann form with positive $\rho$ is incompatible with irreversibility. $\square$

\subsection{Corollary: Structural Status of Microcausality}

\paragraph{Corollary 1.}
In any local interface theory compressing irreversible admissibility constraints, violations of microscopic microcausality are structurally required rather than pathological.

\paragraph{Justification.}
Microcausality in standard quantum field theory follows from the analyticity domain of correlation functions and the spectral condition. Since irreversibility forces a breakdown of at least one of these assumptions, strict microcausality cannot be maintained at arbitrarily small scales. However, this does not imply macroscopic acausality. Interface theories are only required to reproduce causal ordering at scales where coarse-graining remains valid. Beyond that scale, causal violations signal the breakdown of the interface description, not the breakdown of physical law.

\subsection{Irreversibility and the Failure of Wick Rotation}

A closely related consequence concerns Wick rotation. The standard Euclidean path integral relies on the substitution $t \mapsto -i\tau$, which assumes that the action and correlation functions are analytic in time and that boundary conditions can be rotated consistently. Reflection positivity ensures that the Euclidean theory corresponds to a unitary Lorentzian theory upon continuation.

In a constraint-first ontology, admissibility itself depends on causal ordering. The distinction between past and future is not a gauge choice but a structural feature of the constraint functional $C$. As a result, there is no guarantee that admissible histories in Lorentzian signature map to admissible configurations in Euclidean signature. The Euclidean action obtained by naive Wick rotation may fail to encode the same admissibility conditions.

This observation sheds light on the long-standing difficulties encountered in Euclidean formulations of higher-derivative gravity. Instabilities of the Euclidean action are often interpreted as signs of fundamental inconsistency. From the present perspective, they instead reflect the misuse of Euclidean continuation in a context where irreversibility forbids it. The Lorentzian theory remains well-defined, but its Euclidean image does not faithfully represent admissibility.

\subsection{Interpretive Consequences}

The results of this section establish a crucial point. Once irreversibility is taken as ontologically primitive, several pillars of conventional quantum field theory lose their foundational status. Spectral positivity, strict microcausality, and unrestricted analytic continuation are no longer guaranteed. Their failure at microscopic scales is not a defect but a necessary consequence of compressing irreversible global constraints into local variables.

This conclusion prepares the ground for the next step. If local interface theories must violate some of the assumptions underlying standard analyticity, we must ask how such theories can nevertheless remain predictive and approximately local at macroscopic scales. The answer lies in coarse-graining and interface construction, to which we now turn.

\section{Coarse-Graining Admissibility and Interface Theories}

\subsection*{Assumptions and Scope}

In this section we assume the constraint-first ontology developed in Sections 2 and 3. In particular, we assume that admissibility is a global property of histories and that irreversibility is fundamental. No assumption is made that admissibility constraints are local, Markovian, or expressible as equations of motion. The purpose of this section is to show that locality and predictability at macroscopic scales nevertheless require the existence of effective interface theories that compress admissibility information into local variables.

\subsection{Why Coarse-Graining Is Necessary}

Admissibility constraints, as defined in Section 2, act on entire histories. In general, determining whether a history is admissible may require access to correlations across spacetime regions that are arbitrarily distant or temporally separated. Such global dependence is incompatible with finite observers, finite computation, and finite prediction horizons.

Consider an observer located at a spacetime point $p \in M$ attempting to predict observables within their future light cone. If admissibility depends on information outside any finite neighborhood of $p$, then prediction would require evaluating the constraint functional $C(H)$ over the entire spacetime manifold. This is not merely impractical but logically impossible within finite time. Therefore, any framework that permits prediction by local agents must approximate global admissibility using locally accessible data.

This necessity is independent of any particular dynamics. It arises purely from the coexistence of global constraints and local observation. The resolution is coarse-graining: the replacement of exact admissibility by an approximate, locally computable surrogate that is accurate within a specified domain of validity.

\subsection{Definition of an Interface Theory}

We now formalize this notion.

\paragraph{Definition 4 (Interface Theory).}
An interface theory for a space of admissible histories $\mathcal{H}_{\mathrm{adm}}$ is a triple
\[
\mathcal{T}_{\mathrm{int}} = (\Phi_{\mathrm{eff}}, \mathcal{L}_{\mathrm{eff}}, \varepsilon_{\max}),
\]
where $\Phi_{\mathrm{eff}}$ is a set of effective fields defined locally on spacetime, $\mathcal{L}_{\mathrm{eff}}$ is a local Lagrangian density constructed from $\Phi_{\mathrm{eff}}$ and its derivatives, and $\varepsilon_{\max}$ is a coarse-graining scale such that for all histories $H \in \mathcal{H}_{\mathrm{adm}}$ whose variations occur on length scales much larger than $\varepsilon_{\max}$, the Euler--Lagrange equations derived from $\mathcal{L}_{\mathrm{eff}}$ approximate the admissibility-induced behavior of $H$ to prescribed accuracy.

Several aspects of this definition deserve emphasis. First, an interface theory is approximate by construction. It is not required to encode admissibility exactly, only to do so reliably within a specified regime. Second, the fields $\Phi_{\mathrm{eff}}$ need not correspond to fundamental observables; they may contain redundancy or gauge freedom. Third, the scale $\varepsilon_{\max}$ is not a regulator to be removed, but a boundary beyond which the interface description ceases to be faithful.

\subsection{Locality Requires Compression}

The existence of an interface theory is not optional if prediction is to be possible.

\paragraph{Lemma 2 (Locality Requires Compression).}
If admissibility is a global property of histories and observers have access only to local information, then any predictive framework must compress global admissibility into local effective variables.

\paragraph{Proof.}
Suppose no such compression exists. Then determining whether a local configuration near a point $p$ can be extended into an admissible history would require evaluating $C(H)$ over regions of spacetime that are inaccessible to the observer at $p$. In that case, the observer cannot determine which local evolutions are physically allowed, and prediction becomes impossible even in principle. Therefore, predictability implies the existence of local variables whose behavior approximates admissibility constraints within the observer’s domain. These variables constitute an interface theory. $\square$

This lemma is purely logical. It does not assume locality of the underlying ontology, only locality of observation and prediction. It therefore applies equally to classical, quantum, deterministic, and stochastic settings.

\subsection{Redundancy and Gauge Freedom as Interface Signatures}

Because interface theories compress global information into local fields, they necessarily involve redundancy. Multiple configurations of the effective fields may correspond to the same admissible history. This redundancy is not a flaw but a structural feature of compression.

Gauge freedom is the canonical example. In gauge theories, the fundamental admissible objects are gauge-invariant quantities such as Wilson loops. The local gauge field $A_\mu(x)$ provides a convenient parameterization of these objects, but many different $A_\mu$ configurations correspond to the same physical history. Gauge redundancy reflects the fact that the interface field is not fundamental but representational.

This observation will later play a key role in interpreting apparent pathologies of gravitational interface theories. Features that appear unphysical when interpreted as properties of fundamental degrees of freedom may be benign or even necessary when interpreted as artifacts of compression.

\subsection{Examples of Interface Theories}

The abstract structure introduced above is already familiar in several physical domains.

In thermodynamics and hydrodynamics, the admissible histories are microscopic phase-space trajectories satisfying Hamiltonian dynamics. The admissibility constraints include conservation laws and, at a macroscopic level, entropy increase. The interface variables are fields such as temperature, pressure, and entropy density. These variables do not encode microscopic detail, nor do they evolve reversibly. Instead, they provide a local summary of admissibility that enables prediction within a restricted regime. The second law of thermodynamics, which imposes an irreversible constraint on histories, is exactly captured by local hydrodynamic equations despite being incompatible with microscopic reversibility.

In gauge theory, admissibility is determined by gauge-invariant quantities. The local connection $A_\mu(x)$ is an interface field that enables perturbative calculations and local dynamics, even though it is not itself observable. The existence of gauge redundancy reflects the fact that $A_\mu$ is a compression of nonlocal admissibility data into a local field.

These examples illustrate two general lessons. First, interface theories routinely encode irreversible or nonlocal constraints using local fields. Second, apparent violations of fundamental principles at the level of interface variables often reflect the limitations of the compression rather than inconsistencies in the underlying ontology.

\subsection{Toward a Geometric Interface}

The discussion so far has been entirely general. We have not assumed Lorentz invariance, covariance, or even spacetime geometry. We have shown only that if admissibility is global and observers are local, then some form of local interface theory must exist.

In relativistic physics, additional structure is imposed. Observables must transform covariantly, and predictions must respect the causal structure of spacetime at macroscopic scales. These requirements severely restrict the form that an interface theory can take. In the next section, we show that once covariance is imposed, spacetime geometry itself emerges as the unique viable interface variable.

\section{The Metric as a Compression Channel}

\subsection*{Assumptions and Scope}

In this section we assume the existence of an interface theory in the sense of Definition~4, together with the requirement that the interface respect Lorentz covariance and locality at macroscopic scales. No assumption is made that spacetime geometry is fundamental. The purpose of this section is to show that, once covariance and local predictability are imposed, the spacetime metric arises as the unique viable interface variable, and that the Einstein--Hilbert action emerges as the leading-order local penalty on admissibility violations.

\subsection{Covariance and the Need for a Geometric Interface}

An interface theory intended to summarize admissibility in relativistic physics must satisfy two basic conditions. First, it must be local in spacetime, so that predictions can be made using data available in finite regions. Second, it must be covariant, so that no preferred coordinate system or foliation is introduced at the interface level.

Local scalar interface variables, such as a single density field, are insufficient to encode causal structure. Vector fields introduce preferred directions unless supplemented by additional structure. Higher-rank tensor fields could in principle encode more information, but without a notion of distance or causal separation they cannot serve as a universal interface for admissibility.

The interface must therefore encode information about the relative separation of events and the penalties associated with deforming histories in spacetime. This requirement naturally leads to a metric structure.

\subsection{Uniqueness of the Metric Interface}

We now formalize this claim.

\paragraph{Theorem 2 (Uniqueness of the Metric as Covariant Interface).}
Suppose an interface theory satisfies the following conditions:
(i) it is local and defined pointwise on spacetime,
(ii) it is Lorentz covariant,
(iii) it encodes relative separation and causal structure between nearby events,
and (iv) it couples universally to all forms of energy and momentum.
Then the interface variable must be a pseudo-Riemannian metric $g_{\mu\nu}$.

\paragraph{Proof.}
Condition (iii) requires the interface to define an invariant notion of infinitesimal separation between spacetime points, which mathematically corresponds to a bilinear form on the tangent space. Lorentz covariance restricts this bilinear form to have signature $(+,-,-,-)$ up to convention. Condition (iv) requires universal coupling, which in relativistic physics is mediated by the energy--momentum tensor $T^{\mu\nu}$, a symmetric rank-two tensor. Consistency of coupling then requires the interface variable to be a symmetric rank-two tensor as well. These conditions uniquely specify a pseudo-Riemannian metric. $\square$

This result does not assert that the metric is fundamental. It asserts only that any local covariant compression of admissibility information must take the form of a metric field. Geometry, in this sense, is an interface necessity rather than an ontological primitive.

\subsection{Curvature as a Measure of Admissibility Deformation}

Once the metric is identified as the interface variable, admissibility penalties must be constructed from geometric quantities derived from it. Because admissibility concerns how histories deform relative to one another, the relevant quantities must measure deviations from locally flat, minimally constrained configurations.

In differential geometry, such deviations are captured by curvature. Flat spacetime corresponds to the absence of geometric penalties, while nonzero curvature encodes the cost of maintaining admissibility under deformation. Covariance restricts admissibility penalties to scalars constructed from the Riemann curvature tensor and its contractions.

\subsection{Leading-Order Admissibility Penalty}

We now determine the leading-order local scalar that can appear in an interface action.

\paragraph{Lemma 3 (Einstein--Hilbert Action as Minimal Interface Penalty).}
The unique local, generally covariant scalar constructed from the metric and its derivatives involving at most two derivatives is the Ricci scalar $R$. Therefore, the leading-order interface action takes the form
\[
S_{\mathrm{EH}} = \frac{1}{16\pi G} \int d^4x\,\sqrt{-g}\,R,
\]
up to a cosmological constant term.

\paragraph{Proof.}
Locality and covariance restrict admissibility penalties to scalar functions of the metric and its derivatives. At zeroth order in derivatives one obtains only a constant, corresponding to a cosmological constant. At first order no covariant scalars exist. At second order the only nontrivial scalar constructed from the curvature is the Ricci scalar $R$. All other candidates either vanish identically, reduce to total derivatives, or involve higher derivatives. $\square$

The normalization of the action is fixed by matching to the Newtonian limit, ensuring that the interface reproduces known large-scale behavior.

\subsection{Interpretation of General Relativity}

This derivation places General Relativity in a precise structural role. The Einstein--Hilbert action is not postulated as a fundamental law of spacetime dynamics. Rather, it is the lowest-order local functional capable of penalizing inadmissible geometric deformations while respecting covariance and locality.

This interpretation resolves a common tension. General Relativity is empirically successful and conceptually robust at macroscopic scales, yet fails as a quantum theory in the ultraviolet. From the present perspective, this is exactly what one should expect. The leading-order interface is accurate when admissibility violations are mild and slowly varying. It is not designed to regulate arbitrarily fine-grained fluctuations.

The statement that General Relativity is nonrenormalizable can therefore be sharpened. What fails is not gravity per se, but the assumption that the lowest-order interface remains valid at all scales. The absence of ultraviolet damping reflects the absence of higher-order admissibility penalties in the interface action.

\subsection{Limits of the First-Order Interface}

Nothing in the derivation of the Einstein--Hilbert action guarantees ultraviolet admissibility. Indeed, the action contains only two derivatives of the metric, which implies that high-frequency fluctuations are insufficiently suppressed. When histories are summed over with increasing resolution, the interface description loses control.

This failure is not optional. It follows directly from dimensional analysis and locality. If the interface is to remain local and metric-only, additional derivative terms must appear. The next section formalizes this requirement and shows that it uniquely fixes the form of the ultraviolet extension of the metric interface.

\section{Ultraviolet Admissibility and the Necessity of Curvature-Squared Terms}

\subsection*{Assumptions and Scope}

In this section we assume that spacetime geometry has emerged as a local covariant interface field in the form of a metric $g_{\mu\nu}$, and that the Einstein--Hilbert action provides the leading-order admissibility penalty. We do not assume quantization, particle interpretation, or any particular regularization scheme. The goal is to determine what additional structure is required if the metric interface is to remain well-defined under arbitrarily fine-grained history summation.

\subsection{Ultraviolet Admissibility}

The failure of the Einstein--Hilbert action at short distances is often described as a problem of perturbative nonrenormalizability. From the present perspective, this language obscures the underlying issue. What fails is not perturbation theory per se, but admissibility under refinement.

\paragraph{Definition 5 (Ultraviolet Admissibility).}
A metric interface theory is said to be ultraviolet admissible if it satisfies the following conditions:
\begin{enumerate}
\item High-frequency metric fluctuations are dynamically suppressed rather than amplified.
\item No new light degrees of freedom are introduced below the interface cutoff scale.
\item The interface remains local and covariant; admissibility penalties are constructed from local curvature invariants.
\end{enumerate}

The first condition ensures that history summation does not diverge as resolution increases. The second preserves minimal compression, preventing the interface from proliferating representational variables. The third enforces the interface principle established in Sections 4 and 5.

\subsection{Failure of the Einstein--Hilbert Interface}

Consider perturbations of the metric about flat spacetime,
\[
g_{\mu\nu} = \eta_{\mu\nu} + h_{\mu\nu}.
\]
At quadratic order in $h_{\mu\nu}$, the Einstein--Hilbert action yields kinetic terms schematically of the form
\[
S_{\mathrm{EH}}^{(2)} \sim \int d^4k\, h_{\mu\nu}(-k)\, k^2 \, h^{\mu\nu}(k).
\]
The resulting propagator behaves as $1/k^2$ at large momentum. Consequently, fluctuations with arbitrarily large $k$ contribute without sufficient suppression. From the admissibility standpoint, this means that increasingly fine geometric distortions are penalized only weakly, leading to instability of the interface description.

This behavior is not an accident but a direct consequence of dimensional analysis. The Einstein--Hilbert action contains exactly two derivatives of the metric. No local counterterm of the same derivative order can alter the ultraviolet scaling.

\subsection{Minimal Extension Principle}

To restore ultraviolet admissibility while preserving locality and minimal field content, the interface action must include higher-derivative terms constructed solely from the metric.

\paragraph{Lemma 4 (Minimal Extension).}
The minimal modification of the metric interface that improves ultraviolet behavior without introducing new fields is the inclusion of terms quadratic in curvature.

\paragraph{Proof.}
Any local covariant scalar constructed from the metric and its derivatives must be built from curvature tensors. Terms linear in curvature reproduce the Einstein--Hilbert action. To introduce additional derivatives without new fields, one must consider invariants quadratic in curvature, which involve four derivatives of the metric. Higher powers of curvature introduce additional suppression but are not required for minimal ultraviolet admissibility. $\square$

This lemma formalizes the intuition that curvature-squared terms are the first admissible correction once the leading-order interface fails.

\subsection{Classification of Curvature-Squared Invariants}

In four spacetime dimensions, there are three independent curvature-squared scalars:
\[
R^2, \quad R_{\mu\nu}R^{\mu\nu}, \quad R_{\mu\nu\rho\sigma}R^{\mu\nu\rho\sigma}.
\]
However, these are not all dynamically independent. The Gauss--Bonnet identity implies that the combination
\[
\chi = \int d^4x\,\sqrt{-g}\left(
R_{\mu\nu\rho\sigma}R^{\mu\nu\rho\sigma}
-4R_{\mu\nu}R^{\mu\nu}
+R^2
\right)
\]
is topological and does not contribute to the equations of motion in four dimensions.

\paragraph{Theorem 3 (Uniqueness of Quadratic Curvature Interface).}
Up to total derivatives, the most general local covariant ultraviolet-admissible extension of the Einstein--Hilbert interface in four dimensions is
\[
S = \int d^4x\,\sqrt{-g}\left(
\frac{1}{16\pi G}R
+ \alpha R^2
+ \beta C_{\mu\nu\rho\sigma}C^{\mu\nu\rho\sigma}
\right),
\]
where $C_{\mu\nu\rho\sigma}$ is the Weyl tensor and $\alpha,\beta$ are dimensionless constants.

\paragraph{Proof.}
Using the Gauss--Bonnet identity, one may eliminate $R_{\mu\nu\rho\sigma}R^{\mu\nu\rho\sigma}$ in favor of $R^2$ and $R_{\mu\nu}R^{\mu\nu}$. The latter may be further decomposed into a Weyl-squared term plus a Ricci scalar contribution. No other independent curvature-squared scalars exist in four dimensions. $\square$

This action is precisely the one known in the literature as Quadratic Gravity. Here, however, it has been derived as a necessity of ultraviolet admissibility rather than introduced as a candidate theory.

\subsection{Power-Counting and Ultraviolet Behavior}

We now demonstrate explicitly how curvature-squared terms improve ultraviolet behavior.

Expanding the quadratic action about flat spacetime yields kinetic terms of the schematic form
\[
S^{(2)} \sim \int d^4k\, h_{\mu\nu}(-k)
\left(
k^2 + \frac{k^4}{M^2}
\right)
h^{\mu\nu}(k),
\]
where $M$ is a mass scale determined by $\alpha$ and $\beta$. Inverting this operator gives a propagator that behaves as
\[
G(k) \sim \frac{1}{k^2} - \frac{1}{k^2 - M^2}.
\]
At large momentum, the propagator falls as $1/k^4$. This behavior suppresses high-frequency fluctuations and renders loop integrals power-counting renormalizable.

From the admissibility perspective, this suppression ensures that histories with rapidly varying geometry are exponentially penalized, stabilizing the interface under refinement.

\subsection{Interpretive Summary}

The argument of this section establishes a sharp result. If spacetime geometry is to function as a local covariant interface for admissibility, and if that interface is to remain meaningful at arbitrarily short distances without introducing new fields or nonlocal structure, then curvature-squared terms are unavoidable. Quadratic Gravity is therefore not an optional modification of General Relativity. It is the unique minimal ultraviolet completion of the metric interface consistent with the principles established earlier.

What remains is to examine the dynamical and spectral consequences of this structure. In particular, we must understand how the $1/k^4$ behavior required for admissibility manifests in propagators, and why it inevitably leads to apparent violations of spectral positivity. This is the subject of the next section.

\section{Spectral Consequences and the Emergence of the Merlin Mode}

\subsection*{Assumptions and Scope}

In this section we assume the existence of a local covariant metric interface whose action contains curvature-squared terms as derived in Sections 6 and 7. No interpretation in terms of particles or asymptotic states is assumed a priori. The goal is to analyze the spectral and causal implications of ultraviolet admissibility at the level of propagators and to show that the appearance of ghost-like modes is a necessary interface artifact rather than a physical pathology.

\subsection{Quadratic Propagators and Pole Structure}

Consider the quadratic expansion of the curvature-squared interface action about flat spacetime,
\[
g_{\mu\nu} = \eta_{\mu\nu} + h_{\mu\nu}.
\]
After gauge fixing, the spin-2 sector of the quadratic action yields a momentum-space operator of the schematic form
\[
\mathcal{O}(k^2) \sim k^2 - \frac{k^4}{M^2},
\]
where $M$ is a mass scale set by the coefficients of the curvature-squared terms. The corresponding propagator is
\[
G(k^2) = \frac{1}{k^2 - k^4/M^2}.
\]
This expression can be decomposed algebraically as
\[
G(k^2) = \frac{1}{k^2} - \frac{1}{k^2 - M^2}.
\]
The first term corresponds to the familiar massless graviton pole. The second term corresponds to an additional massive pole with an opposite overall sign.

This decomposition is exact at the level of the free quadratic action and is independent of quantization scheme. The relative minus sign is not an arbitrary choice; it is required to produce the ultraviolet $1/k^4$ falloff that enforces admissibility under refinement.

\subsection{Failure of the Källén--Lehmann Representation}

We now examine the spectral implications of this propagator structure.

\paragraph{Theorem 4 (Violation of Källén--Lehmann Positivity).}
The propagator
\[
G(k^2) = \frac{1}{k^2} - \frac{1}{k^2 - M^2}
\]
cannot be represented in the Källén--Lehmann form with a nonnegative spectral density.

\paragraph{Proof.}
A Källén--Lehmann representation would require
\[
G(k^2) = \int_0^\infty d\mu(m^2)\,\frac{\rho(m^2)}{k^2 - m^2 + i\epsilon},
\]
with $\rho(m^2)\ge 0$. However, the decomposition above implies a spectral density of the form
\[
\rho(m^2) = \delta(m^2) - \delta(m^2 - M^2),
\]
which is manifestly indefinite. Therefore no representation with $\rho(m^2)\ge 0$ exists. $\square$

This result is sometimes interpreted as fatal, since spectral positivity is often equated with unitarity and stability. However, as shown in Section 3, spectral positivity presupposes microscopic time-reversal symmetry and analytic continuation properties that are incompatible with irreversible admissibility constraints.

\subsection{Advanced Propagation and the $-i\epsilon$ Prescription}

The physical meaning of the minus sign becomes clearer when one examines the causal prescription. The propagator may be written more precisely as
\[
G(k^2) = \frac{1}{k^2 + i\epsilon} - \frac{1}{k^2 - M^2 - i\epsilon}.
\]
The second pole carries the opposite $i\epsilon$ prescription. This implies that, at short distances, the associated mode propagates with advanced rather than retarded boundary conditions.

In standard state-first quantum field theory, such behavior is interpreted as a violation of causality. In the present framework, it reflects the fact that admissibility constraints propagate bidirectionally in time when compressed into a local interface. The interface theory must encode both past-to-future and future-to-past consistency requirements. When forced into a single-arrow formalism, this bidirectionality appears as a reversed-causal regulator mode.

\subsection{Interpretation of the Extra Pole}

It is common to refer to the massive pole with the wrong sign as a ghost. This terminology is misleading in the present context. The mode does not correspond to a negative-energy asymptotic state, nor does it appear in the physical spectrum of observable excitations. Rather, it serves as a regulator that cancels ultraviolet divergences arising from the massless graviton sector.

We adopt the terminology \emph{Merlin mode} to emphasize this distinction. Like the mythical figure who ages backward in time, the Merlin mode propagates positive energy in the reverse temporal direction relative to ordinary fields. It is not an independent physical degree of freedom but a bookkeeping artifact required to maintain admissibility under local compression.

\subsection{Connection to Irreversibility}

The emergence of the Merlin mode is not an isolated technical feature. It is the concrete realization of the abstract result obtained in Section 3. There we showed that irreversible admissibility constraints forbid standard analytic structure at microscopic scales. Here we see exactly how that prohibition manifests when geometry is used as an interface.

The presence of a reversed-causal pole ensures that ultraviolet fluctuations are suppressed without introducing nonlocality or additional fields. It simultaneously enforces admissibility and violates the assumptions underlying spectral positivity. This is not a failure of the theory but a consistency check: the interface behaves exactly as required given the ontological premises.

\subsection{Macroscopic Causality}

Finally, it is important to emphasize that the Merlin mode decouples at low energies. At momenta $k^2 \ll M^2$, the propagator reduces to the ordinary $1/k^2$ form, and standard causal behavior is recovered. Apparent acausality is confined to distances comparable to the inverse cutoff scale, beyond which the interface description is no longer reliable.

Thus, macroscopic causality is preserved even though microscopic analyticity is relaxed. This distinction mirrors that seen in other interface theories, such as hydrodynamics, where local violations of microscopic reversibility are essential for macroscopic consistency.

\subsection{Transition}

At this stage, all structural elements are in place. We have shown that irreversible admissibility necessitates a local metric interface, that ultraviolet admissibility forces curvature-squared terms, and that such terms inevitably introduce reversed-causal regulator modes. What remains is to identify this derived structure with the established theory known as Quadratic Gravity and to demonstrate that its apparent pathologies have already been resolved when interpreted correctly.

\section{Quadratic Gravity as a Derived Interface of Irreversible Constraint}

\subsection{Framing: What Is Being Derived}

This section does not propose a new candidate theory of quantum gravity, nor does it attempt to select among existing approaches. Its purpose is instead structural. We identify the precise form that a local metric field theory must take when a constraint-first, irreversible ontology is compressed into a geometric interface subject to ultraviolet admissibility.

The derivation target is deliberately narrow. We ask what form a local, covariant metric interface must assume if it is to satisfy the following three conditions simultaneously. First, it must provide ultraviolet suppression of high-frequency geometric fluctuations, rendering history summation well-defined under refinement. Second, it must preserve unitarity for all asymptotic, macroscopic states accessible to observers. Third, it must be permitted to violate at least one assumption ordinarily taken as axiomatic in textbook quantum field theory, such as spectral positivity or microscopic microcausality.

The central claim of this section is that, once these conditions are imposed and the ontological commitments of the preceding sections are accepted, the structure known as Quadratic Gravity is not optional. It is the unique minimal local metric interface compatible with irreversible admissibility.

\subsection{Constraint-First Ontology and Interface Status}

We briefly recall the ontological commitments established earlier. Physical law is specified not by time-evolving states but by a space of admissible histories selected by global consistency constraints. Irreversibility is primitive: admissibility is not invariant under time reversal, and violations accumulate rather than cancel. As shown in Section~3, this immediately precludes the universal validity of microscopic analyticity and the Källén--Lehmann spectral representation.

Nevertheless, admissibility must admit a local description at macroscopic scales. Observers experience approximately local, covariant physics, and any viable interface theory must encode admissibility in a form suitable for prediction within finite regions. As established in Sections~4 and~5, the spacetime metric arises as the unique covariant compression channel capable of performing this role. Geometry is therefore not fundamental substance, but a lossy summary of admissibility satisfaction.

\subsection{The Metric Interface and Curvature}

Once the metric is identified as the interface variable, admissibility penalties must be constructed from local curvature invariants. At leading order in derivatives, there exists a unique scalar invariant: the Ricci scalar $R$. An action proportional to $R$ therefore represents the lowest-order geometric compression of admissibility, recovering General Relativity as the first nontrivial interface theory.

However, as emphasized in Sections~6 and~7, this first-order interface fails in the ultraviolet. The Einstein--Hilbert action provides no suppression of high-curvature fluctuations. In a sum over histories, short-distance geometric configurations contribute without adequate damping, and admissibility becomes unstable under refinement. This failure is not a flaw of General Relativity as a classical theory, but a signal that the metric interface is being pushed beyond the regime in which its lowest-order form remains faithful.

\subsection{Local Covariant Regularization}

Ultraviolet admissibility imposes strict constraints on how the interface may be extended. Locality and covariance restrict admissible corrections to scalars constructed from curvature tensors and their contractions. Introducing nonlocal operators would violate the interface principle, while introducing new fields would enlarge the representational channel beyond minimal compression.

In four spacetime dimensions, the most general local action containing terms up to second order in curvature may be written as
\[
S = \int d^4x \sqrt{-g}
\left(
\frac{2}{\kappa^2} R
+ \alpha R^2
+ \beta R_{\mu\nu}R^{\mu\nu}
+ \gamma R_{\mu\nu\rho\sigma}R^{\mu\nu\rho\sigma}
\right).
\]
The Gauss--Bonnet identity implies that a particular linear combination of the quadratic terms is topological and does not affect the equations of motion in four dimensions. As a result, the action may be reorganized uniquely into a basis consisting of $R^2$ and the square of the Weyl tensor $C_{\mu\nu\rho\sigma}C^{\mu\nu\rho\sigma}$, up to surface terms.

\paragraph{Lemma.}
In four dimensions, the unique minimal local covariant extension of the Einstein--Hilbert interface that provides ultraviolet damping without introducing new fields is quadratic in curvature and may be expressed in terms of $R^2$ and $C_{\mu\nu\rho\sigma}C^{\mu\nu\rho\sigma}$.

This result is purely structural. It relies only on locality, covariance, and ultraviolet admissibility, and does not presuppose quantization or perturbative analysis.

\subsection{Spectral Consequences of Quadratic Curvature}

Expanding the quadratic-curvature action about flat spacetime and inverting the resulting quadratic form yields propagators that fall as $1/k^4$ at large momentum. This ultraviolet behavior is precisely what renders the theory power-counting renormalizable, as originally shown by Stelle. However, it immediately violates the assumptions underlying the Källén--Lehmann spectral representation.

As shown in Section~8, no propagator with $1/k^4$ asymptotics can admit a positive-definite spectral density without additional poles or sign reversals. The failure of spectral positivity is therefore not accidental. It is forced by the requirement of ultraviolet damping in a local metric theory and is a direct reflection of the breakdown of microscopic analyticity predicted by irreversibility.

\subsection{The Merlin Mode as a Reversed-Causal Regulator}

The quadratic-curvature interface introduces an additional pole in the spin-two sector of the graviton propagator. In a naive particle interpretation, this pole corresponds to a massive ghost with negative norm or negative energy. Such an interpretation implicitly assumes that all poles represent asymptotic particle states evolving forward in time.

A more careful analysis shows that this assumption fails. The additional pole behaves as a time-reversed unstable resonance. Its propagator is the complex conjugate of an ordinary resonance propagator, corresponding to positive-energy propagation backward in time. Such modes do not appear as asymptotic states and therefore do not represent physical degrees of freedom. Instead, they function as regulator channels that enforce ultraviolet admissibility while preserving unitarity of observable processes.

Within the present framework, this behavior is expected. The metric interface is attempting to encode irreversible global constraints using local mathematical structures that are themselves time-symmetric. The resulting mismatch manifests as a reversed-causal mode. The so-called ghost is therefore not an ontological entity but an interface artifact required by local compression.

\subsection{Unitarity Without Microscopic Analyticity}

Despite the violation of spectral positivity and microscopic microcausality, unitarity of asymptotic states is preserved. This follows from the fact that only stable asymptotic states contribute to unitarity relations. The reversed-causal regulator mode is unstable and does not appear in the asymptotic spectrum. As a result, it does not enter the unitarity sum, a conclusion that can be demonstrated explicitly using the largest-time formalism.

The cost of this consistency is the loss of naive microscopic analyticity. Amplitudes exhibit modified analytic structure, and causal ordering becomes ambiguous at Planckian scales. From the constraint-first perspective developed here, this is not a defect but an expected signal that the geometric interface has reached the boundary of its domain of validity.

\subsection{Conclusion}

We may now summarize the derivation. If the underlying ontology is constraint-first, history-bound, and irreversible, then any local covariant compression of admissibility into geometry must take the form of a metric theory constructed from curvature invariants. General Relativity emerges as the leading-order interface. Ultraviolet admissibility forces the inclusion of curvature-squared terms. In four dimensions, this extension is unique and yields the theory conventionally known as Quadratic Gravity.

The unusual spectral and causal features of this theory are not pathologies to be eliminated but interface artifacts to be interpreted correctly. Quadratic Gravity is therefore not chosen among alternatives. It is the necessary form taken by local geometry when it is forced to encode irreversible constraints beyond the scale at which geometry remains a faithful representation of the underlying ontology.

\section{Conclusion: Geometry at the Edge of Admissibility}

This work has pursued a deliberately architectural aim. Rather than proposing a new dynamical law or advocating a particular quantum gravity program, we have asked a narrower and more structural question: what form must a local metric theory take if it is required to function as an interface compressing a deeper, irreversible, constraint-first ontology into a form suitable for macroscopic prediction?

The answer, developed step by step, is unexpectedly rigid. Beginning from the replacement of state evolution with admissible histories, we showed that irreversibility is not an emergent accident but a primitive feature of admissibility itself. Once irreversibility is taken seriously, several assumptions commonly treated as axiomatic in quantum field theory—microscopic time-reversal symmetry, spectral positivity, and unrestricted analyticity—lose their foundational status. They become approximate properties of effective descriptions, valid only insofar as the interface remains faithful to the underlying constraint structure.

From this starting point, locality and predictability force the introduction of interface theories. These interfaces do not reproduce admissibility exactly; they compress it. Redundancy, gauge freedom, and limited domains of validity are therefore not pathologies but structural signatures of compression. When relativistic covariance is imposed, spacetime geometry emerges as the unique interface variable capable of summarizing admissibility in a local, observer-accessible form.

Within this framework, General Relativity appears naturally as the leading-order interface theory. The Einstein–Hilbert action is not postulated as a fundamental dynamical principle but derived as the lowest-order local scalar penalty on geometric deformation. Its empirical success at macroscopic scales is thus explained, while its failure in the ultraviolet is rendered unsurprising. The first-order interface lacks the capacity to suppress arbitrarily fine-grained geometric fluctuations and therefore cannot remain admissible under refinement.

Ultraviolet admissibility then sharply constrains the form of permissible corrections. If locality, covariance, and minimal field content are preserved, curvature-squared terms are unavoidable. In four dimensions, this extension is unique up to topological terms. The resulting theory is precisely what is conventionally called Quadratic Gravity. Here, however, it is not introduced as a speculative modification of General Relativity but derived as the minimal ultraviolet completion of the metric interface.

The spectral and causal peculiarities of Quadratic Gravity follow immediately. Ultraviolet damping requires propagators that fall faster than $1/k^2$, which in turn forbids a positive-definite Källén–Lehmann representation. The appearance of an additional spin-two pole with reversed causal prescription is therefore not an anomaly to be repaired but a regulator artifact required by interface compression. The so-called ghost does not represent a physical asymptotic degree of freedom; it is a reversed-causal mode enforcing admissibility while preserving unitarity of observable processes.

This reinterpretation resolves several long-standing objections. Ostrogradsky instability, negative-energy states, and unitarity violation arise only if interface artifacts are misidentified as fundamental ontology. Once the distinction between admissibility and interface is maintained, these objections dissolve. What remains is a controlled, localized violation of microscopic microcausality at scales where the geometric interface itself ceases to be faithful—a feature that is not only acceptable but arguably expected in any theory attempting to describe fluctuating spacetime.

It is important to emphasize what has not been claimed. We have not shown that Quadratic Gravity is the final theory of quantum gravity, nor that it is valid nonperturbatively or at arbitrarily high curvature. Indeed, the present framework suggests that further breakdowns of the geometric interface are likely once curvature approaches the regulator scale, and that new interface variables may then be required. What we have shown is more limited and more robust: if one insists on a local metric description at all, and if one accepts irreversible admissibility as fundamental, then Quadratic Gravity is not a choice among many but the inevitable form that geometry takes at the edge of its own validity.

From this perspective, disagreements between quantum gravity programs may be reframed. The central divide is not technical but ontological. Approaches that preserve state-first evolution, microscopic analyticity, and strict causality must abandon locality or introduce extended degrees of freedom. Approaches that preserve locality and minimal field content must relax microscopic analyticity and accept regulator modes. Quadratic Gravity occupies a precise position within this landscape, dictated not by taste but by structural necessity.

More broadly, the analysis suggests a shift in emphasis for fundamental physics. Rather than asking which dynamics govern states, we may ask which constraints render histories admissible, and how those constraints are compressed into the interfaces through which observers experience the world. Geometry, on this view, is not the arena of physics but one of its most successful—and ultimately limited—compression schemes.

In this sense, Quadratic Gravity is not the end of the story. It is the point at which geometry reveals the cost of representing irreversible constraint locally. That cost is paid in the currency of modified causality, altered analyticity, and regulator modes. Recognizing these features as structural rather than pathological may be essential for any coherent understanding of gravity at the smallest scales.

\section*{Appendix} 
% ==================================================
\appendix
\section{Technical Appendix: Linearization, Propagators, Poles, and the Origin of the Merlin Mode}
% ==================================================

\subsection{Conventions and Quadratic Action}

We work in four spacetime dimensions with metric signature $(+,-,-,-)$ and expand about Minkowski space,
\[
g_{\mu\nu}=\eta_{\mu\nu}+\kappa\,h_{\mu\nu},
\]
with $\kappa^2=32\pi G$. Indices are raised and lowered with $\eta_{\mu\nu}$ and partial derivatives are taken with respect to inertial coordinates. The fundamental local action considered in the main text is the curvature-squared extension of Einstein gravity,
\begin{equation}
S=\int d^4x\,\sqrt{-g}\left(\frac{2}{\kappa^2}R+\frac{1}{6f_0^2}R^2-\frac{1}{2\xi^2}C_{\mu\nu\rho\sigma}C^{\mu\nu\rho\sigma}\right),
\label{eq:Squad}
\end{equation}
which is equivalent up to surface terms to an $R^2$ term plus a specific linear combination of $R_{\mu\nu}R^{\mu\nu}$ and $R^2$. This basis is convenient because the Weyl-squared term isolates the spin-two sector and the $R^2$ term isolates the scalar sector (Stelle 1977; Donoghue and Menezes 2019).

To obtain the free propagator one expands \eqref{eq:Squad} to quadratic order in $h_{\mu\nu}$, adds a covariant gauge-fixing term, and inverts the resulting quadratic operator in momentum space.

\subsection{Linearized Curvatures}

To first order in $h_{\mu\nu}$, the Christoffel symbol is
\[
\Gamma^{(1)\rho}_{\mu\nu}=\frac{\kappa}{2}\eta^{\rho\sigma}\left(\partial_\mu h_{\nu\sigma}+\partial_\nu h_{\mu\sigma}-\partial_\sigma h_{\mu\nu}\right).
\]
The linearized Riemann tensor is
\[
R^{(1)\rho}{}_{\sigma\mu\nu}
=\partial_\mu \Gamma^{(1)\rho}_{\nu\sigma}-\partial_\nu \Gamma^{(1)\rho}_{\mu\sigma},
\]
so the linearized Ricci tensor and scalar are
\begin{align}
R^{(1)}_{\mu\nu}
&=\frac{\kappa}{2}\left(\partial_\rho\partial_\mu h^\rho{}_\nu+\partial_\rho\partial_\nu h^\rho{}_\mu-\Box h_{\mu\nu}-\partial_\mu\partial_\nu h\right),
\label{eq:RicciLin}\\
R^{(1)}
&=\kappa\left(\partial_\mu\partial_\nu h^{\mu\nu}-\Box h\right),
\label{eq:Rlin}
\end{align}
where $h=\eta^{\mu\nu}h_{\mu\nu}$ and $\Box=\eta^{\mu\nu}\partial_\mu\partial_\nu$.

In momentum space with $\partial_\mu\mapsto i k_\mu$, these become algebraic in $k_\mu$.

\subsection{Spin Projectors and Gauge Fixing}

The inversion of the quadratic operator is simplest using the standard spin projectors for symmetric rank-two tensors. Define the transverse and longitudinal projectors
\[
\theta_{\mu\nu}=\eta_{\mu\nu}-\frac{k_\mu k_\nu}{k^2},\qquad
\omega_{\mu\nu}=\frac{k_\mu k_\nu}{k^2},
\]
where $k^2=\eta^{\mu\nu}k_\mu k_\nu$.

The spin-2 and spin-0 (scalar) projectors acting on symmetric tensors are
\begin{align}
(P^{(2)})_{\mu\nu,\rho\sigma}
&=\frac{1}{2}\left(\theta_{\mu\rho}\theta_{\nu\sigma}+\theta_{\mu\sigma}\theta_{\nu\rho}\right)-\frac{1}{3}\theta_{\mu\nu}\theta_{\rho\sigma},
\label{eq:P2}\\
(P^{(0)})_{\mu\nu,\rho\sigma}
&=\frac{1}{3}\theta_{\mu\nu}\theta_{\rho\sigma}.
\label{eq:P0}
\end{align}
There are also spin-1 and longitudinal projectors which depend on $\theta$ and $\omega$. They are gauge artifacts and drop out of gauge-invariant matrix elements; nevertheless they appear in the propagator in a covariant gauge.

We impose de Donder gauge (harmonic gauge) with gauge-fixing functional
\[
F_\mu=\partial^\nu h_{\mu\nu}-\frac{1}{2}\partial_\mu h,
\]
and add
\[
S_{\text{gf}}=-\frac{1}{2\alpha}\int d^4x\, F_\mu F^\mu.
\]
The parameter $\alpha$ controls the gauge; physical observables are independent of $\alpha$.

\subsection{Quadratic Operator and Propagator Structure}

After expanding the action \eqref{eq:Squad} to quadratic order and transforming to momentum space, the quadratic form can be written schematically as
\[
S^{(2)}=\frac{1}{2}\int\frac{d^4k}{(2\pi)^4}\, h^{\mu\nu}(-k)\,\mathcal{O}_{\mu\nu,\rho\sigma}(k)\,h^{\rho\sigma}(k),
\]
with $\mathcal{O}$ diagonal in the spin-projector basis. The key point is that the Einstein--Hilbert piece contributes $\sim k^2$ and the curvature-squared pieces contribute $\sim k^4$ in the spin-2 and spin-0 channels. One can therefore write
\begin{equation}
\mathcal{O}(k)=
\left(\frac{2}{\kappa^2}\,k^2-\frac{1}{\xi^2}\,k^4\right)P^{(2)}
+\left(-\frac{4}{\kappa^2}\,k^2+\frac{1}{3f_0^2}\,k^4\right)P^{(0)}
+\text{(gauge-dependent sectors)}.
\label{eq:OperatorDiag}
\end{equation}
The precise coefficients in the scalar channel depend on conventions and on the treatment of surface terms; the essential fact is the presence of both $k^2$ and $k^4$ pieces with fixed relative signs determined by the choice of tachyon-free action (Stelle 1977; Donoghue and Menezes 2019).

Inverting \eqref{eq:OperatorDiag} yields the propagator
\begin{equation}
iD_{\mu\nu,\rho\sigma}(k)
=i\left[
\frac{P^{(2)}}{\frac{2}{\kappa^2}k^2-\frac{1}{\xi^2}k^4}
+\frac{P^{(0)}}{-\frac{4}{\kappa^2}k^2+\frac{1}{3f_0^2}k^4}
\right]
+\text{(gauge-dependent terms)}.
\label{eq:FullProp}
\end{equation}
The ultraviolet behavior of each physical projector channel is $D\sim 1/k^4$, which is the origin of power-counting renormalizability (Stelle 1977).

\subsection{Pole Decomposition in the Spin-Two Sector}

The spin-two denominator in \eqref{eq:FullProp} is of the form
\[
\frac{2}{\kappa^2}k^2-\frac{1}{\xi^2}k^4
=\frac{2}{\kappa^2}k^2\left(1-\frac{\kappa^2}{2\xi^2}k^2\right).
\]
Define the spin-two mass scale
\begin{equation}
M_2^2=\frac{2\xi^2}{\kappa^2}.
\label{eq:M2def}
\end{equation}
Then the spin-two propagator factor is
\begin{equation}
\frac{i}{k^2-\frac{k^4}{M_2^2}}
=\frac{i}{k^2\left(1-\frac{k^2}{M_2^2}\right)}
=\frac{i}{k^2}-\frac{i}{k^2-M_2^2}.
\label{eq:PartialFraction}
\end{equation}
The relative minus sign in \eqref{eq:PartialFraction} is forced by the requirement that the high-energy falloff be $1/k^4$. It is this sign that obstructs the usual positive spectral representation and motivates the ``ghost'' terminology in naive treatments.

In the framework emphasized by Donoghue and Menezes, the correct interpretation is not that the second term represents a negative-energy asymptotic particle. Rather, once interactions are included, the would-be massive pole becomes an unstable resonance with a width whose analytic structure is the complex conjugate of the usual one (Donoghue and Menezes 2019). This is the origin of the reversed-causal ``Merlin'' behavior.

\subsection{Including Self-Energy: Location and Form of the Resonance}

Coupling to ordinary matter fields generates a spin-two self-energy $\Pi_2(k^2)$ and shifts the denominator schematically to
\[
k^2-\frac{k^4}{M_2^2}-\Pi_2(k^2).
\]
Above the relevant thresholds, $\Pi_2(k^2)$ develops an imaginary part. Near the resonance one can write an effective Breit--Wigner-like form
\begin{equation}
iD^{(2)}(k)\;\approx\;\frac{-i}{k^2-M_2^2-i\gamma},
\label{eq:MerlinBW}
\end{equation}
with $\gamma>0$ (Donoghue and Menezes 2019). Two unusual signs appear relative to a conventional unstable particle: the overall minus sign in the numerator and the sign of the imaginary part in the denominator. Taken together they produce the correct sign of the absorptive part needed for unitarity in physical amplitudes, while reversing the causal interpretation of the intermediate resonance.

\subsection{Why K\"all\'en--Lehmann Positivity Must Fail}

In ordinary local quantum field theory under standard assumptions, a two-point function admits the K\"all\'en--Lehmann representation
\begin{equation}
D(k)=\frac{1}{\pi}\int_0^\infty ds\,\frac{\rho(s)}{k^2-s+i\epsilon},
\label{eq:KL}
\end{equation}
with $\rho(s)\ge 0$ (K\"all\'en and Lehmann 1957). Expanding \eqref{eq:KL} at large $k^2$ gives
\[
D(k)=\frac{1}{k^2}\left(\frac{1}{\pi}\int_0^\infty ds\,\rho(s)\right)
+\mathcal{O}\!\left(\frac{1}{k^4}\right),
\]
so unless the zeroth moment of $\rho$ vanishes, the propagator cannot fall faster than $1/k^2$. With $\rho(s)\ge 0$, the zeroth moment cannot vanish for a nontrivial theory. Therefore a genuine $1/k^4$ falloff implies that at least one assumption behind \eqref{eq:KL} must be violated. In gauge theories the presence of unphysical states modifies the argument in gauge-dependent channels, but the spin-two sector of higher-derivative gravity exhibits the same obstruction as non-gauge higher-derivative theories, indicating a genuine departure from standard microcausal analyticity rather than a pure gauge artifact (Donoghue and Menezes 2019).

Coleman emphasized that higher-derivative theories admit modified spectral representations with additional complex poles that allow cancellations producing $1/k^4$ behavior (Coleman 1974; Coleman 1969). This provides a consistent mathematical mechanism by which ultraviolet damping can coexist with unitarity, at the cost of altered analyticity and microscopic causality.

\subsection{Time-Ordered Propagator and Reversed-Causal Interpretation}

The causal interpretation is most transparent in the time-ordered Green function after performing the $k^0$ contour integral. Consider for simplicity the scalar prototype with a propagator of the form
\[
D(k)=\frac{-i}{(k^0)^2-\omega_{\mathbf{k}}^2-i\gamma},
\qquad
\omega_{\mathbf{k}}=\sqrt{\mathbf{k}^2+M^2},
\]
which captures the relevant sign structure of the spin-two resonance \eqref{eq:MerlinBW}. The time-ordered propagator in position space is
\[
D(t,\mathbf{x})=\int\frac{d^4k}{(2\pi)^4}\,e^{-ik^0 t+i\mathbf{k}\cdot\mathbf{x}}\,D(k).
\]
The pole prescription implied by \eqref{eq:MerlinBW} places the poles at
\[
k^0=\pm \omega_{\mathbf{k}}-i\,\frac{\gamma}{2\omega_{\mathbf{k}}}
\]
rather than the conventional
\[
k^0=\pm \omega_{\mathbf{k}}+i\,\frac{\gamma}{2\omega_{\mathbf{k}}}.
\]
As a result, closing the contour for $t>0$ picks up the pole structure corresponding to a contribution that behaves as
\[
e^{+i\omega_{\mathbf{k}} t}\,e^{-\gamma t/(2\omega_{\mathbf{k}})}
\]
rather than $e^{-i\omega_{\mathbf{k}} t}e^{-\gamma t/(2\omega_{\mathbf{k}})}$. The sign flip in the oscillatory phase is precisely what is meant by positive energy propagating in the opposite time direction. The decay factor remains exponentially damped, which is why the resonance does not exhibit runaway growth even though its causal interpretation is reversed. This is the Lee--Wick mechanism in its modern resonance form (Lee and Wick 1969; Lee and Wick 1970; Donoghue and Menezes 2019).

\subsection{A Minimal Unitarity Check in the Resonant Channel}

A concise way to see how unitarity survives is to consider a partial-wave amplitude near a resonance. For a conventional narrow resonance one has
\[
T(s)\approx \frac{g^2}{s-M^2+iM\Gamma}.
\]
In the present case the propagator-level structure \eqref{eq:MerlinBW} yields an amplitude with the schematic form
\begin{equation}
T(s)=\frac{A(s)}{f(s)-iA(s)},
\label{eq:TunitForm}
\end{equation}
where $A(s)$ is real on the physical cut and $f(s)$ is real near the resonance. Then
\[
\operatorname{Im}T(s)=\frac{A(s)^2}{f(s)^2+A(s)^2},
\qquad
|T(s)|^2=\frac{A(s)^2}{f(s)^2+A(s)^2},
\]
so
\[
\operatorname{Im}T(s)=|T(s)|^2,
\]
which is the elastic unitarity condition in a normalization where phase space factors are absorbed into $A(s)$. The crucial point is that the absorptive part remains positive even though the resonance pole is the complex conjugate of the usual one; the unusual numerator and denominator signs conspire to preserve the sign of the discontinuity across the cut (Donoghue and Menezes 2019). The general proof for diagrams uses the largest-time formalism and the principle that only stable asymptotic states appear in the unitarity sum (Veltman 1963).

\subsection{On Contours, Analyticity, and the Lee--Wick Prescription}

Although the on-shell unitarity relations can be verified at the level of discontinuities, reproducing the same result directly from Feynman integrals can require a deformation of integration contours in loop momentum space. The historical name for the needed prescription is the Lee--Wick contour (Lee and Wick 1969; Cutkosky, Landshoff, Olive, and Polkinghorne 1969). The operational meaning is that the contour must be chosen so that the contributions from complex poles are accounted for in a manner consistent with the causal assignment implicit in the propagator. In ordinary QFT the $+i\epsilon$ prescription already encodes a unique arrow of causality and thereby fixes contours. In higher-derivative theories with reversed-causal resonances there are effectively dueling causal assignments, and the contour choice becomes an additional piece of structure rather than a derived consequence. This is the precise technical sense in which microcausal analyticity is sacrificed while unitarity is retained.

\subsection{What This Appendix Establishes}

The derivations above isolate the structural mechanism behind the main text. Local curvature-squared regularization forces $1/k^4$ ultraviolet behavior. That behavior forces a departure from the assumptions behind the standard positive spectral representation. In a Lorentzian formulation, the resulting extra spin-two structure behaves as a reversed-causal unstable resonance rather than a negative-energy asymptotic particle.

This is the technical content of the claim that the Merlin mode is an interface regulator channel: it enforces ultraviolet admissibility while preserving unitarity of observable asymptotic processes, at the cost of modified analyticity and microscopic causality (Stelle 1977; Lee and Wick 1969; Veltman 1963; Donoghue and Menezes 2019).

\begin{thebibliography}{99}

\bibitem{Stelle1977}
K.~S.~Stelle.
\newblock Renormalization of Higher Derivative Quantum Gravity.
\newblock \emph{Physical Review D}, 16:953--969, 1977.

\bibitem{DonoghueMenezes2019}
J.~F.~Donoghue and G.~Menezes.
\newblock On Quadratic Gravity.
\newblock \emph{Physical Review D}, 100:105006, 2019.

\bibitem{Donoghue2012}
J.~F.~Donoghue.
\newblock The Effective Field Theory Treatment of Quantum Gravity.
\newblock \emph{AIP Conference Proceedings}, 1483:73--94, 2012.

\bibitem{KallenLehmann1957}
G.~Källén and H.~Lehmann.
\newblock On the Definition of Propagators.
\newblock \emph{Il Nuovo Cimento}, 5:1592--1620, 1957.

\bibitem{LeeWick1969}
T.~D.~Lee and G.~C.~Wick.
\newblock Negative Metric and the Unitarity of the S-Matrix.
\newblock \emph{Nuclear Physics B}, 9:209--243, 1969.

\bibitem{LeeWick1970}
T.~D.~Lee and G.~C.~Wick.
\newblock Finite Theory of Quantum Electrodynamics.
\newblock \emph{Physical Review D}, 2:1033--1048, 1970.

\bibitem{Coleman1974}
S.~Coleman.
\newblock Secret Symmetry of Massive Quantum Field Theories.
\newblock \emph{Physical Review D}, 10:2491--2499, 1974.

\bibitem{ColemanAcausality}
S.~Coleman.
\newblock Acausality.
\newblock In \emph{Proceedings of the International School of Subnuclear Physics},
  Erice, 1969.

\bibitem{Ostrogradsky1850}
M.~Ostrogradsky.
\newblock Mémoire sur les équations différentielles relatives au problème des
  isopérimètres.
\newblock \emph{Mémoires de l'Académie Impériale des Sciences de
  Saint-Pétersbourg}, VI:385--517, 1850.

\bibitem{PaisUhlenbeck1950}
A.~Pais and G.~E.~Uhlenbeck.
\newblock On Field Theories with Non-Localized Action.
\newblock \emph{Physical Review}, 79:145--165, 1950.

\bibitem{Veltman1963}
M.~Veltman.
\newblock Unitarity and Causality in a Renormalizable Field Theory with Unstable
  Particles.
\newblock \emph{Physica}, 29:186--207, 1963.

\bibitem{Cutkosky1969}
R.~E.~Cutkosky, P.~V.~Landshoff, D.~I.~Olive, and J.~C.~Polkinghorne.
\newblock A Non-Analytic S-Matrix.
\newblock \emph{Nuclear Physics B}, 12:281--300, 1969.

\bibitem{Anselmi2018}
D.~Anselmi.
\newblock Fakeons and Lee--Wick Models.
\newblock \emph{Journal of High Energy Physics}, 2018(2):141, 2018.

\bibitem{Menezes2021}
G.~Menezes.
\newblock Unitarity of Higher-Derivative Theories via Modern Amplitudes.
\newblock \emph{Physical Review D}, 103:065021, 2021.

\end{thebibliography}

\end{document}
